\documentclass[a4paper]{article}
\usepackage{fontspec} 
\usepackage{polyglossia}
\setmainlanguage{russian} 
\setotherlanguage{english}
\newfontfamily{\cyrillicfont}{Times New Roman}

\usepackage{mathtools}
\usepackage{fullpage}
\usepackage[utf8x]{inputenc}
\usepackage{amsmath}
\usepackage[colorinlistoftodos]{todonotes}


\title{Введение в вычислительную математику}
\author{MIPT DIHT}
\begin{document}
\maketitle

\section{Билет №1}
\subsection{Численное дифференцирование. Простейшие формулы численного дифференцирования.}
Функция заменяется таблицей значений. Производная $$\frac{df}{dx} \lim_{h \to 0} \frac{f(x + h) - f(x)}{h}$$ заменяется $$f'(x) \approx \frac{f(x + h) - f(x)}{h}$$
Пусть функция $f(x)$ имеет достаточное число производных, требуется вычислить ее производную $f'(x)$ в данной точке х. Задачу отыскания h содержащую предельный переход, можно заменить приближенно задачами вычисления по одной из формул
$$f'(x) \approx \frac{f(x + h) - f(x)}{h}$$
$$f'(x) \approx \frac{f(x) - f(x - h)}{h}$$
$$f'(x) \approx \frac{f(x + h) - f(x - h)}{2h}$$
Для замены производной f"(x) можно воспользоваться формулой
$$f''(x) \approx \frac{f(x + h) - 2f(x) + f(x - h)}{h^2}$$

\subsection{Метод неопределенных коэффициентов для вывода формул численного дифференцирования.}
\subsection{Порядок аппроксимации формул численного дифференцирования.}
$\triangleright$ \textbf{Def:} Порядок аппроксимации – показатель степени h в главном члене погрешности 
\subsection{Оптимальный шаг сетки численного дифференцирования.}

\section{Билет №2}
\subsection{Постановка задачи интерполяции.}
Заданы точки $x=x_0 \ldots x=x_n$, значения в этих точках $f(x_0) \ldots f(x_n)$ и таблица значений ф-ции f(x) в этих точках
Задача: найти алгебраический интерполяционный многочлен $P_n(x,f,x_0 \ldots x_n): \ P_n(x) = c_nx^n + \ldots +c_0$, принимающий в узлах заданные значения и степени не выше n
\subsection{Полиномиальная интерполяция, ее существование и единственность.}
$\triangleright$ \textbf{Th:} Пусть заданы $x_0 \neq \ldots \neq x_n,\ f(x_0) \ldots f(x_n). \ \exists !$ многочлен $P_n(x) = P_n(x,f,x_0 \ldots x_n)$ степени не выше n, принимающий в заданных узлах заданные значения \\
	$\circ$ Сначала покажем, что существует единственный, затем построим.
	Если бы было два многочлена $P_n^1(x),\ P_n^2(x) \rightarrow R_n(x) = P_n^1(x)-P_n^2(x)$ обращается в ноль в n+1 точке. Но $\forall R_n(x) \not\equiv 0$ имеет столько корней, какова его степень $\rightarrow R_n(x) \equiv 0 \rightarrow P_n^1(x)=P_n^2(x)$.
	Введем вспомогательные многочлены $$l_k(x) = \frac{(x-x_0) \ldots (x-x_{k-1})(x-x_{k+1}) \ldots (x-x_n)}{(x_k-x_0) \ldots (x_k-x_{k-1})(x_k-x_{k+1}) \ldots (x_k-x_n)}$$ 
	$$l_k(x_j)=\begin{cases} 1,  & \mbox{if }x_j = x_k\\ 0, & \mbox{if }x_j \neq x_k \end{cases}$$
	$$P_n(x) = P_n(x,f,x_0 \ldots x_n) = f(x_0)l_0(x) + \ldots + f(x_n)l_n(x)$$ – искомый интерполяционный многочлен. Каждое слагаемое степени не выше n $\rightarrow$ весь многочлен степени не выше n и очевидно выполнено $P_n(x_j)=f(x_j)$
	$\bullet$
\subsection{Интерполяционный полином в форме Лагранжа.}
	$$P_n(x) = P_n(x,f,x_0 \ldots x_n) = f(x_0)l_0(x) + \ldots + f(x_n)l_n(x)$$ из док-ва

\section{Билет №3}
\subsection{Постановка задачи интерполяции.}
Заданы точки $x=x_0 \ldots x=x_n$, значения в этих точках $f(x_0) \ldots f(x_n)$ и таблица значений ф-ции f(x) в этих точках
Задача: найти алгебраический интерполяционный многочлен $P_n(x,f,x_0 \ldots x_n): \ P_n(x) = c_nx^n + \ldots +c_0$, принимающий в узлах заданные значения и степени не выше n
\subsection{Разделенные разности.}
Пусть f(x) в точках $x_a, x_b, x_c \ldots$ принимает $f(x_a), f(x_b), f(x_c) \ldots$. $f(x_k)$ – разностное отношение нулевого порядка в точке. Разностное отношение первого порядка: $$ f(x_k,x_t) = \frac{f(x_t) - f(x_k)}{x_t - x_k}$$
Разностное отношение n-1 порядка: $$ f(x_0 \ldots x_n) = \frac{f(x_1 \ldots x_n) - f(x_0 \ldots x_{n-1})}{x_n - x_0}$$
$\triangleright$ \textbf{Свойства}
\begin{enumerate}
	\item $P_n(x,f,x_0 \ldots x_n) = P_{n-1}(x,f,x_0 \ldots x_{n-1}) + f(x_0 \ldots x_n)(x-x_0)\ldots (x-x_{n-1})$
	\item $P_n(x,f,x_0 \ldots x_n) = c_nx^n + \ldots + c_0$, т е $f(x_0 \ldots x_n) = c_n$
	\item $f(x_0 \ldots x_n)=0 \iff f(x_0) \ldots f(x_n)$ – значения $Q_m, m<n$
\end{enumerate}
\subsection{Интерполяционный полином в форме Ньютона.}
Интерполяционный многочлен в форме Ньютона: $$P_n(x, f, x_0 \ldots x_n) = f(x_0) + (x-x_0)f(x_0,x_1) + \ldots + (x-x_0)\ldots (x-x_{n-1})f(x_0 \ldots x_n) $$
\subsection{Обусловленность задачи интерполяции и константа Лебега.}
Пусть узлы интерполяции $x_0 \ldots x_n$ лежат на $a \leq x \leq b$, пусть $f(x_0) \ldots f(x_n)$ заданные числа. Соответствующий интерп. мн-н: $P_n(x) = P_n(x,f,x_0 \ldots x_n) = P_n(x,f)$. Приданим значениям $f(x_j)$ возмущение $\delta f(x_j), \ P_n(x,f) \rightarrow P_n(x, f + \delta f) = P_n(x,f) + P_n(x, \delta f)$.
Мера чувствительности интерп. мн-на к возмущению – наименьшее $L_n: \ \forall \delta f$ $$\max_{x \in (a,b)} |P_n(x,\delta f)| \leq L_n \max_{j} |\delta f(x_j)|$$
$L_n = L_n(x_0 \ldots x_n,a,b)$ – константы Лебега. В случае равномерно расположенных узлов: $$2^{n-1} > L_n > 2^{n-3} \frac{1}{\sqrt{n-1}} \frac{1}{n-3/2} $$ т е чувствительность рез-та интерполяции резко возрастат при росте n  

\section{Билет №4}
\subsection{Теорема об остаточном члене интерполяционного полинома (с доказательством)}
\subsection{Оценка остаточного члена на равномерной сетке.}

\section{Билет №5}
\subsection{Полиномы Чебышева, их свойства.}
$$ T_k(x) = cos(k\varphi) = cos(k)arccos(x) \ (cos(\varphi)=x)$$
$\triangleright$ \textbf{Th:} $T_k(x)$ – многочлены степени k=0,1,... При этом $T_0(x) = 1, T_1(x) = x, \ldots$
$$ T_{k+1}(x) = 2x T_k(x) - T_{k-1}(x) $$
	$\circ$ 
	Очевидно: $T_0(x) = cos(0) = 1; \ T_1(x) = cos(arccos(x)) = x$
	$$cos (k+1)\varphi = 2 cos\varphi cos k\varphi - cos(k-1)\varphi$$, что в случае $\varphi=arccosx$ переходит в формулу из теоремы.
	Докажем что многочлен – степени k с помощью индукции по k. k=0, k=1 уже доказано. Фиксируем $k \ge 1$ . Пусть доказано для $T_j(x)$. Тогда обе части рекуррентной формулы есть многочлен степени k+1
	$\bullet$
\subsection{Применение полиномов Чебышева при построении узлов интерполяции.}
$$ Q_n(cos \varphi, sin\varphi, F) = \sum_{k=0}^n a_kcos k\varphi \equiv P_n(x,f), \ P_n(x,f) = \sum_{k=0}^n a_k T_k(x)$$
$$ a_0 = \frac{1}{n+1} \sum_{m=0}^n f_m = \frac{1}{n+1} \sum_{m=0}^n f_mT_0(x_m)$$
$$ a_k = \frac{2}{n+1} \sum_{m=0}^n f_m cos k\varphi_m = \frac{2}{n+1} \sum_{m=0}^n f_mT_k(x_m) $$
Т е $P_n(x,f)$ - алгебраический многочлен степени не выше n, принимающий в узлах интерполяции $x_m = cos\varphi_m$ заданные значения $f(x_m) = f_m$ \\
Отметим, что точки $$\varphi_m = \frac{\pi}{n+1}m + \frac{\pi}{2(n+1)} $$ являются нулями $cos(n+1)\varphi$, а $x_m=cos\varphi_m$ нулями $T_{n+1}(x) = cos(n+1)\varphi$. Т. е. интерп. мн-н $P_n(x,f)$ реализует алгебр. интерп. ф-ции в точках, являющихся нулями многочлена Чебышева


\section{Билет №6}
\subsection{Понятие сплайн-интерполяции}

$\triangleright$ \textbf{Def:} На отрезке [a,b] задана система узловых точек $\{t_n\}, n=0..N$. Сплайн $S_m(t)$ - определенная на [a,b] функция, имеющая l непрерывных проиводных и являющаяся на каждом интервале $[t_{n-1}, t_n]$ многочленом степени m \\
$\triangleright$ \textbf{Def:} Дефект сплайна d=m-l - разность между степенью сплайна и показателем его гладкости \\
$\triangleright$ \textbf{Def:} Кубический сплайн дефекта 1, интерполирующий заданную ф-цию на [a,b] - ф-ция S(t):
\begin{enumerate}
	\item $S(t_n) = f(t_n)$
	\item $S(t) \in C^2[a,b]$
	\item $\forall [t_{n-1},t_n] S(t)$ является кубическим многочленом
	\item Краевые условия \begin{enumerate}
		\item S'(a) = f'(a), S'(b) = f'(b)
		\item S''(a) = f''(a), S''(b) = f''(b)
		\item S(a) = S(b); S'(a) = S'(b)
	\end{enumerate}
\end{enumerate}

\subsection{Процедура построения кубического сплайна показателя гладкости два.}
$\triangleright$ \textbf{Th:} Интерполяционный кубический сплайн, удовлетворяющий условим 1-3 и одному из условий 4 существует и единственен\\
	$\circ$ 
	Пусть S(z) на каждом отрезке представлен как 
	$$ S(z) = f_n(1-z)^2(1+2z) +f_{n+1}z^2(3-2z) + m_n\tau_nz(1-z)^2 -m_{n+1}\tau_nz^2(1-z)$$
	где $\tau_n = t_{n+1}-t_n, \ z = \frac{t-t_n}{\tau_n}, \ m_n = S'(t_n) \rightarrow$
	$$ S''(t) = \frac{(f_{n+1} - f_n)(6-12z)}{\tau_n^2} + m_n \frac{6z-12}{\tau_n} + m_{n+1} \frac{6z-2}{\tau_n}$$
	$$ S''(t_n+0) = 6 \frac{f_{n+1}-f_n}{\tau_n^2} - \frac{4m_n}{\tau_n} - \frac{2m_{n+1}}{\tau_n}$$
	$$ S''(t_n-0) = -6 \frac{f_n-f_{n-1}}{\tau_{n-1}^2} + \frac{2m_{n-1}}{\tau_{n-1}} + m_{n+1}\frac{4m_n}{\tau){n-1}}$$
	Условие непрерывности второй производной:
	$$r_nm_{n-1}+2m_n+s_nm_{n+1}=c_n$$
	$$ c_n=3 \big(s_n\frac{f_{n+1}-f_n}{\tau_n}+r_n\frac{f_n-f_{n-1}}{\tau_{n-1}}, \ s_n=\frac{\tau_{n-1}}{\tau_{n-1}+\tau_n},\ r_n=1-s_n$$
	После добавления краевых условий получаем систему N+1 уравнений
	\begin{itemize}
		\item Для краевых условий 1-го типа:
		$$ m_0=f_0'$$
		$$ f_nm_{n-1} + 2m_n + s_nm_{n+1} = c_n$$
		$$ m_N=f_N'$$
		\item Для краевых условий 2-го типа (заданы производные)
		$$ 2m_0 + m_1 = 3\frac{f_1-f_0}{\tau_0} - \frac{\tau_0}{2}f_0'' $$
		$$ r_nm_{n-1} + 2m_n + s_nm_{n+1} = c_n $$
		$$ m_{N-1} + 2m_N = 3\frac{f_N-f_{N-1}}{\tau_{N-1}} + \frac{\tau_{N-1}}{2}f_N'' $$
	\end{itemize}
	Аналогично для третьего типа краевых уравнений. Во всех случаях матрицы трехдиагональные, симметричные, положительно определенные => решение СЛАУ существует и единственно
	$\bullet$

\section{Билет №7}
\subsection{Численное интегрирование.}
$$a=x_0<x_1\ldots <x_n=b$$
$$P_k(x)=P_k(x,f,x_0 \ldots x_n)$$ – на каждом отрезке многочлен степени не выше k. Приближенное рав-во для вычисление интеграла
$$ \int_{a}^b f(x)dx \approx \int_{a}^b P_k(x,f,x_0 \ldots x_n) dx = \int_{a}^{x_1} P_k dx + \int_{x_1}^{x_2} P_k dx + \ldots \int{x_{n-1}}^b P_k dx$$
$\triangleright$ \textbf{Th:} Пусть f(x) имеет ограниченную производную порядка k+1: $\max_{x\in [a,b]} |f^{(k+1)}(x)|=M_{k+1}$ и $h=\max_{0 \leq i \leq n-1} |x_{i+1} - x_i| \rightarrow$
$$ \big| \int_{a}^b f(x)dx - \int_{a}^b P_k dx \big| \leq const |b-a|M_{k+1}h^{k+1}$$
	$\circ$ 
		$$\max_{x \in [a,b]} |f(x)-P_k| \leq const M_{k+1}h^{k+1} $$
	$\bullet$ 
\subsection{Квадратурные формулы прямоугольников, трапеций, Симпсона}
\subsection{Оценки погрешности квадратных формул.}

\section{Билет №8}
\subsection{Норма матрицы. Согласованные и подчиненные нормы. Примеры подчиненных матричных норм}
\subsection{Число обусловленности матрицы}
\subsection{Теорема об относительной погрешности решения системы линейных алгебраических уравнений (с доказательством)}

\section{Билет №9}
\subsection{Метод Гаусса решения систем линейных алгебраических уравнений и его связь с LU-разложением матрицы}
\subsection{Выбор ведущего элемента}

\section{Билет №10}
\subsection{Итерационные методы решения систем линейных алгебраических уравнений.}
\subsection{Достаточное условие (с доказательством) и критерий (без доказательства) сходимости метода простых итераций.}

\section{Билет №11}
\subsection{Метод Якоби решения систем линейных алгебраических уравнений: матричная и покомпонентная формулировка}
\subsection{Достаточное условие и критерий сходимости (оба с доказательством)}
\subsection{Геометрическая интерпретация метода для системы 2 на 2.}

\section{Билет №12}
\subsection{Метод Зейделя решения систем линейных алгебраических уравнений: матричная и покомпонентная формулировка}
\subsection{Достаточное условие и критерий сходимости (оба с доказательством)}
\subsection{Геометрическая интерпретация метода для системы 2 на 2.}

\section{Билет №13}
\subsection{Теорема об эквивалентности задач минимизации квадратичной функции и решения системы линейных алгебраических уравнений (с доказательством)}
\subsection{Методы решения систем линейных алгебраических уравнений, основанные на минимизации функции – метод наискорейшего спуска и метод минимальных невязок.}

\section{Билет №14}
\subsection{Переопределенные системы линейных алгебраических уравнений}
\subsection{Задачи, приводящие к переопределенным системам. }
\subsection{Определение обобщенного решения переопределенных систем линейных алгебраических уравнений}

\section{Билет №15}
\subsection{Решение нелинейных алгебраических уравнений методами простых итераций и релаксации}
\subsection{Критерий сходимости простых итераций (с доказательством).}
\subsection{Графическая интерпретация метода простых итераций.}
\subsection{Метод Ньютона решения нелинейных алгебраических уравнений и систем.}

\section{Билет №16}
\subsection{Численное решение задачи Коши для обыкновенных дифференциальных уравнений первого порядка}
\subsection{Понятие сходимости разностной схемы на примере простейшего линейного уравнения и явной схемы Эйлера.}

\section{Билет №17}
\subsection{Численное решение задачи Коши для обыкновенных дифференциальных уравнений первого порядка.}
\subsection{Понятие сходимости разностной схемы на примере простейшего линейного уравнения и схемы с центральной разностью.}

\section{Билет №18}
\subsection{Численное решение задачи Коши для обыкновенных дифференциальных уравнений первого порядка}
\subsection{Пример неустойчивой схемы, аппроксимирующей исходную задачу.}

\section{Билет №19}
\subsection{Аппроксимация, устойчивость, сходимость разностных схем решения обыкновенных дифференциальных уравнений}
\subsection{Теорема Рябенького-Лакса.}


\end{document}