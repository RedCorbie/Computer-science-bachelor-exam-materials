\documentclass[a4paper]{article}
\usepackage{fontspec} 
\usepackage{polyglossia}
\setmainlanguage{russian} 
\setotherlanguage{english}
\newfontfamily{\cyrillicfont}{Times New Roman}

\usepackage{mathtools}
\usepackage{fullpage}
\usepackage[utf8x]{inputenc}
\usepackage{amsmath}
\usepackage[colorinlistoftodos]{todonotes}


\title{Введение в вычислительную математику}
\author{MIPT DIHT}
\begin{document}
\maketitle

\section{Билет №1}
\subsection{Численное дифференцирование. Простейшие формулы численного дифференцирования.}
Функция заменяется таблицей значений. Производная $$\frac{df}{dx} \lim_{h \to 0} \frac{f(x + h) - f(x)}{h}$$ заменяется $$f'(x) \approx \frac{f(x + h) - f(x)}{h}$$
Пусть функция $f(x)$ имеет достаточное число производных, требуется вычислить ее производную $f'(x)$ в данной точке х. Задачу отыскания h содержащую предельный переход, можно заменить приближенно задачами вычисления по одной из формул
$$f'(x) \approx \frac{f(x + h) - f(x)}{h}$$
$$f'(x) \approx \frac{f(x) - f(x - h)}{h}$$
$$f'(x) \approx \frac{f(x + h) - f(x - h)}{2h}$$
Для замены производной f"(x) можно воспользоваться формулой
$$f''(x) \approx \frac{f(x + h) - 2f(x) + f(x - h)}{h^2}$$

\subsection{Метод неопределенных коэффициентов для вывода формул численного дифференцирования.}
\subsection{Порядок аппроксимации формул численного дифференцирования.}
$\triangleright$ \textbf{Def:} Порядок аппроксимации – показатель степени h в главном члене погрешности 
\subsection{Оптимальный шаг сетки численного дифференцирования.}

\section{Билет №2}
\subsection{Постановка задачи интерполяции.}
Заданы точки $x=x_0 \ldots x=x_n$, значения в этих точках $f(x_0) \ldots f(x_n)$ и таблица значений ф-ции f(x) в этих точках
Задача: найти алгебраический интерполяционный многочлен $P_n(x,f,x_0 \ldots x_n): \ P_n(x) = c_nx^n + \ldots +c_0$, принимающий в узлах заданные значения и степени не выше n
\subsection{Полиномиальная интерполяция, ее существование и единственность.}
$\triangleright$ \textbf{Th:} Пусть заданы $x_0 \neq \ldots \neq x_n,\ f(x_0) \ldots f(x_n). \ \exists !$ многочлен $P_n(x) = P_n(x,f,x_0 \ldots x_n)$ степени не выше n, принимающий в заданных узлах заданные значения \\
	$\circ$ Сначала покажем, что существует единственный, затем построим.
	Если бы было два многочлена $P_n^1(x),\ P_n^2(x) \rightarrow R_n(x) = P_n^1(x)-P_n^2(x)$ обращается в ноль в n+1 точке. Но $\forall R_n(x) \not\equiv 0$ имеет столько корней, какова его степень $\rightarrow R_n(x) \equiv 0 \rightarrow P_n^1(x)=P_n^2(x)$.
	Введем вспомогательные многочлены $$l_k(x) = \frac{(x-x_0) \ldots (x-x_{k-1})(x-x_{k+1}) \ldots (x-x_n)}{(x_k-x_0) \ldots (x_k-x_{k-1})(x_k-x_{k+1}) \ldots (x_k-x_n)}$$ 
	$$l_k(x_j)=\begin{cases} 1,  & \mbox{if }x_j = x_k\\ 0, & \mbox{if }x_j \neq x_k \end{cases}$$
	$$P_n(x) = P_n(x,f,x_0 \ldots x_n) = f(x_0)l_0(x) + \ldots + f(x_n)l_n(x)$$ – искомый интерполяционный многочлен. Каждое слагаемое степени не выше n $\rightarrow$ весь многочлен степени не выше n и очевидно выполнено $P_n(x_j)=f(x_j)$
	$\bullet$
\subsection{Интерполяционный полином в форме Лагранжа.}
	$$P_n(x) = P_n(x,f,x_0 \ldots x_n) = f(x_0)l_0(x) + \ldots + f(x_n)l_n(x)$$ из док-ва

\section{Билет №3}
\subsection{Постановка задачи интерполяции.}
Заданы точки $x=x_0 \ldots x=x_n$, значения в этих точках $f(x_0) \ldots f(x_n)$ и таблица значений ф-ции f(x) в этих точках
Задача: найти алгебраический интерполяционный многочлен $P_n(x,f,x_0 \ldots x_n): \ P_n(x) = c_nx^n + \ldots +c_0$, принимающий в узлах заданные значения и степени не выше n
\subsection{Разделенные разности.}
Пусть f(x) в точках $x_a, x_b, x_c \ldots$ принимает $f(x_a), f(x_b), f(x_c) \ldots$. $f(x_k)$ – разностное отношение нулевого порядка в точке. Разностное отношение первого порядка: $$ f(x_k,x_t) = \frac{f(x_t) - f(x_k)}{x_t - x_k}$$
Разностное отношение n-1 порядка: $$ f(x_0 \ldots x_n) = \frac{f(x_1 \ldots x_n) - f(x_0 \ldots x_{n-1})}{x_n - x_0}$$
$\triangleright$ \textbf{Свойства}
\begin{enumerate}
	\item $P_n(x,f,x_0 \ldots x_n) = P_{n-1}(x,f,x_0 \ldots x_{n-1}) + f(x_0 \ldots x_n)(x-x_0)\ldots (x-x_{n-1})$
	\item $P_n(x,f,x_0 \ldots x_n) = c_nx^n + \ldots + c_0$, т е $f(x_0 \ldots x_n) = c_n$
	\item $f(x_0 \ldots x_n)=0 \iff f(x_0) \ldots f(x_n)$ – значения $Q_m, m<n$
\end{enumerate}
\subsection{Интерполяционный полином в форме Ньютона.}
Интерполяционный многочлен в форме Ньютона: $$P_n(x, f, x_0 \ldots x_n) = f(x_0) + (x-x_0)f(x_0,x_1) + \ldots + (x-x_0)\ldots (x-x_{n-1})f(x_0 \ldots x_n) $$
\subsection{Обусловленность задачи интерполяции и константа Лебега.}
Пусть узлы интерполяции $x_0 \ldots x_n$ лежат на $a \leq x \leq b$, пусть $f(x_0) \ldots f(x_n)$ заданные числа. Соответствующий интерп. мн-н: $P_n(x) = P_n(x,f,x_0 \ldots x_n) = P_n(x,f)$. Приданим значениям $f(x_j)$ возмущение $\delta f(x_j), \ P_n(x,f) \rightarrow P_n(x, f + \delta f) = P_n(x,f) + P_n(x, \delta f)$.
Мера чувствительности интерп. мн-на к возмущению – наименьшее $L_n: \ \forall \delta f$ $$\max_{x \in (a,b)} |P_n(x,\delta f)| \leq L_n \max_{j} |\delta f(x_j)|$$
$L_n = L_n(x_0 \ldots x_n,a,b)$ – константы Лебега. В случае равномерно расположенных узлов: $$2^{n-1} > L_n > 2^{n-3} \frac{1}{\sqrt{n-1}} \frac{1}{n-3/2} $$ т е чувствительность рез-та интерполяции резко возрастат при росте n  

\section{Билет №4}
\subsection{Теорема об остаточном члене интерполяционного полинома (с доказательством)}
\subsection{Оценка остаточного члена на равномерной сетке.}

\section{Билет №5}
\subsection{Полиномы Чебышева, их свойства. }
\subsection{Применение полиномов Чебышева при построении узлов интерполяции.}

\section{Билет №6}
\subsection{Понятие сплайн-интерполяции}

\subsection{Процедура построения кубического сплайна показателя гладкости два.}

\section{Билет №7}
\subsection{Численное интегрирование.}
\subsection{Квадратурные формулы прямоугольников, трапеций, Симпсона}
\subsection{Оценки погрешности квадратных формул.}

\section{Билет №8}
\subsection{Норма матрицы. Согласованные и подчиненные нормы. Примеры подчиненных матричных норм}
\subsection{Число обусловленности матрицы}
\subsection{Теорема об относительной погрешности решения системы линейных алгебраических уравнений (с доказательством)}

\section{Билет №9}
\subsection{Метод Гаусса решения систем линейных алгебраических уравнений и его связь с LU-разложением матрицы}
\subsection{Выбор ведущего элемента}

\section{Билет №10}
\subsection{Итерационные методы решения систем линейных алгебраических уравнений.}
\subsection{Достаточное условие (с доказательством) и критерий (без доказательства) сходимости метода простых итераций.}

\section{Билет №11}
\subsection{Метод Якоби решения систем линейных алгебраических уравнений: матричная и покомпонентная формулировка}
\subsection{Достаточное условие и критерий сходимости (оба с доказательством)}
\subsection{Геометрическая интерпретация метода для системы 2 на 2.}

\section{Билет №12}
\subsection{Метод Зейделя решения систем линейных алгебраических уравнений: матричная и покомпонентная формулировка}
\subsection{Достаточное условие и критерий сходимости (оба с доказательством)}
\subsection{Геометрическая интерпретация метода для системы 2 на 2.}

\section{Билет №13}
\subsection{Теорема об эквивалентности задач минимизации квадратичной функции и решения системы линейных алгебраических уравнений (с доказательством)}
\subsection{Методы решения систем линейных алгебраических уравнений, основанные на минимизации функции – метод наискорейшего спуска и метод минимальных невязок.}

\section{Билет №14}
\subsection{Переопределенные системы линейных алгебраических уравнений}
\subsection{Задачи, приводящие к переопределенным системам. }
\subsection{Определение обобщенного решения переопределенных систем линейных алгебраических уравнений}

\section{Билет №15}
\subsection{Решение нелинейных алгебраических уравнений методами простых итераций и релаксации}
\subsection{Критерий сходимости простых итераций (с доказательством).}
\subsection{Графическая интерпретация метода простых итераций.}
\subsection{Метод Ньютона решения нелинейных алгебраических уравнений и систем.}

\section{Билет №16}
\subsection{Численное решение задачи Коши для обыкновенных дифференциальных уравнений первого порядка}
\subsection{Понятие сходимости разностной схемы на примере простейшего линейного уравнения и явной схемы Эйлера.}

\section{Билет №17}
\subsection{Численное решение задачи Коши для обыкновенных дифференциальных уравнений первого порядка.}
\subsection{Понятие сходимости разностной схемы на примере простейшего линейного уравнения и схемы с центральной разностью.}

\section{Билет №18}
\subsection{Численное решение задачи Коши для обыкновенных дифференциальных уравнений первого порядка}
\subsection{Пример неустойчивой схемы, аппроксимирующей исходную задачу.}

\section{Билет №19}
\subsection{Аппроксимация, устойчивость, сходимость разностных схем решения обыкновенных дифференциальных уравнений}
\subsection{Теорема Рябенького-Лакса.}


\end{document}