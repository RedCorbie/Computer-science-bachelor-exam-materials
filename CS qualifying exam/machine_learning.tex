\documentclass[a4paper]{article}

\usepackage[utf8]{inputenc}
\usepackage[english, russian]{babel}
\usepackage[fleqn]{amsmath}
\usepackage{amsfonts, amssymb, amsthm, mathtools}

\usepackage{mathtools}
\usepackage{fullpage}
\usepackage[colorinlistoftodos]{todonotes}


\title{Машинное обучение}
\author{MIPT DIHT}


\theoremstyle{plain}
\newtheorem*{theorem-star}{Theorem}
\newtheorem{theorem}{Theorem}
\newtheorem*{lem-star}{Lemma}
\newtheorem{lem}{Lemma}
\newtheorem*{proposition-star}{Proposition}
\newtheorem{proposition}{Proposition}
\newtheorem{statement}{Statement}
\newtheorem*{statement-star}{Statement}
\newtheorem{corollary}{Corollary}
\newtheorem*{corollary-star}{Corollary}

\theoremstyle{remark}
\newtheorem*{remark}{Remark}

\theoremstyle{definition}
\newtheorem*{definition-star}{Definition}
\newtheorem{definition}{Definition}
\newtheorem{example}{Example}
\newtheorem*{example-star}{Example}
\newtheorem{problem}{Problem}
\newtheorem{problem-star}{Problem}

\renewenvironment{proof}{{\bfseries Proof}}{$\bullet$}

\newcommand{\myequat}[1]{\begin{equation} #1 \nonumber \end{equation}}
\newcommand{\pars}[1]{\left( #1 \right)} 
\newcommand{\class}[1]{\left[ #1 \right]} 
\newcommand{\dd}{\; \mathrm{d}}
\newcommand{\setR}{\mathbb{R}}
\newcommand{\setRn}{\mathbb{R}^n}
\newcommand{\setRinf}{\mathbb{R}^{\infty}}
\newcommand{\setC}{\mathbb{C}}
\newcommand{\setN}{\mathbb{N}}
\newcommand{\setZ}{\mathbb{Z}}
\newcommand{\setQ}{\mathbb{Q}}
\newcommand{\setM}{\mathcal{M}}
\newcommand{\setL}{\mathcal{L}}
\newcommand{\setA}{\mathcal{A}}
\newcommand{\setF}{\mathcal{F}}

\newcommand*\circled[1]{\tikz[baseline=(char.base)]{\node[shape=circle,draw,inner sep=2pt] (char) {#1};}}

\newcommand{\walls}[1]{\left | #1 \right |} % |smth_vertically_large|
\newcommand{\braces}[1]{\left\{ #1 \right\}} % {smth_vertically_large}

\newcommand{\condset}[2]{\braces{\, #1 \mid #2 \,}} % definition of set with condition

\newcommand{\expl}[1]{\walls{\text{#1}}} % explanation inside formula

\newcommand{\toup}[1]{\xrightarrow{#1}}
\newcommand{\toae}{\toup{\text{\,п.н.}}} % almost everywhere convergence designation
\newcommand{\todown}[1]{\xrightarrow[#1]{}}

\newcommand{\equp}[1]{\stackrel{#1}{=}}

\newcommand{\conj}[1]{\overline{#1}} % complex conjugation
\newcommand{\comp}[1]{\overline{#1}} % set complement

\DeclareMathOperator{\cov}{cov}

\renewcommand{\emptyset}{\varnothing}
\renewcommand{\epsilon}{\varepsilon}
\renewcommand{\phi}{\varphi}
\renewcommand{\leq}{\leqslant}
\renewcommand{\geq}{\geqslant}
\renewcommand{\Im}{\mathop{\mathrm{Im}}\nolimits}
\renewcommand{\Re}{\mathop{\mathrm{Re}}\nolimits}

\DeclareMathOperator*{\argmin}{arg\,min}

\newcommand{\bigtitle}[1]{\title{\textbf{\underline{#1}}}}
\newcommand{\boldtitle}[1]{\title{\textbf{#1}}}

\newcommand*{\hm}[1]{#1\nobreak\discretionary{}%
{\hbox{$\mathsurround=0pt #1$}}{}} % a\hm=b makes "=" carriable to the next line with duplication of the sign


\begin{document}
\maketitle

\section{Билет №1}
\subsection{Байесовские методы классификации. Наивный байесовский классификатор}
\bigtitle{Постановка задачи}
Пусть X — множество объектов, Y — конечное множество имён классов, множество X × Y является вероятностным пространством с плотностью распределения $p(x, y) = P(y)p(x|y)$. Вероятности появления объектов каждого из классов $P_y = P(y)$ называются априорными вероятностями классов. Плотности распределе-
ния $p_y(x) = p(x|y)$ называются функциями правдоподобия классов. Вероятностная постановка задачи классификации разделяется на две независимые подзадачи. 
\begin{problem}Имеется простая выборка $X^l = (x_i, y_i)_{i=1}^l$ из неизвестного распределения $p(x, y) = P_yp_y(x)$. Требуется построить эмпирические оценки априорных вероятностей $\hat{P}_y$ и функций правдоподобия $\hat{p}_y(x)$ для каждого из классов $y \in Y$ .
\end{problem}
\begin{problem} По известным плотностям распределения $p_y(x)$ и априорным вероятностям $P_y$ всех классов $y \in Y$ построить алгоритм a(x), минимизирующий вероятность ошибочной классификации.
\end{problem}
Первая задача имеет множество решений, поскольку многие распределения $p(x, y)$ могли бы породить одну и ту же выборку $X^l$

\bigtitle{Функционал среднего риска}
Событие вида «$x \in \Omega$ при условии, что x принадлежит классу y»:
$$ P(\Omega|y) = \int_{\Omega} p_y(x) dx $$
\begin{definition} Функционалом среднего риска называется ожидаемая величина потери
при классификации объектов алгоритмом a: $$ R(a) = \sum_{\substack{y \in Y}} \sum_{\substack{s \in Y}} \lambda_{ys} P_y P(a_s|y)$$\end{definition}

\bigtitle{Оптимальное байесовское решающее правило}
\begin{theorem}
Если известны априорные вероятности $P_y$ и функции правдоподобия $p_y(x)$, то минимум среднего риска R(a) достигается алгоритмом
$$a(x) = arg \min_{\substack{s \in Y}} \sum_{\substack{y \in Y}} \lambda_{ys} P_y p_y(x) $$
\end{theorem}
\begin{proof} Для произвольного t из Y запишем функционал среднего риска:
$$ R(a) = \sum_{\substack{y \in Y}} \sum_{\substack{s \in Y}}\lambda_{ys} P_y P(A_s|y) = $$
$$ = \sum_{\substack{y \in Y}}\lambda_{ys} P_y P(A_t|y) + \sum_{\substack{s \in Y\\{t}}} \sum_{\substack{y \in Y}}\lambda_{ys} P_y P(A_s|y)  $$
Из формулы полной вероятности $ P(A_t|y) = 1- \sum_{\substack{s \in Y\\{t}}}P(A_s|y) $
$$ R(a) = \sum_{\substack{y \in Y}}\lambda_{ys} P_y + \sum_{\substack{s \in Y\\{t}}} \sum_{\substack{y \in Y}}(\lambda_{ys} - \lambda_{yt}) P_y P(a_s|y) = $$
$$ = const(a) +  \sum_{\substack{s \in Y\\{t}}} \int_{A_s} \sum_{\substack{y \in Y}}(\lambda_{ys} - \lambda_{yt}) P_y p_y(x)$$
Обозначим $ g_s(x) =  \sum_{\substack{y \in Y}}\lambda_{ys} P_y p_y(x)$
$$ R(a) = const(a) +  \sum_{\substack{s \in Y\\{t}}} \int_{A_s} (g_s(x) - g_t(x))dx $$
В выражении неизвестны только области $A_s$. Функционал R(a) есть сумма |Y|−1 слагаемых $I(A_s) = \int_{A_s}(g_s(x) - g_t(x)dx$, каждое из которых зависит только от одной области $A_s$. Минимум $I(A_s)$ достигается, когда $A_s$ совпадает с областью неположительности подынтегрального выражения. В силу произвольности t:
$$ A_s = \{x \in X|g_s(x) \leq _t(x), \forall t \in Y, t \neq s} $$
С другой стороны $a_s = \{x \in X| a(x)=s}$. Значит $a(x) = s \iff s = arg \min_{\substack{t \in Y}} g_t(x)$
\end{proof}
\begin{theorem}
Если $P_y$ и $p_y(x)$ известны, $\lambda_{ss}=0, \lambda_{ys} \equiv \lambda_y \forall y,s \in Y$, то минимум среднего риска достигается алгоритмом
$$ a(x) = arg \max_{\substack{y \in Y}} \lambda_y P_y p_y(x) $$
\end{theorem}
\bigtitle{Наивный байесовский классификатор}
\begin{problem} Задано множество объектов $X^m = {x_1, \ldots , x_m}$, выбранных случайно и независимо согласно неизвестному распределению p(x). Требуется построить эмпирическую оценку плотности — функцию $\hat{p}(x)$, приближающую p(x) на всём X.
Допустим, что объекты $x \in X$ описываются n числовыми признаками $f_j: X \rightarrow R, j = 1, \ldots, n$. Обозначим через $x =
= (\xi_1, . . . , \xi_n)$ произвольный элемент пространства объектов $X = \setR^n$, где $\xi_j = f_j (x)$.
\end{problem}
\begin{proposition}
Признаки $f_1(x),\ldots, f_n(x)$ являются независимыми случайными величинами. Следовательно, функции правдоподобия классов представимы в виде $p_y(x) = p_{y1}(\xi_1)· · · p_{yn}(\xi_n), y \in Y$, где $p_{yj} (\xi_j )$ — плотность распределения значений j-го признака для класса y.
\end{proposition}
Подставим эмпирические оценки одномерных плотностей $\hat{p}_{yj} (\xi)$ в предположение и затем в теорему 2. Получим алгоритм
$$ a(x) = arg \max_{\substack{y \in Y}} \bigg( ln \lambda_y \hat{P}_y + \sum){\substack{j=1}}^n ln \hat{p}_{yj}(\xi_j) \bigg) $$


\section{Билет №2}
\subsection{Логистическая регрессия}
\subsection{L1 и L2 регуляризации}
\subsection{Теорема об оптимальности}

\section{Билет №3}
\subsection{Метод опорных векторов. Решение двойственной задачи}
\subsection{Спрямляющее пространство}
\subsection{Ядра}

\section{Билет №4}
\subsection{Многомерная линейная регрессия}
\subsection{Лассо Тибширани}
\subsection{Гребневая регрессия}

\section{Билет №5}
\subsection{Бустинг}
\subsection{Алгоритм AdaBoost}

\end{document}