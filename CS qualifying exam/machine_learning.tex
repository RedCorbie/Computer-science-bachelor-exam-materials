\documentclass[a4paper]{article}

\usepackage[utf8]{inputenc}
\usepackage[english, russian]{babel}
\usepackage[fleqn]{amsmath}
\usepackage{amsfonts, amssymb, amsthm, mathtools}

\usepackage{mathtools}
\usepackage{fullpage}
\usepackage[colorinlistoftodos]{todonotes}


\title{Машинное обучение}
\author{MIPT DIHT}

\theoremstyle{plain}
\newtheorem*{theorem-star}{Theorem}
\newtheorem{theorem}{Theorem}
\newtheorem*{lem-star}{Lemma}
\newtheorem{lem}{Lemma}
\newtheorem*{proposition-star}{Proposition}
\newtheorem{proposition}{Proposition}

\theoremstyle{remark}
\newtheorem*{remark}{Remark}

\theoremstyle{definition}
\newtheorem*{definition-star}{Definition}
\newtheorem{definition}{Definition}

\renewenvironment{proof}{{\bfseries Proof}}{$\bullet$}

\newcommand{\myequat}[1]{\begin{equation} #1 \nonumber \end{equation}}
\newcommand{\pars}[1]{\left( #1 \right)} 
\newcommand{\class}[1]{\left[ #1 \right]} 
\newcommand{\myN}{\mathbb{N}} 
\newcommand{\myZ}{\mathbb{Z}}
\newcommand{\myR}{\mathbb{R}}
\newcommand{\myC}{\mathbb{C}}
\newcommand{\myQ}{\mathbb{Q}}
\newcommand{\myE}{\mathcal{E}} 
\newcommand{\myM}{\mathcal{M}}
\newcommand{\myO}{(1+o(1))}

\begin{document}
\maketitle

\section{Билет №1}
\subsection{Байесовские методы классификации. Наивный байесовский классификатор}

\section{Билет №2}
\subsection{Логистическая регрессия}
\subsection{L1 и L2 регуляризации}
\subsection{Теорема об оптимальности}

\section{Билет №3}
\subsection{Метод опорных векторов. Решение двойственной задачи}
\subsection{Спрямляющее пространство}
\subsection{Ядра}

\section{Билет №4}
\subsection{Многомерная линейная регрессия}
\subsection{Лассо Тибширани}
\subsection{Гребневая регрессия}

\section{Билет №5}
\subsection{Бустинг}
\subsection{Алгоритм AdaBoost}

\end{document}