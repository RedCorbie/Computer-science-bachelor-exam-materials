\documentclass[a4paper]{article}

\usepackage[utf8]{inputenc}
\usepackage[english, russian]{babel}
\usepackage[fleqn]{amsmath}
\usepackage{amsfonts, amssymb, amsthm, mathtools}

\usepackage{mathtools}
\usepackage{fullpage}
\usepackage[colorinlistoftodos]{todonotes}


\title{Машинное обучение}
\author{MIPT DIHT}


\theoremstyle{plain}
\newtheorem*{theorem-star}{Theorem}
\newtheorem{theorem}{Theorem}
\newtheorem*{lem-star}{Lemma}
\newtheorem{lem}{Lemma}
\newtheorem*{proposition-star}{Proposition}
\newtheorem{proposition}{Proposition}
\newtheorem{statement}{Statement}
\newtheorem*{statement-star}{Statement}
\newtheorem{corollary}{Corollary}
\newtheorem*{corollary-star}{Corollary}

\theoremstyle{remark}
\newtheorem*{remark}{Remark}

\theoremstyle{definition}
\newtheorem*{definition-star}{Definition}
\newtheorem{definition}{Definition}
\newtheorem{example}{Example}
\newtheorem*{example-star}{Example}
\newtheorem{problem}{Problem}
\newtheorem{problem-star}{Problem}

\renewenvironment{proof}{{\bfseries Proof}}{$\bullet$}

\newcommand{\myequat}[1]{\begin{equation} #1 \nonumber \end{equation}}
\newcommand{\pars}[1]{\left( #1 \right)} 
\newcommand{\class}[1]{\left[ #1 \right]} 
\newcommand{\dd}{\; \mathrm{d}}
\newcommand{\setR}{\mathbb{R}}
\newcommand{\setRn}{\mathbb{R}^n}
\newcommand{\setRinf}{\mathbb{R}^{\infty}}
\newcommand{\setC}{\mathbb{C}}
\newcommand{\setN}{\mathbb{N}}
\newcommand{\setZ}{\mathbb{Z}}
\newcommand{\setQ}{\mathbb{Q}}
\newcommand{\setM}{\mathcal{M}}
\newcommand{\setL}{\mathcal{L}}
\newcommand{\setA}{\mathcal{A}}
\newcommand{\setF}{\mathcal{F}}

\newcommand*\circled[1]{\tikz[baseline=(char.base)]{\node[shape=circle,draw,inner sep=2pt] (char) {#1};}}

\newcommand{\walls}[1]{\left | #1 \right |} % |smth_vertically_large|
\newcommand{\braces}[1]{\left\{ #1 \right\}} % {smth_vertically_large}

\newcommand{\condset}[2]{\braces{\, #1 \mid #2 \,}} % definition of set with condition

\newcommand{\expl}[1]{\walls{\text{#1}}} % explanation inside formula

\newcommand{\toup}[1]{\xrightarrow{#1}}
\newcommand{\toae}{\toup{\text{\,п.н.}}} % almost everywhere convergence designation
\newcommand{\todown}[1]{\xrightarrow[#1]{}}

\newcommand{\equp}[1]{\stackrel{#1}{=}}

\newcommand{\conj}[1]{\overline{#1}} % complex conjugation
\newcommand{\comp}[1]{\overline{#1}} % set complement

\DeclareMathOperator{\cov}{cov}

\renewcommand{\emptyset}{\varnothing}
\renewcommand{\epsilon}{\varepsilon}
\renewcommand{\phi}{\varphi}
\renewcommand{\leq}{\leqslant}
\renewcommand{\geq}{\geqslant}
\renewcommand{\Im}{\mathop{\mathrm{Im}}\nolimits}
\renewcommand{\Re}{\mathop{\mathrm{Re}}\nolimits}

\DeclareMathOperator*{\argmin}{arg\,min}

\newcommand{\bigtitle}[1]{\title{\textbf{\underline{#1}}}}
\newcommand{\boldtitle}[1]{\title{\textbf{#1}}}

\newcommand*{\hm}[1]{#1\nobreak\discretionary{}%
{\hbox{$\mathsurround=0pt #1$}}{}} % a\hm=b makes "=" carriable to the next line with duplication of the sign


\begin{document}
\maketitle

\section{Билет №1}
\subsection{Байесовские методы классификации. Наивный байесовский классификатор}
\bigtitle{Постановка задачи}
Пусть X — множество объектов, Y — конечное множество имён классов, множество $X \times Y$ является вероятностным пространством с плотностью распределения $p(x, y) = P(y)p(x|y)$. Вероятности появления объектов каждого из классов $P_y = P(y)$ называются априорными вероятностями классов. Плотности распределения $p_y(x) = p(x|y)$ называются функциями правдоподобия классов. Вероятностная постановка задачи классификации разделяется на две независимые подзадачи. 
\begin{problem}Имеется простая выборка $X^l = (x_i, y_i)_{i=1}^l$ из неизвестного распределения $p(x, y) = P_yp_y(x)$. Требуется построить эмпирические оценки априорных вероятностей $\hat{P}_y$ и функций правдоподобия $\hat{p}_y(x)$ для каждого из классов $y \in Y$ .
\end{problem}
\begin{problem} По известным плотностям распределения $p_y(x)$ и априорным вероятностям $P_y$ всех классов $y \in Y$ построить алгоритм a(x), минимизирующий вероятность ошибочной классификации.
\end{problem}
Первая задача имеет множество решений, поскольку многие распределения $p(x, y)$ могли бы породить одну и ту же выборку $X^l$

\bigtitle{Функционал среднего риска}
Событие вида «$x \in \Omega$ при условии, что x принадлежит классу y»:
$$ P(\Omega|y) = \int_{\Omega} p_y(x) dx $$
\begin{definition} Функционалом среднего риска называется ожидаемая величина потери
при классификации объектов алгоритмом a: $$ R(a) = \sum_{\substack{y \in Y}} \sum_{\substack{s \in Y}} \lambda_{ys} P_y P(a_s|y)$$\end{definition}

\bigtitle{Оптимальное байесовское решающее правило}
\begin{theorem}
Если известны априорные вероятности $P_y$ и функции правдоподобия $p_y(x)$, то минимум среднего риска R(a) достигается алгоритмом
$$a(x) = arg \min_{\substack{s \in Y}} \sum_{\substack{y \in Y}} \lambda_{ys} P_y p_y(x) $$
\end{theorem}
\begin{proof} Для произвольного t из Y запишем функционал среднего риска:
$$ R(a) = \sum_{\substack{y \in Y}} \sum_{\substack{s \in Y}}\lambda_{ys} P_y P(A_s|y) = $$
$$ = \sum_{\substack{y \in Y}}\lambda_{ys} P_y P(A_t|y) + \sum_{\substack{s \in Y\\{t}}} \sum_{\substack{y \in Y}}\lambda_{ys} P_y P(A_s|y)  $$
Из формулы полной вероятности $ P(A_t|y) = 1- \sum_{\substack{s \in Y\\{t}}}P(A_s|y) $
$$ R(a) = \sum_{\substack{y \in Y}}\lambda_{ys} P_y + \sum_{\substack{s \in Y\\{t}}} \sum_{\substack{y \in Y}}(\lambda_{ys} - \lambda_{yt}) P_y P(a_s|y) = $$
$$ = const(a) +  \sum_{\substack{s \in Y\\{t}}} \int_{A_s} \sum_{\substack{y \in Y}}(\lambda_{ys} - \lambda_{yt}) P_y p_y(x)$$
Обозначим $ g_s(x) =  \sum_{\substack{y \in Y}}\lambda_{ys} P_y p_y(x)$
$$ R(a) = const(a) +  \sum_{\substack{s \in Y\\{t}}} \int_{A_s} (g_s(x) - g_t(x))dx $$
В выражении неизвестны только области $A_s$. Функционал R(a) есть сумма $|Y|-1$ слагаемых $I(A_s) = \int_{A_s}(g_s(x) - g_t(x)dx$, каждое из которых зависит только от одной области $A_s$. Минимум $I(A_s)$ достигается, когда $A_s$ совпадает с областью неположительности подынтегрального выражения. В силу произвольности t:
$$ A_s = \{x \in X|g_s(x) \leq _t(x), \forall t \in Y, t \neq s\} $$
С другой стороны $a_s = \{x \in X| a(x)=s\}$. Значит $a(x) = s \iff s = arg \min_{\substack{t \in Y}} g_t(x)$
\end{proof}
\begin{theorem}
Если $P_y$ и $p_y(x)$ известны, $\lambda_{ss}=0, \lambda_{ys} \equiv \lambda_y \forall y,s \in Y$, то минимум среднего риска достигается алгоритмом
$$ a(x) = arg \max_{\substack{y \in Y}} \lambda_y P_y p_y(x) $$
\end{theorem}
\bigtitle{Наивный байесовский классификатор}
\begin{problem} Задано множество объектов $X^m = {x_1, \ldots , x_m}$, выбранных случайно и независимо согласно неизвестному распределению p(x). Требуется построить эмпирическую оценку плотности — функцию $\hat{p}(x)$, приближающую p(x) на всём X.
Допустим, что объекты $x \in X$ описываются n числовыми признаками $f_j: X \rightarrow R, j = 1, \ldots, n$. Обозначим через $x =
= (\xi_1, . . . , \xi_n)$ произвольный элемент пространства объектов $X = \setR^n$, где $\xi_j = f_j (x)$.
\end{problem}
\begin{proposition}
Признаки $f_1(x),\ldots, f_n(x)$ являются независимыми случайными величинами. Следовательно, функции правдоподобия классов представимы в виде $p_y(x) = p_{y1}(\xi_1)· · · p_{yn}(\xi_n), y \in Y$, где $p_{yj} (\xi_j )$ — плотность распределения значений j-го признака для класса y.
\end{proposition}
Подставим эмпирические оценки одномерных плотностей $\hat{p}_{yj} (\xi)$ в предположение и затем в теорему 2. Получим алгоритм
$$ a(x) = arg \max_{\substack{y \in Y}} \bigg( ln \lambda_y \hat{P}_y + \sum){\substack{j=1}}^n ln \hat{p}_{yj}(\xi_j) \bigg) $$

\section{Билет №2}
\subsection{Логистическая регрессия}
Базовые предположения. Пусть классов два, $Y = {-1, +1}$, объекты описываются n числовыми признаками $f_j: X \rightarrow R, j = 1, \ldots , n$. Будем полагать $X = \setR^n$ отождествляя объекты с их признаковыми описаниями: $x \equiv (f_1(x), \ldots , f_n(x))$.
\subsection{L1 и L2 регуляризации}

\subsection{Теорема об оптимальности}

\section{Билет №3}
\subsection{Метод опорных векторов. Решение двойственной задачи}
Задача классификации: $X = \setRn, Y = \{-1,+1\}$ по обучающей выборке $X^l=(x_i,y_i)_{i=1}^l$ найти параметры $w \in \setRn, w_0 \in \setR$ алгоритма классификации
$$ a(x,w) = sign(<x,w>-w_0) $$
\textbf{Метод минимизации аппроксимированного регуляризованного эмпирического риска} \\
$$ \sum_{\substack{i=1}}^l (1-M_i(w,w_0))_+ + \frac{1}{2C}||w||^2 \rightarrow min_{\substack{w,w_0}} $$
где $m_i(w,w_0)=y_i(<x_i,w>-w_0)$ - отступ объекта $x_i$ \\
Линейный классификатор: $$ a(x,w) = sign(<w,x>-w_0), \ w,x \in \setRn, w_0 \in \setR $$
Пусть выборка $X^l=(x_i,y_i)_{i=1}^l$ линейно разделима:
$$ \exists w,w_0: \ M_i(w,w_0)=y_i(<w,x>-w_0)>0, \ i = 0, \ldots l $$
Нормировка: $ min M_i(w,w_0)=1$. Разделяющая полоса: $\{x: \ -1 \leq <w,x> \leq 1\}$. Ширина полосы: $\frac{<x_+-x_-, w>}{||w||}=\frac{2}{||w||} \rightarrow max$ \\
Линейно разделимая выборка: $ 
\begin{cases}
	||w||^2 \rightarrow \min_{w,w_0};\\
	M_i(w,w_0) \geq 1, i = 1, \ldots, l
\end{cases}$ \\
Переходим к линейно неразделимой выборке (эвристика): $
\begin{cases}
	\frac{1}{2}||w||^2 + C \sum_{i=1}^l \xi_i \rightarrow min_{w,w_0,\xi};\\
	M_i(w,w_0) \geq 1-\xi_i; \\
	\xi_i \geq 0
\end{cases}$ \\
Эквивалентная задача безусловной минимизации: $$ C\sum_{\substack{i=1}}^l \big( 1-M_i(w,w_0)\big)_+ + \frac{1}{2}||w||^2 \rightarrow \min_{\substack{w,w_0}} $$

\subsubsection{Условие Каруша-Куна-Таккера}
Задача математического программирования:
$$ \begin{cases}
	f(x) \rightarrow \min_{\substack{x}} \\
	g_i(x) \leq 0, i=1, \ldots, m \\
	h_j(x) = 0, j=1, \ldots, k
\end{cases}$$
Необходимые условия: если х - точка локального минимума, то существуют множители $\mu_i, \lambda_j$:
$$\begin{cases}
	\frac{\partial L}{\partial x} = 0, \ L(x,\mu, \lambda) = f(x) + \sum_{\substack{i=1}}^m \mu_ig_i(x) + \sum_{\substack{j=1}}^k \lambda_j g_j(x)\\
	g_i(x) \leq 0, \ h_j(x) = 0 \text{ - исходные ограничения} \\
	\mu_i \geq 0 \text{ - двойственные ограничения} \\
	\mu_i g_i(x)=0 \text{ - условие дополняющей нежесткости}
\end{cases}$$
Функция Лагранжа: $$ L(w,w_0,\xi; \lambda, \eta) = \frac{1}{2}||w||^2 - \sum_{\substack{i=1}}^l \lambda_i \bigg(M_i(w,w_0)-1 \bigg) - \sum_{\substack{i=1}}^l \xi_i (\lambda_i + \eta_i -C) $$
$\lambda_i$ - переменные, двойственные к ограничениям $M_i \geq 1 -\xi_i$.\\
$\eta_i$ - переменные, двойственные к ограничениям $\xi_i \geq 0$
$$ \begin{cases}
	\frac{\partial L}{\partial w} = 0, \ \frac{\partial L}{\partial w_0} = 0, \ \frac{\partial L}{\partial \xi} = 0,\\
	\xi_i \geq 0, \ \lambda_i \geq 0, \ \eta_i \geq 0 \\
	\lambda_i = 0 \ или \ M_i(w,w_0)=1-\xi_i \\
	\eta_i = 0 \ или \ \xi_i=0
\end{cases} $$
Необходимые условия медловой точки функции Лагранжа 
$$ \begin{cases}
	\frac{\partial L}{\partial w} = w - \sum_{\substack{i=1}}^l \lambda_i y_i x_i = 0, & \Rightarrow  w =  \sum_{\substack{i=1}}^l \lambda_i y_i x_i\\
	\frac{\partial L}{\partial w_0} =  -\sum_{\substack{i=1}}^l \lambda_i y_i = 0 & \Rightarrow  \sum_{\substack{i=1}}^l \lambda_i y_i =0, \\ 
	\frac{\partial L}{\partial \xi} = - \lambda_i - \eta_i + C = 0 & \Rightarrow  \eta_i + \lambda_i = C

\end{cases} $$
\subsubsection{Понятие опорного вектора}
Типизация объектов:
\begin{enumerate}
	\item $\lambda_i=0, \eta_i=C, \xi_i=0, M_i \geq 1$ - периферийные (неинформативные) объекты
	\item $0 < \lambda_i < C, 0 < \eta_i < C, \xi_i = 0, M_i = 1$ - опорные граничные объекты
	\item $\lambda_i=C, \eta_i=0, \xi_i >0, M_i <1$ - опорные-нарушители
\end{enumerate}
$x_i$ - опорный, если $\lambda_i \neq 0$

\subsubsection{Двойственная задача}
$$ \begin{cases}
	-L(\lambda) = -\sum_{\substack{i=1}}^l \lambda_i + \frac{1}{2} \sum_{\substack{i=1}}^l\sum_{\substack{j=1}}^l \lambda_i \lambda_j y_i, y_j<x_i,x_j> \rightarrow \min_{\substack{\lambda}}\\
	0 \leq \lambda_i \leq C, i = 1, \ldots, l \\
	\sum_{\substack{i=1}}^l \lambda_iy_i=0
\end{cases} $$
Решение прямой задачи выражается через решение двойственной:
$$ \begin{cases}
	w = \sum_{\substack{i=1}}^l \lambda_i y_i x_i \\
	w_0 = <w,x> - y_i, \ \forall i: \ \lambda_i>, M_i=1
\end{cases} $$
Линейный классификатор: $$ a(x) = sign \bigg( \sum_{\substack{i=1}}^l \lambda_i y_i <x_i,x> - w_0 \bigg)$$
\subsection{Спрямляющие пространства и ядра}
Спрямляющее пространство более высокой размерности: $\psi: X \rightarrow H$
\begin{definition}
Функция $K: X \times X \rightarrow \setR$ - ядро, если $K(x,x')=<\psi(x),\psi(x')>$ при некотором $\psi: X \rightarrow H$, где Н - гильбертово пространство
\end{definition}
\begin{theorem}
Функция $K(x,x')$ является ядром $\iff K(x,x')=K(x',x)$ и $\int_X \int_X K(x,x')g(x)g(x')dxdx' \geq 0 \forall g:X \rightarrow \setR$ - неотрицательно определена
\end{theorem}
Конструктивные методы синтеза ядер:
\begin{itemize}
	\item $K(x,x')=<x,x'>$
	\item $K(x,x')=1$
	\item $K(x,x')=K_1(x,x')*K_2(x,x')$ - произведение ядер
	\item $\forall \psi: X \rightarrow \setR \ K(x,x')=\psi(x) \psi(x')$
	\item $K(x,x') = \alpha_1 K_1(x,x') + \alpha_2 K_2(x,x'), \ \alpha_{1,2} >0$
	\item $\forall \varphi: X \rightarrow X$ если $K_0$ ядро, то $K(x,x')=K_0(\varphi(x),\varphi(x'))$
	\item $s: X \times X \rightarrow \setR$ - симметричная интегрируемая, то $K(x,x') = \int_X s(x,z)s(x',z)dz$
	\item $K_0$ - ядро и ф-ция $f: \setR \rightarrow \setR$ представима через сход. степ. ряд с неотриц. коэффициентами, то $K(x,x')=f(K_0(x,x'))$
\end{itemize}

\section{Билет №4}
\subsection{Многомерная линейная регрессия}
\bigtitle{Метод наименьших квадратов}
$$ Q(\alpha, X^l) = \sum_{\substack{i=1}}^l w_i \Big( f(x_i, \alpha) - y_i)^2 \Big) \rightarrow \min_{\substack{\alpha}} $$
где $w_i$ - вес i-го объекта
$Q(\alpha^*, X^l$ - остаточная сумма квадратов
\bigtitle{Многомерная логрегрессия}
$f_1(x) \ldots f_n(x)$ - числовые признаки. Модель многомерной регрессии:
$$f(x, \alpha) = \sum_{\substack{j=1}}^n \alpha_j f_j(x), \alpha \in \setR^n$$
Функионал квадрата ошибки:
$$ Q(\alpha, X^l) = \sum_{\substack{i=1}}^l (f(x_i, \alpha) - y_i)^2 = ||F\alpha - y ||^2 \rightarrow \min_{\substack{\alpha}} $$
Необхожимое условие минимума в матричном виде: $$ \frac{\partial Q}{\partial \alpha}(\alpha) = 2F^T(F\alpha-y) = 0 $$
откуда следует нормальная система задачи метода наименьших квадратов: $F^TF\alpha = F^Ty$
Решение системы: $\alpha^* = (F^TF)^{-1}F^Ty = F^{+}y$. Значение функционала: $Q(\alpha^*) = ||P_Fy -y ||^2$, где $P_F = FF^+ = (F^TF)^{-1}F^T$ - проекционная матрица
\subsection{Лассо Тибширани}
$$
\begin{cases}
	Q(\alpha) = ||F\alpha - y||^2 \rightarrow \min_{\substack{\alpha}}, \\
	\sum_{\substack{j=1}}^n|\alpha_j| \leq \varkappa
\end{cases}
$$
Лассо приводит к отбору признаков. После замены переменных
$$ \begin{cases}
	\alpha_j = \alpha_j^+ - \alpha_j^- \\
	|\alpha_j| = \alpha_j^+ + \alpha_j^- 
\end{cases}, \alpha_j^+, \alpha_j^- \geq 0 $$
ограничения принимают канонический вид $$ \sum_{\substack{j=1}}^n \alpha_j^+ + \alpha_j^- \leq \varkappa $$
Чем меньше $\varkappa$, тем больше таких j что $\alpha_j^+ = \alpha_j^- =0$

\subsection{Гребневая регрессия}
Штраф за увеличение норма вектора весов $||\alpha||$:
$$ Q_\tau(\alpha) = ||F\alpha - y||^2 + \frac{1}{\sigma} ||\alpha||^2$$
где $\tau = \frac{1}{\sigma}$ - неотрицательный параметр регуляризации.
\bigtitle{Вероятностная интерпретация}
Априорное распределение вектора $\alpha$ - гауссовское с ковариационной матрицей $\sigma I_n$
Модифицированное МНК-решение: $$ \alpha_\tau^* = (F^{T}F + \tau I_n)^{-1}F^{T}y $$


\section{Билет №5}
\subsection{Бустинг}
$X^l=(x_i,y_i)_{i=1}^l \subset X \times Y$ - обучающая выборка, $y_i=y^*(x_i)$. \\
$a(x)=C(b(x))$ - алгоритм, где \begin{itemize}
	\item $b: X \rightarrow R$ - безовый алгоритм (алгоритмический оператор)
	\item $C: R \rightarrow Y$ - решающее правило
	\item $r$ - пространство оценок
\end{itemize}
\begin{definition}
Композиция базовых алгоритмов $b_1, \ldots, b_{\tau}$
$$a(x) = C(F(b_1(x), \ldots, b_{tau}(x)))$$ где $F:R^T \rightarrow R$ - корректирующая операция
\end{definition}
\begin{example} 
$Y=\{1, \ldots, M\}$
$$a(x) = arg \max_{\substack{y \in Y}}b_y(x)$$ где $R=\setR^M, b: X \rightarrow \setR^M, C(b_1, \ldots, b_M)=arg \max_{\substack{y \in Y}} b_y$
\end{example}
Функционал качества базового алгоритма (L - функция потерь):
$$ Q(b,X^l) = \sum_{\substack{i=1}}^l L(b(x_i),y_i) $$
Итерационный процесс: \begin{itemize}
	\item $b_1 = arg \min_{\substack{b}} Q(b, X^l) $\\
	\ldots
	\item $b_t = arg \min_{\substack{b}} Q(F(b_1, \ldots, b_{t-1}), X^l)$
\end{itemize}
Идея - свести задачу к стандартной с весами объектор и, возможно, другой функцией потерь:
$$ b_t = \arg \min_{\substack{b}} w_i \hat{L}(b(x_i), y_i) $$

\begin{theorem} \textbf{Основная теорема бустинга} \\
Пусть для любого нормированного вектора весов $U^l \ \exists b \in B$, классифицирующий алгоритм лучше, чем наугад: $P(b;U^l)>N(b;U^l$ \\
Тогда минимум функционала $\hat{Q}_T$ достигается при
$$ b_{\tau} = arg \max_{\substack{b}} \sqrt{P(b;\hat{W}^l)} - \sqrt{N(b;\hat{W}^l)} $$
$$ a_{\tau} = \frac{1}{2} ln \frac{{P(b;\hat{W}^l)}}{{N(b;\hat{W}^l)}} $$
\end{theorem}

\subsection{Алгоритм AdaBoost}
Классический вариант AdaBoost: \\
Пусть отказов нет, $b_t: X \rightarrow {-1;+1}$ => P=1-N
\begin{theorem} \textbf{Основная теорема бустинга} \\
Пусть для любого нормированного вектора весов $U^l \ \exists b \in B$, классифицирующий алгоритм лучше, чем наугад: $P(b;U^l)>N(b;U^l$ \\
Тогда минимум функционала $\hat{Q}_T$ достигается при
$$ b_T = arg \min_{\substack{b}} N(b;\hat{W}^l) $$
$$ a_{\tau} = \frac{1}{2} ln \frac{1-N(b_T;\hat{W}^l)}{N(b_t;\hat{W}^l)} $$
\end{theorem}
\begin{corollary}
Если на каждом шаге семейство В и метод обучения обеспечивают построение базового алгоритма, такого что
$$ \sqrt{P(b_T, \hat{W}^l)} - \sqrt{N(b_t, \hat{W}^l)} = \gamma_{\tau} > \gamma$$
при некотором $\gamma>0$, то за конечное число шагов будет построен корректный алгоритм a(x)
\end{corollary}

Алгоритм AdaBoost: \\
Вход: обучающая выборка, параметр Т\\
Выход: базовые алгоритмы и их веса $\alpha_t b_t, t=1,\ldots,T$ \\
\begin{enumerate}
	\item Инициализировать веса объектов \\
	$w_i=1/l$
	\item $\forall t = 1,\ldots,T$
	\begin{itemize}
		\item обучить базовый алгоритм
		\item $b_t = arg \min_{\substack{b}} N(b;W^l)$
		\item $\alpha_t = \frac{1}{2} ln \frac{1-N(b_t;W^l)}{N(b_t;W^l)}$
		\item обновить веса объектов: $w_i = w_i \ exp(-\alpha_t y_i b_t(x_i))$
		\item нормировать веса объектов: \\
		$ w_0 = \sum_{j=1}^l w_j$ \\
		$w_i = w_i/w_0$
	\end{itemize}
\end{enumerate}

\end{document}