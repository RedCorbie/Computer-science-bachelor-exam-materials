\documentclass[a4paper]{article}

\usepackage[utf8]{inputenc}
\usepackage[english, russian]{babel}
\usepackage[fleqn]{amsmath}
\usepackage{amsfonts, amssymb, amsthm, mathtools}

\usepackage{mathtools}
\usepackage{fullpage}
\usepackage[colorinlistoftodos]{todonotes}


\title{Формальные языки и трансляции}
\author{MIPT DIHT}

\theoremstyle{plain}
\newtheorem*{theorem-star}{Theorem}
\newtheorem{theorem}{Theorem}
\newtheorem*{lem-star}{Lemma}
\newtheorem{lem}{Lemma}
\newtheorem*{proposition-star}{Proposition}
\newtheorem{proposition}{Proposition}

\theoremstyle{remark}
\newtheorem*{remark}{Remark}

\theoremstyle{definition}
\newtheorem*{definition-star}{Definition}
\newtheorem{definition}{Definition}

\renewenvironment{proof}{{\bfseries Proof}}{$\bullet$}

\newcommand{\myequat}[1]{\begin{equation} #1 \nonumber \end{equation}}
\newcommand{\pars}[1]{\left( #1 \right)} 
\newcommand{\class}[1]{\left[ #1 \right]} 
\newcommand{\myN}{\mathbb{N}} 
\newcommand{\myZ}{\mathbb{Z}}
\newcommand{\myR}{\mathbb{R}}
\newcommand{\myC}{\mathbb{C}}
\newcommand{\myQ}{\mathbb{Q}}
\newcommand{\myE}{\mathcal{E}} 
\newcommand{\myM}{\mathcal{M}}
\newcommand{\myO}{(1+o(1))}

\begin{document}
\maketitle

\section{Билет №1}
\subsection{Недетерминированный конечный автомат}
\subsubsection{Конфигурация автомата, начальная конфигурация, такт}
\subsubsection{Допуск слова автоматом. Язык определяемый автоматом. Граф переходов автомата}
\subsection{Детерминированный конечный автомат}
\subsection{Эквивалентность НКА и ДКА (с доказательством)}

\section{Билет №2}
\subsection{Иерархия грамматик Хомского. Нормальная грамматика Хомского}
\subsection{Алгоритм устранения $\varepsilon$-правил}
\subsection{Алгоритм устранения цепных правил}
\subsection{Алгоритм преобразования грамматики к нормальной форме Хомского}

\section{Билет №3}
\subsection{Нисходящий синтаксический анализ. Множества FIRST и FOLLOW}
\subsection{LL(1)-грамматики}
\subsection{Предиктивный синтаксический анализатор, управляемый таблицей}
\subsection{Построение ТПСА для LL(1)-грамматик}

\section{Билет №4}
\subsection{Восходящий синтаксический анализ}
\subsection{Синтаксический анализ «перенос-свертка»}
\subsection{Конфликты в процессе ПС-анализа}
\subsection{Алгоритм построения множеств FIRST и FOLLOW}

\section{Билет №5}
\subsection{Общее устройство и алгоритм LR-анализатора}
\subsection{LR(0)- и LR(1)-пункты. Функции CLOSURE и GOTO для LR(0)- и LR(1)-пунктов}
\subsection{Построение множеств LR(1)-пунктов. Построение таблиц ACTION и GOTO для LR(1)-анализатора}

\end{document}