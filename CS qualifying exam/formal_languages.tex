\documentclass[a4paper]{article}

\usepackage[utf8]{inputenc}
\usepackage[english, russian]{babel}
\usepackage[fleqn]{amsmath}
\usepackage{amsfonts, amssymb, amsthm, mathtools}

\usepackage{mathtools}
\usepackage{fullpage}
\usepackage[colorinlistoftodos]{todonotes}


\title{Формальные языки и трансляции}
\author{MIPT DIHT}

\theoremstyle{plain}
\newtheorem*{theorem-star}{Theorem}
\newtheorem{theorem}{Theorem}
\newtheorem*{lem-star}{Lemma}
\newtheorem{lem}{Lemma}
\newtheorem*{proposition-star}{Proposition}
\newtheorem{proposition}{Proposition}

\theoremstyle{remark}
\newtheorem*{remark}{Remark}

\theoremstyle{definition}
\newtheorem*{definition-star}{Definition}
\newtheorem{definition}{Definition}

\renewenvironment{proof}{{\bfseries Proof}}{$\bullet$}

\newcommand{\myequat}[1]{\begin{equation} #1 \nonumber \end{equation}}
\newcommand{\pars}[1]{\left( #1 \right)}
\newcommand{\class}[1]{\left[ #1 \right]}
\newcommand{\myN}{\mathbb{N}}
\newcommand{\myZ}{\mathbb{Z}}
\newcommand{\myR}{\mathbb{R}}
\newcommand{\myC}{\mathbb{C}}
\newcommand{\myQ}{\mathbb{Q}}
\newcommand{\myE}{\mathcal{E}}
\newcommand{\myM}{\mathcal{M}}
\newcommand{\myO}{(1+o(1))}

\begin{document}
\maketitle

\section{Билет №1}
\subsection{Недетерминированный конечный автомат}
\subsubsection{Конфигурация автомата, начальная конфигурация, такт}
\begin{definition-star} Недетерминированный конечный автомат (далее НКА)  $M$ - пятерка $<Q,\Sigma,\delta,q_0,F>$, где:\\
$Q$ - конечное множество состояний\\
$\Sigma$ - алфавит входных символов\\
$\delta$ - отображение $Q*\Sigma \rightarrow 2^Q$(функция переходов)\\
$q_0 \in Q$ - начальное состояние\\
$F \subseteq Q $ - мн-во конечных состояний.
\end{definition-star}
\begin{definition-star} Пара $(q,a) \in Q*\Sigma^*$ - конфигурация автомата
\end{definition-star}
\begin{definition-star} Пара $(q_0,a) \in Q*\Sigma^*$ - начальная конфигурация автомата
\end{definition-star}
\begin{definition-star}  Такт автомата - бинарное отношение, определенное на конфигурации. Если $q' \in \delta(q,aw)$, то $(q,aw) \vdash_M (q',w)  \forall w \in \Sigma^*$
\end{definition-star}
\subsubsection{Допуск слова автоматом. Язык определяемый автоматом. Граф переходов автомата}
\begin{definition-star} Автомат $M$ допускает слово $w$, если  $(q_0,w) \vdash^*_M (q,\varepsilon)$, где $q \in F$
\end{definition-star}
\begin{definition-star} Язык $L(M)$ определяемый автоматом $М$: $L(M)=\{w| w \in \Sigma^*: (q_0,w) \vdash^*_M (q,\varepsilon) \} $где $q \in F$
\end{definition-star}
\begin{definition-star}  Граф переходов автомата $М$ - граф, вершины которого помечены состояниями, и в котором есть дуга $(q,a)$ если  $\exists  a \in \Sigma : q \in \delta(p,a)$.Кроме того, дуга $(q,a)$ помечена спиком таких $a$, что $q \in \delta(p,a)$
\end{definition-star}
\subsection{Детерминированный конечный автомат}
\begin{definition-star} пусть $M$ - НКА. $M$ называется Детерминированным конечным автоматом, если $|\delta(q,a)|\le 1 \ \forall q\in Q, \forall a \in \Sigma^*$
\end{definition-star}
\subsection{Эквивалентность НКА и ДКА (с доказательством)}
\begin{theorem} Если $L=L(M)$ для некоторого НКА, то 	$\exists$ $M'$ ДКА : $L=L(M')$
\end{theorem}
\begin{proof} Пусть $M = <Q,\Sigma,\delta,q_0,F>$. Построим $M'=<Q',\Sigma,\delta',q_0',F'>$:\\
$Q'=2^Q$\\
$q_0'=\{q_0\}$\\
$F'=\{q|q \in Q',\exists p\in F: p\in q\}$\\
$\delta'(q',a)=\{p|p\in \delta(q,a),\forall q\in q'\}$\\
\\
1. Докажем, что любое слово, принимаемое НКА, будет принято ДКА.\\
Заметим, что $\forall q \in q', \forall a \in \Sigma, \forall p \in \delta(q,a) \ p \in \delta'(q',a)$\\
Рассмотрим слово $w=w_1w_2w_3...w_n$, которое принимает НКА.\\
$(q_0,w_1..w_n) \vdash_M (u_1,w_2..w_n) \vdash^*_M(u_m,\varepsilon) $\\
Проверим, что  построенное ДКА тоже  принимает это слово.\\
$q_0 \in q' \Rightarrow u_1 \in u_1' $ где $ u_1' \in\delta'(q_0',w_1)$\\
Далее $\forall i \le m:  u_i \in u_i', (q_0',w_1..w_n) \vdash(u_1',w_2..w_n) \vdash^*(u_i',w_{i+1}..w_n)$
Т.о. $u_m \in u_m', u_m' \in F'$, ДКА принимает слово $w$.\\
\\
2.Докажем, что любое слово, принимаемое ДКА, будет принято НКА.\\
Заметим, то если $q'=\{q\}$, и из него мы достигли состояния $p'$ по строке $S$, то  $\forall p \in p'$  существует путь  из $q$ в $p$ в НКА по строке  $S$.(2)\\
Рассмотрим слово $w_1..w_n$, принимаемое ДКА.\\
$(q_0',w_1..w_n) \vdash (u_1',w_2..w_n) \vdash^*(u_m',\varepsilon), u_m' \in F' $\\
Проверим, что НКА также принимает это слово.\\
Т.к. $q_0'=\{q_0\}$,  и мы достигли из $q_0'\  u_m' \in F'$,  то возьмем любое состояние $u_m \in u_m'$. По (2) в НКа будет существовать путь из $q_0$ в $u_m$ по строке $w \Rightarrow$ НКА принимает это слово.\end{proof}
\section{Билет №2}
\subsection{Иерархия грамматик Хомского. Нормальная грамматика Хомского}
\subsubsection {Порождающая грамматика}
\begin{definition-star} Порождающая грамматика (типа 0) - четверка $G=<N,\Sigma,P,S>$, где
$N$ - конечный нетерминальный алфавит, $\Sigma$ - конечный терминальный алфавит,$N\cap\Sigma=\emptyset$ ,$S \in N$ - начальный символ
$P \subset (N\cup\Sigma)^+*(N\cup\Sigma)^*$конечно.
\end{definition-star}
Пары $(\alpha,\beta)\in P$ - правила. $\alpha \in(N\cup\Sigma)^+, \beta \in(N\cup\Sigma)^*,\alpha\rightarrow\beta$
$w_0\Rightarrow_G w_1 \Rightarrow_G w_2 ..\Rightarrow_G w_n,\ n>0$,то $w_0\Rightarrow^*_G w_n$
$w_0,w_1,..,w_n$- вывод слова $w_n$ из $w_0$ в $G$
\begin{definition-star}Контекстная грамматика типа 1 (Контекстно-зависимая) - Порождающая грамматика , каждое правило которой имеет вид $\eta A \theta \rightarrow \eta \alpha \theta$, где  $A \in N, \eta,\theta \in (N\cup\Sigma)^*, \alpha \in (N\cup \Sigma)^+$
\end{definition-star}
\begin{definition-star}Контекстная грамматика типа 2 (Контекстно-свободная) - Порождающая грамматика , каждое правило которой имеет вид $A \rightarrow  \alpha$, где  $A \in N,  \alpha \in (N\cup \Sigma)^*$
\end{definition-star}
\begin{definition-star}Контекстная грамматика типа 3 (Праволинейная) - Порождающая грамматика , каждое правило которой имеет вид $A \rightarrow  u$ или $A \rightarrow  uB$, где  $A,B \in N,  u\in \Sigma^*$
\end{definition-star}
\begin{definition-star}Назовем $X \in N$ бесполезным симвом, если $L(G')=\emptyset$, где $G'=<N,\Sigma,P,X>$
\end{definition-star}
\begin{definition-star}Назовем $X \in N\cup \Sigma$ недостижимым симвом, если не существует вывода, где этот символ встречается.
\end{definition-star}
Алгоритм:
Вход: КС грамматика $G=<N,\Sigma,P,S>$
Выход: КС грамматика $G'=<N',\Sigma',P',S'>$т.ч.
a)$L(G)=L(G')$
б)любой символ достижим
1)$V_0=\{S\},i=1$
2)$V_i=\{X|A\rightarrow\alpha X \beta \in P \ и \ A\in V_{i-1}\}\cup V_{i-1}$
3)Если $V_i\ne V_{i-1}:  i++; goto $ 2)
4)$N'=N\cap V_i, \Sigma'=\Sigma \cap V_i$
$P'$ содержит те правила из $P$, что содержат символы из $V_i$
\subsection{Алгоритм устранения $\varepsilon$-правил}
\begin{definition-star}$G=<N,\Sigma,P,S>$ - КС грамматика без $\varepsilon$-правил, если $P$ не содержит $\varepsilon$-правил, или содержит $\{S\rightarrow \varepsilon\}$, но нет правил, где $S$ содержится в правых частях.
\end{definition-star}
Алгоритм:
Вход: КС грамматика $G=<N,\Sigma,P,S>$
Выход: КС грамматика $G'=<N',\Sigma',P',S'>$
1)построить $N_\varepsilon = \{A|A\in N,A \Rightarrow_G^* \varepsilon\}$
а)$N_0=\{A|(A\rightarrow \varepsilon) \in P\}$
б)$N_i = \{A|A\in N,(A \rightarrow w)\in P, w \in N_{i-1}^*\} \cup N_{i-1}$
в)Если$N_i = N_{i-1}$, то $N_i=N_\varepsilon$, иначе пункт б)

2)Построить $P'$
а)Если $(A\rightarrow \alpha_0B_1\alpha_1...B_k\alpha_k)\in P,\  \forall i,j \ B_i \in N_\varepsilon$ и $\alpha_i \in (\Sigma \cup N\setminus N_\varepsilon)^*$
то включить в $P'$ все правила вида $(A\rightarrow \alpha_0X_1\alpha_1...X_k\alpha_k)$где $X_i=B_i$ или $X_i=\varepsilon$ , но без $\varepsilon$-правил.
б) Если $S\in N_\varepsilon$, то включить в $P'$ правила  $S'\rightarrow\varepsilon|S$ и $N'=N\cup\{S'\}$, иначе $N=N',S=S'$
\subsection{Алгоритм устранения цепных правил}
Алгоритм:
Вход: КС грамматика $G=<N,\Sigma,P,S>$ без эпсилон-правил
Выход: КС грамматика $G'=<N',\Sigma',P',S'>$без эпсилон и цепных правил
1)построить $ \forall A \in N \  N_A = \{B|B\in N,A \Rightarrow^*B\}$
а)$N_{A_0}=\{A\}$
б)$N_{A_i}=\{C|(B\rightarrow C)\in P , B\in N_{A_i-1}\}\cup N_{A_i-1}$
в)Если $N_{A_i-1}\ne N_A$  повторить пункт б), иначе остановка
2)построить $P'$:
если $(B\rightarrow \alpha)\in P$ и не является цепным правилом, то  включить в $P'$ все правила вида $A\rightarrow \alpha, \forall A: B\in N_A$
\subsection{Алгоритм преобразования грамматики к нормальной форме Хомского}
\begin{definition-star}КС грамматика называется грамматикой в НФХ, если  каждое правило из $P$  имеет один из следующих видов:
1)$A\rightarrow BC; A,B,C\in N$
2)$A\rightarrow a; A\in N, a\in \Sigma$
3)$S'\rightarrow \varepsilon$ если  $\varepsilon \in L(G)$ причем $S$  не встречается  в правых частях правил при подстановке
\end{definition-star}
Алгоритм:
Вход:Приведенная КС грамматика $G=<N,\Sigma,P,S>$
Выход: КС грамматика в НФХ $G'=<N',\Sigma,P',S'>$
1)включить в $P'$ все правила вида $A\rightarrow a$
2)включить в $P'$ все правила вида $A\rightarrow BC$
3)включить в $P'$ $(S\rightarrow \varepsilon)$ если оно было в $P$
4)для каждого правила из $P$ вида
$A\rightarrow X_1..X_k\  k>2$ включить в $P'$ правила вида:
$A\rightarrow X_1'X_2..X_k$
$X_2..X_k\rightarrow X_2'X_3..X_k$
...
$X_{k-1}X_k\rightarrow X'_{k-1}X'_k$ где $X'_i=X_i$  если  $X_i \in N$ или  $X_i'$- новый нетерминал если $X_i \in \Sigma$
5)Для каждого нетерминала  вида $X_i'$ добавить в $P'$  правило:
$X'_i\rightarrow X_i ,X_i \in \Sigma$
6)$N'=N+$ все введенные новые нетерминалы
\section{Билет №3}
\subsection{Нисходящий синтаксический анализ. Множества FIRST и FOLLOW}
\begin{definition-star}
1)НСА можно рассматривать как задачу построения дерева разбора, начиная с корня, и связывая узлы в порядке левого разбора.
2)На каждом шаге ключевой проблемой является определение продукции, применимой для нетерминала. Отсальная часть процесса - проверка соответствий
Предиктивный синтаксический анализатор может выбрать нужную продукцию, просматривая несколько непрочитанных символов.
\end{definition-star}
\begin{definition-star}
$LL(k)$ - класс грамматик, для которых можно построить предиктивный синтаксический анализатор, просматривающий не более $k$ символов вперед.
\end{definition-star}
\begin{definition-star}
$FIRST(\alpha), \alpha \in (N\cup \Sigma)^*$ - множество терминалов, с которых начинаются строки, порождаемые $\alpha$. Если $\alpha\Rightarrow^*\varepsilon$, то $\varepsilon \in FIRST(\alpha)$
\end{definition-star}
\begin{definition-star}
$Follow(A), A \in N$ - множество терминалов, таких что существует порождение вида $S\Rightarrow^*\alpha Aa\beta$
\end{definition-star}
Алгоритм FIRST
Будем придерживаться следующих правил, пока ни к одному из мн-в $FIRST$ не могут быть добавлены ни $\varepsilon$, ни нетерминал.
$FIRST(X)$
*если $X$-терминал, то $FIRST(X)=\{X\}$
*если $X\rightarrow\varepsilon\in P$, то добавим $\varepsilon$ в $FIRST(X)$
*если $X\in N$ и имеется продукция вида $X\rightarrow Y_1Y_2Y_3$, то поместим $a$ в $FIRST(X)$, если $\exists i: a \in FIRST(Y_i)$ и $\varepsilon \in FIRST(Y_j) j=1...i-1$
*если $\varepsilon \in FIRST(Y_i) \forall i=1..k$, то добавить $\varepsilon$ в $FIRST(X)$
Алгоритм FOLLOW
будем применять правила, пока ни в одно из множеств $FOLLOW(A)$ нельзя будет добавить ни одного символа.
*поместим @ в $FOLLOW(S)$, $S$ - аксиома, @ - правый ограничитель входного потока.
*если имеется продукция $A\rightarrow\alpha B\beta$, то $FIRST(\beta)\setminus\{\varepsilon\}$ помещаем в $FOLLOW(B)$
*если имеется продукция $A\rightarrow\alpha B$ или $A\rightarrow\alpha B\beta$, где$\varepsilon \in FIRST(\beta)$, то $FOLLOW(A)$ помещаем в $FOLLOW(B)$
\subsection{LL(1)-грамматики}
\begin{definition-star} КС-грамматика G принадлежит классу $LL(1)$ т. и т.т, когда $\forall$  2ух продукций из $P$ $A\rightarrow\alpha|\beta$:
*$FIRST(\alpha)\cap FIRST(\beta)=\emptyset$
*либо если $\varepsilon \in FIRST(\beta), то FIRST(\alpha)\cup Follow(A)=\emptyset$
*либо если $\varepsilon \in FIRST(\alpha), то FIRST(\beta)\cup Follow(A)=\emptyset$
\end{definition-star}
\subsection{Предиктивный синтаксический анализатор, управляемый таблицей}
Алгоритм ПСА для $LL(1)$-грамматики G
Вход: $w@$ - входная строка
М-ТПСА для $LL(1)$-грамматики $G$
Изначальное содержимое стека $S@$
$X$-символ на вершине стека.
$a$-текущий входной символ.
while ($x\ne @$){
    if ($x=a$){
        pop();
        переход к следующему символу
        }
        else if($x \in\Sigma$){Error();}
        else if($|M(x,a)|=0$){Error();}
        else if ($M(x,a)=X\rightarrow Y_1..Y_k$){
        выводим $X\rightarrow Y_1..Y_k$;
        pop();
        push($Y_1..Y_k$);
        }
\subsection{Построение ТПСА для LL(1)-грамматик}
\begin{definition-star}
Таблица предиктивного синтаксического анализатора (ТПСА) - двумерный массив $M(A,a),A\in N, a \in \Sigma\cup\{@\}$.
Ячейка $M(A,a)$ - некоторое множество $A$-продукций.
Для $LL(1)$-грамматики каждая ячейка содержит не более одной продукции.
\end{definition-star}
Алгоритм построения ТПСА:
Для каждой продукции $A\rightarrow\alpha$ нужно сделать
1)$\forall$ терминала $a$ из $FIRST(\alpha)$ добавляем $A\rightarrow\alpha$ в $M(A,a)$
2)Если $\varepsilon \in FIRST(\alpha)$, то $\forall$ терминала $b$ из $FOLLOW(A)$ добавляем $A\rightarrow\alpha$ в $M(A,b)$
3)Если $\varepsilon \in FIRST(\alpha)$ и $@ \in FOLLOW(A)$, то добавляем $A\rightarrow\alpha$ в $M(A,@)$
\section{Билет №4}
\subsection{Восходящий синтаксический анализ}
\begin{definition-star}
Восходящий синтаксический анализ соответствует построению дерева разбора, начиная с листьев по направлению к корню.
\end{definition-star}
\subsection{Синтаксический анализ «перенос-свертка»}
Можно рассматривать Восходящий синтаксический анализ как процесс свертки строки w к стартовому символу. На каждом шаге свертки определенная строка, соответствующая телу продукции, заменяется нетерминалом.
\begin{definition-star}
Перенос-свертка - разновидность восходящего анализа, в котором для хранения символов грамматики используется стек.
Изначально во входном буфере находится входная строка.
В процессе ее сканирования СА выполняет 0 или более переносов символов в стек. Затем выполняется свертка. Процесс выполняется, пока не всплывет ошибка или достигнется конечное состояние (т.е. стартовый символ).
\end{definition-star}
\subsection{Конфликты в процессе ПС-анализа}
1)Конфликт "Перенос-свертка":
Анализатор не может понять, увуое из двух действий ему выполнить.
2)Конфликт "Свертка-свертка":
Анализатор не может понять, по какой продукции следовать
\subsection{Алгоритм построения множеств FIRST и FOLLOW}
Алгоритм FIRST
Будем придерживаться следующих правил, пока ни к одному из мн-в $FIRST$ не могут быть добавлены ни $\varepsilon$, ни нетерминал.
$FIRST(X)$
*если $X$-терминал, то $FIRST(X)=\{X\}$
*если $X\rightarrow\varepsilon\in P$, то добавим $\varepsilon в FIRST(X)$
*если $X\in N$ и имеется продукция вида $X\rightarrow Y_1Y_2Y_3$, то поместим $a$ в $FIRST(X)$, если $\exists i: a \in FIRST(Y_i)$ и $\varepsilon \in FIRST(Y_j) j=1...i-1$
*если $\varepsilon \in FIRST(Y_i) \forall i=1..k$, то добавить $\varepsilon в FIRST(X)$
Алгоритм FOLLOW
будем применять правила, пока ни в одно из множеств $FOLLOW(A)$ нельзя будет добавить ни одного символа.
*поместим @ в $FOLLOW(S)$, $S$ - аксиома, @ - правый ограничитель входного потока.
*если имеется продукция $A\rightarrow\alpha B\beta$, то $FIRST(\beta)\setminus\{\varepsilon\}$ помещаем в $FOLLOW(B)$
*если имеется продукция $A\rightarrow\alpha B$ или $A\rightarrow\alpha B\beta$, где$\varepsilon \in FIRST(\beta)$, то $FOLLOW(A)$ помещаем в $FOLLOW(B)$
\section{Билет №5}
\subsection{Общее устройство и алгоритм LR-анализатора}
\begin{definition-star}Функция $ACTION(i,a)$ принимает состояние $i$ (состояния хранятся на стеке) и символ $a$.
$ACTION(i,a)$ может равняться:
1)переносу $j$, $j$-состояние
2)свертке $A\rightarrow\beta$
3)принятию
4)ошибке
\end{definition-star}
\begin{definition-star}$GOTO(i,a)=j$, где $i$,$j$ - состояния, $a$-символ
\end{definition-star}
Алгоритм $LR$-анализ
Вход: строка $w$ и таблицы $ACTION$ и $GOTO$ для КС-грамматики $G$
Выход: Если $w\in L(R)$ - шаги сверток, иначе сообщение об ошибке.
Изначально в стеке находится начальное состояние $О$, а во входном буфере $w@$
Пусть $a$ - текущий символ входной строки, $s$ - состояние на вершине стека.
1)В зависимости от $ACTION(s,a)$:
*перенос $t$ - положить $t$ в стек и перейти к следующему входному символу.
*свертка $A\rightarrow\beta$ - снять $|\beta|$ состояний со стека. Пусть теперь $t$ на вершине стека. Положить в стек $GOTO(t,A)$. Вывести свертку $A\rightarrow\beta$
*если слово принято, то алгоритм завершает работу.
*если ошибка, алгоритм завершает работу с уведомлением об ошибке.
2)Перейти к 1)
\subsection{LR(0)- и LR(1)-пункты. Функции CLOSURE и GOTO для LR(0)- и LR(1)-пунктов}
\begin{definition-star}
$LR(0)$-пункт - продукция с точкой.
\end{definition-star}
\begin{definition-star}
Функция $CLOSURE(I)$, где $I$ - множество $LR(0)$-пунктов
1)В $CLOSURE(I)$ добавляем $I$
2)Если $(A\rightarrow \alpha\cdot B\beta)\in CLOSURE(I)$ и $(B\rightarrow\gamma)$, то в $CLOSURE(I)$ добавляем $(B\rightarrow\cdot\gamma)$
3)Повторяем 2) пока $CLOSURE(I)$ изменяется
\end{definition-star}
\begin{definition-star}
$GOTO(I,X)$, где $X$ - символ грамматики -  замыкание мн-ва пунктов вида $(A\rightarrow\alpha X\dot\beta)$ таких, что $(A\rightarrow\alpha X\cdot\beta)\in I$
\end{definition-star}
\begin{definition-star}
$LR(0)$-автомат - конечный автомат с начальным состоянием $CLOSURE(\{s'\rightarrow\dot s\})$ и функцией переходов $GOTO$
Множество состояний $LR(0)$-автоматов - канонический набор $LR(0)$
\end{definition-star}
\begin{definition-star}$LR(1)$-пункт - $[A\rightarrow\alpha\cdot\beta,a]$, где $A\rightarrow\alpha\cdot\beta$ - $LR(0)$ пункт, а $a\in\Sigma\cup\{@\}$
\end{definition-star}
\begin{definition-star}$CLOSURE(I)$
1)$\forall [A\rightarrow\alpha\cdot\beta,a]\in I$
$\forall B\rightarrow\gamma$
$\forall b \in FIRST(\beta a)$
добавить $[B\rightarrow\cdot\gamma,b]$ в $I$
2)повторять 1) пока I изменяется
\end{definition-star}
\begin{definition-star}$GOTO(I,X)$ $I$ - мн-во пунктов, $X$ - любой символ
1) $J=\{\}$
2)$\forall [A\rightarrow\alpha\cdot X\beta,a] \in I$, добавим
$[A\rightarrow\alpha X\cdot\beta,a]$ в J
3)вернуть $CLOSURE(J)$
\end{definition-star}

\subsection{Построение множеств LR(1)-пунктов. Построение таблиц ACTION и GOTO для LR(1)-анализатора}
Построение мн-в $LR(1)$-пунктов
1)$C=\{CLOSURE(\{s'\rightarrow\cdot s,@\})\} CLOSURE=I_0$
2)$\forall I \in C, \forall x \in \Sigma\cup N$
if $GOTO(I,x)=\emptyset$ и $GOTO(I,x)\notin C$
добавить $GOTO(I,x)$ в $C$

Построение $LR(1)$ функций $ACTION$ и $GOTO$
1) Построить $C$-набор мн-в $LR(1)$-пунктов
2)$\forall I_i\in C$ построить состояние $i$
а)$GOTO(I_i,a)=I_j$
б)if $[A\rightarrow\alpha\cdot,a]\in I_i$ и $A\ne S$
$ACTION(i,a)= A\rightarrow\alpha$
в)if $[s'\rightarrow s\cdot,@] \in I_i$
$ACTION (i,@)="принятие"$
3)if $GOTO(I_i,A)=I_j$, то $GOTO(i,A)=j$ где $A \in N$
4)Начальное состояние $I_0$
5) Во всех остальных случаях - ошибка.
\end{document} 