\documentclass[a4paper]{article}

\usepackage[utf8]{inputenc}
\usepackage[english, russian]{babel}
\usepackage[fleqn]{amsmath}
\usepackage{amsfonts, amssymb, amsthm, mathtools}

\usepackage{mathtools}
\usepackage{fullpage}
\usepackage[colorinlistoftodos]{todonotes}


\title{Математическая логика и теория алгоритмов}
\author{MIPT DIHT}

\theoremstyle{plain}
\newtheorem*{theorem-star}{Theorem}
\newtheorem{theorem}{Theorem}
\newtheorem*{lem-star}{Lemma}
\newtheorem{lem}{Lemma}
\newtheorem*{proposition-star}{Proposition}
\newtheorem{proposition}{Proposition}

\theoremstyle{remark}
\newtheorem*{remark}{Remark}

\theoremstyle{definition}
\newtheorem*{definition-star}{Definition}
\newtheorem{definition}{Definition}

\renewenvironment{proof}{{\bfseries Proof}}{$\bullet$}

\newcommand{\myequat}[1]{\begin{equation} #1 \nonumber \end{equation}}
\newcommand{\pars}[1]{\left( #1 \right)} 
\newcommand{\class}[1]{\left[ #1 \right]} 
\newcommand{\myN}{\mathbb{N}} 
\newcommand{\myZ}{\mathbb{Z}}
\newcommand{\myR}{\mathbb{R}}
\newcommand{\myC}{\mathbb{C}}
\newcommand{\myQ}{\mathbb{Q}}
\newcommand{\myE}{\mathcal{E}} 
\newcommand{\myM}{\mathcal{M}}
\newcommand{\myO}{(1+o(1))}

\begin{document}
\maketitle

\section{Билет №1}
\subsection{Множества и подмножества. Операции над множествами, тождества.}
\subsection{Отображения и соответствия. Сравнения мн-в по мощности}
\subsection{Теорема Кантора-Бернштейна}
\subsection{Счетные множества и их св-ва. Теорема Кантора}

\section{Билет №2}
\subsection{Булевы функции и пропозициональные формулы. КНФ, ДНФ. Тавтологии}
\subsection{Исчисление высказываний: аксиомы, правила вывода, определение выводимости. Корректность исчисления высказываний}
\subsection{Лемма о дедукции}
\subsection{Полнота исчисления высказываний}

\section{Билет №3}
\subsection{Языки первого порядка: сигнатуры, термы, правила построения формул}
\subsection{Интерпретация, оценки, определение истинности формулы}
\subsection{Выразимость предикатов}

\section{Билет №4}
\subsection{Общезначимые формулы первого порядка. Исчисление предикатов: формулы и правила вывода}
\subsection{Корректность исчисления предикатов}
\subsection{Теорема Геделя о полноте исчисления предикатов: формулировки, схема доказательства}

\section{Билет №5}
\subsection{Машины Тьюринга. Вычислимые функции}
\subsection{Разрешимые и перечислимые множества}
\subsection{Неразрешимость проблем самоприменимости и остановки}
\subsection{Теорема Райса-Успенского (б/д)}
\subsection{Теорема Клини о неподвижной точке (б/д)}
\subsection{Существование программы, печатающей собственный текст}

\section{Билет №6}
\subsection{Формальная арифметика}
\subsection{Моделирование МТ в формальной арифметике}
\subsection{Теорема Геделя и неполноте (б/д)}

\section{Билет №7}
\subsection{Лямбда-исчисление. Лямбда-термы и комбинаторы}
\subsection{Теорема Черча-Россера (б/д). Нумералы Черча}
\subsection{Комбинаторы, представляющие арифметические операции. Представление логических значений и операций}
\subsection{Комбинатор неподвижной точки}

\section{Билет №8}
\subsection{Классы сложности P и NP}
\subsection{NP-hard и NP-complete}

\end{document}