\documentclass[a4paper]{article}
\usepackage{fontspec} 
\usepackage{polyglossia}
\setmainlanguage{russian} 
\setotherlanguage{english}
\newfontfamily{\cyrillicfont}{Times New Roman}

\usepackage{mathtools}
\usepackage{fullpage}
\usepackage[utf8x]{inputenc}
\usepackage{amsmath}
\usepackage[colorinlistoftodos]{todonotes}


\title{Теория вероятностей и математическая статистика}
\author{MIPT DIHT}
\begin{document}
\maketitle

\section{Билет №1}
\subsection{Вероятностное пространство, аксиомы Колмогорова, свойства вероятностной меры}
\subsection{Условные вероятности. Формула полной вероятности. Формула Байеса}

\section{Билет №2}
\subsection{Случайные величины и векторы. Их характеристики: распределение вероятностей, функция распределения, ее свойства, $\sigma$=алгера, порожденная с. в.}
\subsection{Примеры конкретных распределений}

\section{Билет №3}
\subsection{Матожидание случайной величины: опр-ние для простых, неотрицательных, произвольних с.в.}
\subsection{Основные свойства матожидания (док-ва только для простых с.в.)}
\subsection{Дисперсия, ковариация, их св-ва}

\section{Билет №4}
\subsection{Сходимость случайных величин: по вероятности, по распределению, почти наверное, в среднем}
\subsection{Связь между сходимостями (б/д). Теорема о наследовании сходимости}

\section{Билет №5}
\subsection{Неравенства Маркова и Чебышева. ЗБЧ в форме Чебышева}
\subsection{УЗБЧ(все) (б/д)}

\section{Билет №6}
\subsection{Характеристические функции с.в и векторов. Их св-ва}
\subsection{Теорема непрерывности (б/д)}

\section{Билет №7}
\subsection{ЦПТ для незав. одинаково распр-х с.в.}

\section{Билет №8}
\subsection{Выборка, выборочное пр-во. Точные оценки параметров и их св-ва: смещенность, состоятельность, асимптотическая нормальность}
\subsection{Выборочные среднее, медиана, дисперсия}
\subsection{Сравнение оценок, ф-ция потерь, ф-ция риска}
\subsection{Подходы к сравнению оценок: равномерный, байесовский, минимаксный}

\section{Билет №9}
\subsection{Методы построения оценок: метод моментов и метод максимального правдоподобия}
\subsection{Состоятельность оценки метода моментов}
\subsection{Теорема о св-вах оценок максимального правдоподобия (б/д)}

\section{Билет №10}
\subsection{Доверительные интервалы. Метод центральной статистики}

\section{Билет №11}
\subsection{Статистические гипотезы, ошибки первого и второго рода}
\subsection{Общие принципы сравнения критериев, авномерно наиболее мощные критерии}
\subsection{Лемма Неймана-Пирсона. Построение с ее помощью наиболее мощных критериев}


\end{document}