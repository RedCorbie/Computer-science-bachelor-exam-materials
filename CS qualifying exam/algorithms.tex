\documentclass[a4paper]{article}
\usepackage{fontspec} 
\usepackage{polyglossia}
\setmainlanguage{russian} 
\setotherlanguage{english}
\newfontfamily{\cyrillicfont}{Times New Roman}

\usepackage{mathtools}
\usepackage{fullpage}
\usepackage[utf8x]{inputenc}
\usepackage{amsmath}
\usepackage[colorinlistoftodos]{todonotes}


\title{Алгоритмы и структуры данных}
\author{MIPT DIHT}
\begin{document}
\maketitle

\section{Сортировки}
\subsection{Quick Sort}
\subsection{Merge Sort}
\subsection{Heap Sort}

\section{Hash-table and hash-function}
\label{sec:section_name}

\section{Динамическое программирование}
\label{sec:section_name}
\label{Общая идея. Линейная и матричная динамика. Динамика на отрезках}


\section{Амортизационный анализ}

\section{RMQ and LCA}
\subsection{RQM}
\subsection{LCA: сведение к RQM}
\subsection{Метод двоичного подъема}

\section{Алгоритмы на деревьях}
\subsection{Декартовы деревья}
\subsubsection{Декартово дерево}
\subsubsection{Декартово дерево по неявному ключу}

\subsection{Минимальное основное дерево}
\subsubsection{Алгоритм Прима}
\subsubsection{Алгоритм Крускала}

\section{Минимальные потоки в сети}
\subsection{Метод Форда-Фалкерсона}
\subsection{Метод Эдмондса-Карпа (б/д)}

\section{Алгоритмы на графах}
\subsection{Обход в ширину и глубину}
\subsection{Поиск кратчайших путей в графе}
\subsubsection{Алгоритм Дейкстры}
\subsubsection{Алгоритм Форда-Беллмана}
\subsubsection{Алгоритм Флойда-Уоршелла}

\subsection{Поиск сильносвязных компонент в графе}
\subsection{Мосты и точки сочленения в графе}

\section{STL и стандартные контейнеры}
\subsection{vector, deque, queue, priority\_queue, set, map}
\subsection{Итераторы и компараторы}

\end{document}