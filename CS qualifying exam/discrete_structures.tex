\documentclass[a4paper]{article}

\usepackage[utf8]{inputenc}
\usepackage[english, russian]{babel}
\usepackage[fleqn]{amsmath}
\usepackage{amsfonts, amssymb, amsthm, mathtools}

\usepackage{mathtools}
\usepackage{fullpage}
\usepackage[utf8x]{inputenc}
\usepackage{amsmath}
\usepackage[colorinlistoftodos]{todonotes}


\title{Дискретные структуры}
\author{MIPT DIHT}
\begin{document}
\maketitle

\section{Билет №1}
\subsection{Правила комбинаторики: правила сложения, умножения, принцип Дирихле. Формула включения и исключения}
\subsection{Размещения, сочетания и перестановки. Формула Стирлинга (б/д)}

\section{Билет №2}
\subsection{Размещения, сочетания, перестановки}
\subsection{Формулы для чисел размещения и сочетания с повторениями и без}
\subsection{Бином Ньютона, полиномиальная формула}
\subsection{Простейшие тождества. Оценки биномиальных коэффициентов}

\section{Билет №3}
\subsection{Формульные степенные ряды. Производящие функции и тождества}

\section{Билет №4}
\subsection{Линейные рекуррентные соотношения с постоянными коэффициентами}

\section{Билет №5}
\subsection{Граф, орграф, псевдограф, мультиграф, гиперграф}
$\triangleright$ \textbf{Def:} Граф - множество вершин и неориентированных рёбер.
\\
$\triangleright$ \textbf{Def:} Псевдограф - граф с петлями.
\\
$\triangleright$ \textbf{Def:} Мультиграф - граф с кратными рёбрами.
\\
$\triangleright$ \textbf{Def:} Дерево - связный ациклический граф. Оно же ~--- граф, в котором любые две вершины соединены ровно одним путём; связный граф, в котором вершин на единицу больше, чем рёбер; ациклический граф,  в котором вершин на единицу больше, чем рёбер.
\\
$\triangleright$ \textbf{Def:} Гиперграф - множество вершин и рёбер,  каждое ребро ~--- произвольное подмножество вершин.
\\
$\triangleright$ \textbf{Def:} $k$-однородный гиперграф - каждое ребро содержат ровно $k$ вершин.
\\
$\triangleright$ \textbf{Def:} $t$-пересекающийся гиперграф - любые 2 ребра гиперграфа имеют хотя бы $t$ общих вершин.
\\

\subsection{Маршруты в графах. Степени вершин}

\subsection{Изоморфизм и планарность графов}
\subsection{Эйлеровы и гамильтоновы циклы в графах}
$\triangleright$ \textbf{Def:} Эйлеров цикл (цепь) ~--- цикл (цепь), содержащий все рёбра графа.
\\
$\triangleright$ \textbf{Def:} Эйлеров граф ~--- граф, обладающий эйлеровым циклом.
\\
$\triangleright$ \textbf{Def:} Гамильтонов цикл (цепь) ~--- цикл (цепь), содержащая все вершины по одному разу.
\\

\subsection{Критерий Эйлеровости. Достаточное условие гамильтоновости.}
$\triangleright$ \textbf{Th:}  Связный (мульти)граф является эйлеровым (1) тогда и только тогда, когда степень каждой вершины чётна (2), или тогда и только тогда, когда множество рёбер графа можно покрыть без пересечений простыми циклами (3).
\\
$\circ$  $(1)\Rightarrow(2)$: если степень какой-либо вершины нечётна, то мы, двигаясь в порядке рёбер эйлерова цикла, не сможем в какой-то момент войти в эту вершину по одному ребру и выйти по другому ребру, поскольку её степень нечётна. Это означает, что наш обход не является циклом. Противоречие.\\
$(3)\Rightarrow(1)$: объединение всех этих простых циклов является эйлеровым циклом.\\
Что мы подразумеваем под словом "объединение"? Давайте рассмотрим это как последовательный процесс: на нулевом шаге мы рассмотрим любой простой цикл, и будем добавлять к нему простые циклы из числа ещё не задействованных по одному. Таким образом, на каждом шаге мы имеем некоторый цикл и множество (возможно, пустое) тех простых циклов, которые мы ещё не рассмотрели.\\
Пусть это множество непусто. Тогда, так как граф связен, в построенном на данный момент цикле обязательно найдётся вершина, лежащая в одном из незадействованных простых циклов. Обозначим эту вершину $v$, уже построенный нами цикл ~--- $a_1\dots a_i v a_{i+1}\dots a_1$, незадействованный простой цикл ~--- $vb_1b_2\dots b_kv$. Тогда новый цикл мы определим как $a_1\dots a_i v b_1\dots b_k v a_{i+1}\dots a_1$, и мы уменьшили на 1 количество не рассмотренных простых циклов.\\
Пусть это множество оставшихся циклов пусто. По предположению, тогда пусто и множество ребёр, которые лежат вне построенного нами цикла ~--- следовательно, этот цикл эйлеров.\\
$(2)\Rightarrow(3)$: индукция по количеству рёбер.\\
База индукции: если рёбер $0$, то множество рёбер тривиально состоит из нуля простых непересекающихся друг с другом циклов.\\
Переход: выберем произвольную вершину ненулевой степени и пойдём в обход по графу, не проходя дважды одного и того же ребра, пока не вернёмся в какую-либо вершину ~--- таким образом, мы выделили простой цикл. Рёбра этого цикла мы удалим из графа, и чётность степеней всех вершин сохранится, а число рёбер уменьшится.
$\bullet$ \\ 
$\triangleright$ \textbf{Th:}  Слабо связный орграф является эйлеровым тогда и только тогда, когда входящие степени (каждой вершины) равны исходящим.
\\
$\circ$  Аналогично предыдущей теореме.
$\bullet$ \\ 
$\triangleright$ \textbf{Th:} [Критерий Дирака] Если в графе на $n$ вершинах степень каждой вершины не менее $\lceil \frac{n}{2} \rceil$, то граф содержит гамильтонов цикл.
\\
$\circ$  Начнём со вспомогательного утверждения:
$\triangleright$ \textbf{Lem:} Пусть в графе максимальный простой путь состоит из $m$ вершин, и суммарная степень двух концов этого пути не меньше $m$. Тогда в графе существует простой цикл длины $m$.
\\
$\circ$  Обозначим вершины этого пути $a_1,a_2,a_3,\dots,a_m$. Так как путь максимален, то рёбра вида $(a_1,\:v)$ и $(a_m,\:v)$, где $v \notin \{a_i\}_{i=1}^m$, в графе отсутствуют. 
Если вершины $a_1$ и $a_m$ соединены ребром, то искомый цикл найден.\\
Если одновременно есть рёбра $(a_{i+1},\:a_1)$ и $(a_i,a_m)$ (для произвольного $ i \in \overline{2,\, m-2} $ ), то искомый цикл выглядит так: $a_1a_2\dots a_ia_ma_{m-1}\dots a_{i+1} a_1$.
Предположим, что цикла всё же нет. Тогда в силу предыдущего утверждения каждое ребро, проведённое из $a_m$, "запрещает"\ одно ребро из $a_1$, и наоборот (кроме заведомо существующих рёбер $(a_1,\,a_2)$ и $(a_m,\,a_{m-1})$, которые мы сейчас не учитываем). При этом из вершины $a_1$ могут быть рёбра к вершинам $a_3,a_4,\dots,a_{m-1}$ ~--- всего $m-3$ возможности, столько же для $a_m$. Однако эти возможности взаимоисключающие, а нам необходимо (согласно посылке леммы) провести из $a_1$ и $a_m$ суммарно $m-2$ ребра. Противоречие.
$\bullet$ \\ 
Заметим, что граф связен, поскольку суммарная степень любых двух вершин не менее $n$ ~--- это означает, что они либо соединены ребром, либо (по принципу Дирихле) имеют общего соседа.\\
Рассмотрим в нашем графе максимальный простой путь. Согласно лемме, существует простой цикл, проходящий по всем вершинам этого пути (и только по ним). Обозначим его вершины в порядке следования цикла $a_1,a_2,a_3,\dots ,a_m$.\\
Если $m<n$, то рассмотрим любую вершину $v$, не лежащую в цикле. Так как граф связен, для некоторого $i$ существует ребро $(a_i,\:v)$. Тогда путь $va_ia_{i+1}\dots a_ma_1a_2\dots a_{i-1}$ содержит на одну вершину больше, чем рассмотренный нами максимальный. Противоречие.\\
Если же $m=n$, то цикл $a_1a_2a_3\dots a_{m-1}a_ma_1$ ~--- гамильтонов.
$\bullet$ \\ 

$\triangleright$ \textbf{Th:}  Пусть в графе $G$ хотя бы 3 вершины и $k(G)\geq\alpha(G)$. Тогда $G$ содержит гамильтонов цикл.
\\
$\circ$  Если в $G$ нет циклов, то $k(G)\geq\alpha(G)\geq 1 \Rightarrow G$ связен $\Rightarrow k=1, \alpha\geq 2$. Противоречие. Иначе рассмотрим максимальный простой цикл $C = \{v_1,v_2,\dots,v_m\}$ и предположим, что он не гамильтонов, то есть $G\setminus C$ непусто. Пусть $W$ ~--- любая связная компонента $G\setminus C$, $N(W) = \{x\notin W\: |\:\exists y \in W: \: (x,y)\in E(G)\}$. Имеют место следующие утверждения:
\begin{enumerate}
\item $N(W)\subset C$ (сосед компоненты связности, не лежащий в $C$, должен лежать в самом $W$). 
\item Никакие две соседние вершины цикла не лежат в $N(W)$ одновременно. В противном случае для некоторого $i$ в графе есть рёбра $(v_i,x)$, $(y,v_{i+1})$, где $x,y \in W$, а также путь (возможно, нулевой длины) между $x$ и $y$, так как $W$ связно. Тогда, удаляя ребро $(v_i,v_{i+1})$ из $C$ и заменяя его на путь $v_ix\dots yv_{i+1}$, мы получаем цикл большей длины, чем $C$ ~--- значит, $C$ не был максимален.
\item $|N(W)|\geq k(G)$. Действительно, если мы удалим множество $N(W)$ из графа, то $C\setminus N(W)$ и $W$ окажутся в различных компонентах связности.
\item Определим $M = \{v_{i+1}\:|\: v_i\in N(W)\}$ и заметим, что $|M|=|N(W)|$ (по построению).
\item $M \cap N(W) = \emptyset$, что вытекает из пункта 2.
\item $M$ ~--- независимое множество. Иначе рассмотрим индексы $i,j$, для которых $v_{i+1},v_{j+1}\in M$, $v_i,v_j\in N(W)$, $(v_{i+1},v_{j+1})\in E$. Пусть $x$ и $y$ ~--- те вершины в $W$ (возможно, совпадающие), которые соединены с $v_i$ и $v_j$ соответственно. Рассмотрим цикл $v_1v_2\dots v_i x\dots y v_j \dots v_{i+1}v_{j+1}\dots v_1$ ~--- по существу, мы удалили из $C$ два ребра $(v_i,v_{i+1})$ и $(v_j,v_{j+1})$, добавили три ребра $(v_i,x)$, $(v_j,y)$, $(v_{i+1},v_{j+1})$ и прошли путь от $v_j$ до $v_{i+1}$ в обратной последовательности. Значит, $C$ ~--- не максимальный цикл.
\item Пусть $w\in W$ ~--- произвольная вершина, тогда $M\cup w$ ~--- также независимое множество. Действительно, если $v\in M$, $(v,w)\in E$, то по определению $v\in N(W)$, поскольку по построению $M\subset C$. Но тогда $v\in M\cap N(W)$, что противоречит пункту 5.
\end{enumerate}
Пункт 7 означает, что $|M|<\alpha(G)$. хотя из пунктов 3 и 4 следует $|M|\geq k(G)$. Противоречие.

$\bullet$ \\ 

\section{Билет №6}
\subsection{Хроматическое число, число независимости, кликовое число и соотношения между ними}
$\triangleright$ \textbf{Def:} Хроматическое число графа $\chi(G)$ ~--- минимальное число цветов, в которое можно раскрасить вершины графа так, что все рёбра соединяют вершины разного цвета.
\\
$\triangleright$ \textbf{Suppose:} $\chi(G)\geq \omega(G),\:\: \chi(G)\geq \frac{n}{\alpha(G)}$.
\\
$\triangleright$ \textbf{Def:} $\alpha(G)$ (число независимости) ~--- максимальная мощность независимого множества.
\\
$\triangleright$ \textbf{Def:} $w(G)$ (кликовое число) ~--- максимальный размер клики в графе.
\\

\section{Билет №7}
\subsection{Системы общих представителей. Тривиальная верхняя и нижняя оценки}
$\triangleright$ \textbf{Def:} Пусть имеется $s$ $k$-элементных подмножеств $\{1,2,\dots,n\}$. Обозначим систему этих множеств $\mathcal{M}(n,k,s)$. Система общих представителей для $\mathcal{M}$ ~--- любое подмножество $\{1,2,\dots,n\}$, пересечение которого с каждым множеством системы непусто. Минимально возможный размер с.о.п. обозначим $\tau(\mathcal{M})$.
\\
$\triangleright$ \textbf{Suppose:} Для любой совокупности $\mathcal{M}$ выполнено $\tau(\mathcal{M})\leq min\{s,n-k+1\}$.
\\
$\circ$  Можно взять по элементу из каждого множества совокупности $\mathcal{M}$, а можно взять любое множество размера $n-k+1$ ~--- оно неизбежно пересекается с любым множеством размера $k$.
$\bullet$ \\ 
$\triangleright$ \textbf{Suppose:} Всегда имеется совокупность $\mathcal{M}$, для которой $\tau(\mathcal{M})\geq min\{[n/k],s\}$.
\\
$\circ$  Если $[n/k]\geq s$, то построим совокупность из непересекающихся множеств. Если $[n/k]<s$, то сделаем первые $[n/k]$ множеств не пересекающимися, а остальные возьмём произвольно.
$\bullet$ \\ 
\subsection{Верхняя оценка с помощью жадного алгоритма. Ее точность (б/д)}
$\triangleright$ \textbf{Th:}  Для любой совокупности: $\tau(\mathcal{M})\leq max\{\frac{n}{k},\frac{n}{k}\ln \frac{sk}{n}\}+\frac{n}{k}+1$.
\\
$\circ$  Если $s\leq \frac{n}{k}$, то (предложение 13) $\tau(\mathcal{M})\leq s\leq \frac{n}{k}$.\\
Если $\frac{n}{k}\,ln\:\frac{sk}{n}\geq n$, то $\tau(\mathcal{M})\leq n \leq \frac{n}{k}\ln \frac{sk}{n}$.\\
Иначе воспользуемся жадным алгоритмом: на каждом шаге берём элемент, лежащий в наибольшем числе множеств совокупности, и удаляем из совокупности эти множества. На каждом шаге мы удаляем $sk/n$ множеств. Сделаем $N = [\frac{n}{k}\ln \frac{sk}{n}]+1$ шагов, тогда в совокупности останется не более $\frac{n}{k}$ множеств. Значит, $\tau(\mathcal{M})\leq N+\frac{n}{k}$.
$\bullet$ \\ 

\section{Билет №8}
\subsection{Гиперграфы с запрещенными пересечениями ребер}
\subsection{Основы линейно-алгебраического метода}

\end{document}