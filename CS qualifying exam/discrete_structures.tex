\documentclass[a4paper]{article}
\usepackage{fontspec} 
\usepackage{polyglossia}
\setmainlanguage{russian} 
\setotherlanguage{english}
\newfontfamily{\cyrillicfont}{Times New Roman}

\usepackage{mathtools}
\usepackage{fullpage}
\usepackage[utf8x]{inputenc}
\usepackage{amsmath}
\usepackage[colorinlistoftodos]{todonotes}


\title{Дискретные структуры}
\author{MIPT DIHT}
\begin{document}
\maketitle

\section{Билет №1}
\subsection{Правила комбинаторики: правила сложения, умножения, принцип Дирихле. Формула включения и исключения}
\subsection{Размещения, сочетания и перестановки. Формула Стирлинга (б/д)}

\section{Билет №2}
\subsection{Размещения, сочетания, перестановки}
\subsection{Формулы для чисел размещения и сочетания с повторениями и без}
\subsection{Бином Ньютона, полиномиальная формула}
\subsection{Простейшие тождества. Оценки биномиальных коэффициентов}

\section{Билет №3}
\subsection{Формульные степенные ряды. Производящие функции и тождества}

\section{Билет №4}
\subsection{Линейные рекуррентные соотношения с постоянными коэффициентами}

\section{Билет №5}
\subsection{Граф, орграф, псевдограф, мультиграф, гиперграф}
\subsection{Маршруты в графах. Степени вершин}
\subsection{Изоморфизм и планарность графов}
\subsection{Эйлеровы и гамильтоновы циклыв графах}
\subsection{Критерий Эйлеровости. Достаточное условие гамильтоновости.}

\section{Билет №6}
\subsection{Хроматическое число, число независимости, кликовое число и соотношения между ними}

\section{Билет №7}
\subsection{Системы общих представителей. Тривиальная верхняя и нижняя оценки}
\subsection{Верхняя оценка с помощью жадного алгоритма. Ее точность (б/д)}

\section{Билет №8}
\subsection{Гиперграфы с запрещенными пересечениями ребер}
\subsection{Основы линейно-алгебраического метода}

\end{document}