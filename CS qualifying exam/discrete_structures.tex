\documentclass[a4paper]{article}

\usepackage[utf8]{inputenc}
\usepackage[english, russian]{babel}
\usepackage[fleqn]{amsmath}
\usepackage{amsfonts, amssymb, amsthm, mathtools}

\usepackage{mathtools}
\usepackage{fullpage}
\usepackage[colorinlistoftodos]{todonotes}


\title{Дискретные структуры}
\author{MIPT DIHT}

\theoremstyle{plain}
\newtheorem*{theorem-star}{Theorem}
\newtheorem{theorem}{Theorem}
\newtheorem*{lem-star}{Lemma}
\newtheorem{lem}{Lemma}
\newtheorem*{proposition-star}{Proposition}
\newtheorem{proposition}{Proposition}

\theoremstyle{remark}
\newtheorem*{remark}{Remark}

\theoremstyle{definition}
\newtheorem*{definition-star}{Definition}
\newtheorem{definition}{Definition}

\renewenvironment{proof}{{\bfseries Proof}}{$\bullet$}

\newcommand{\myequat}[1]{\begin{equation} #1 \nonumber \end{equation}}
\newcommand{\pars}[1]{\left( #1 \right)} % (smth_vertically_large)
\newcommand{\class}[1]{\left[ #1 \right]} % [smth_vertically_large]
\newcommand{\myN}{\mathbb{N}} % nice letters for common number sets
\newcommand{\myZ}{\mathbb{Z}}
\newcommand{\myR}{\mathbb{R}}
\newcommand{\myC}{\mathbb{C}}
\newcommand{\myQ}{\mathbb{Q}}
\newcommand{\myE}{\mathcal{E}} % basis
\newcommand{\myM}{\mathcal{M}} % some set
\newcommand{\myO}{(1+o(1))}

\begin{document}
\maketitle

\section{Билет №1}
\subsection{Правила комбинаторики: правила сложения, умножения, принцип Дирихле. Формула включения и исключения}
\subsection{Размещения, сочетания и перестановки. Формула Стирлинга (б/д)}

\section{Билет №2}
\subsection{Размещения, сочетания, перестановки}
\subsection{Формулы для чисел размещения и сочетания с повторениями и без}
\subsection{Бином Ньютона, полиномиальная формула}
\subsection{Простейшие тождества. Оценки биномиальных коэффициентов}

\section{Билет №3}
\subsection{Формульные степенные ряды. Производящие функции и тождества}

\section{Билет №4}
\subsection{Линейные рекуррентные соотношения с постоянными коэффициентами}

\section{Билет №5}
\subsection{Граф, орграф, псевдограф, мультиграф, гиперграф}
\begin{definition-star} Граф ~--- множество вершин и неориентированных рёбер.
\end{definition-star}
\begin{definition-star} Псевдограф ~--- граф с петлями.
\end{definition-star}
\begin{definition-star} Мультиграф ~--- граф с кратными рёбрами.
\end{definition-star}
\begin{definition-star} Дерево ~--- связный ациклический граф. Оно же ~--- граф, в котором любые две вершины соединены ровно одним путём; связный граф, в котором вершин на единицу больше, чем рёбер; ациклический граф,  в котором вершин на единицу больше, чем рёбер.
\end{definition-star}ф, в котором вершин на единицу больше, чем рёбер; ациклический граф,  в котором вершин на единицу больше, чем рёбер.
\begin{definition-star} Гиперграф ~--- множество вершин и рёбер,  каждое ребро ~--- произвольное подмножество вершин.
\end{definition-star}
\begin{definition-star} $k$-однородный гиперграф ~--- каждое ребро содержат ровно $k$ вершин.
\end{definition-star}
\begin{definition-star} $t$-пересекающийся гиперграф ~--- любые 2 ребра гиперграфа имеют хотя бы $t$ общих вершин.
\end{definition-star}

\subsection{Маршруты в графах. Степени вершин}

\subsection{Изоморфизм и планарность графов}
\subsection{Эйлеровы и гамильтоновы циклы в графах}
\begin{definition-star} Эйлеров цикл (цепь) ~--- цикл (цепь), содержащий все рёбра графа.
\end{definition-star}
\begin{definition-star} Эйлеров граф ~--- граф, обладающий эйлеровым циклом.
\end{definition-star}
\begin{definition-star} Гамильтонов цикл (цепь) ~--- цикл (цепь), содержащая все вершины по одному разу.
\end{definition-star}

\subsection{Критерий Эйлеровости. Достаточное условие гамильтоновости.}
\begin{theorem} Связный (мульти)граф является эйлеровым (1) тогда и только тогда, когда степень каждой вершины чётна (2), или тогда и только тогда, когда множество рёбер графа можно покрыть без пересечений простыми циклами (3).
\end{theorem}
\begin{proof} $(1)\Rightarrow(2)$: если степень какой-либо вершины нечётна, то мы, двигаясь в порядке рёбер эйлерова цикла, не сможем в какой-то момент войти в эту вершину по одному ребру и выйти по другому ребру, поскольку её степень нечётна. Это означает, что наш обход не является циклом. Противоречие.\\
$(3)\Rightarrow(1)$: объединение всех этих простых циклов является эйлеровым циклом.\\
Что мы подразумеваем под словом "объединение"? Давайте рассмотрим это как последовательный процесс: на нулевом шаге мы рассмотрим любой простой цикл, и будем добавлять к нему простые циклы из числа ещё не задействованных по одному. Таким образом, на каждом шаге мы имеем некоторый цикл и множество (возможно, пустое) тех простых циклов, которые мы ещё не рассмотрели.\\
Пусть это множество непусто. Тогда, так как граф связен, в построенном на данный момент цикле обязательно найдётся вершина, лежащая в одном из незадействованных простых циклов. Обозначим эту вершину $v$, уже построенный нами цикл ~--- $a_1\dots a_i v a_{i+1}\dots a_1$, незадействованный простой цикл ~--- $vb_1b_2\dots b_kv$. Тогда новый цикл мы определим как $a_1\dots a_i v b_1\dots b_k v a_{i+1}\dots a_1$, и мы уменьшили на 1 количество не рассмотренных простых циклов.\\
Пусть это множество оставшихся циклов пусто. По предположению, тогда пусто и множество ребёр, которые лежат вне построенного нами цикла ~--- следовательно, этот цикл эйлеров.\\
$(2)\Rightarrow(3)$: индукция по количеству рёбер.\\
База индукции: если рёбер $0$, то множество рёбер тривиально состоит из нуля простых непересекающихся друг с другом циклов.\\
Переход: выберем произвольную вершину ненулевой степени и пойдём в обход по графу, не проходя дважды одного и того же ребра, пока не вернёмся в какую-либо вершину ~--- таким образом, мы выделили простой цикл. Рёбра этого цикла мы удалим из графа, и чётность степеней всех вершин сохранится, а число рёбер уменьшится.
\end{proof}
\begin{theorem} Слабо связный орграф является эйлеровым тогда и только тогда, когда входящие степени (каждой вершины) равны исходящим.
\end{theorem}
\begin{proof} Аналогично предыдущей теореме.
\end{proof}
\begin{theorem}[Критерий Дирака] Если в графе на $n$ вершинах степень каждой вершины не менее $\lceil \frac{n}{2} \rceil$, то граф содержит гамильтонов цикл.
\end{theorem}
\begin{proof} Начнём со вспомогательного утверждения:
\begin{lem-star} Пусть в графе максимальный простой путь состоит из $m$ вершин, и суммарная степень двух концов этого пути не меньше $m$. Тогда в графе существует простой цикл длины $m$.
\end{lem-star}
\begin{proof} Обозначим вершины этого пути $a_1,a_2,a_3,\dots,a_m$. Так как путь максимален, то рёбра вида $(a_1,\:v)$ и $(a_m,\:v)$, где $v \notin \{a_i\}_{i=1}^m$, в графе отсутствуют. 
Если вершины $a_1$ и $a_m$ соединены ребром, то искомый цикл найден.\\
Если одновременно есть рёбра $(a_{i+1},\:a_1)$ и $(a_i,a_m)$ (для произвольного $ i \in \overline{2,\, m-2} $ ), то искомый цикл выглядит так: $a_1a_2\dots a_ia_ma_{m-1}\dots a_{i+1} a_1$.
Предположим, что цикла всё же нет. Тогда в силу предыдущего утверждения каждое ребро, проведённое из $a_m$, "запрещает"\ одно ребро из $a_1$, и наоборот (кроме заведомо существующих рёбер $(a_1,\,a_2)$ и $(a_m,\,a_{m-1})$, которые мы сейчас не учитываем). При этом из вершины $a_1$ могут быть рёбра к вершинам $a_3,a_4,\dots,a_{m-1}$ ~--- всего $m-3$ возможности, столько же для $a_m$. Однако эти возможности взаимоисключающие, а нам необходимо (согласно посылке леммы) провести из $a_1$ и $a_m$ суммарно $m-2$ ребра. Противоречие.
\end{proof}
Заметим, что граф связен, поскольку суммарная степень любых двух вершин не менее $n$ ~--- это означает, что они либо соединены ребром, либо (по принципу Дирихле) имеют общего соседа.\\
Рассмотрим в нашем графе максимальный простой путь. Согласно лемме, существует простой цикл, проходящий по всем вершинам этого пути (и только по ним). Обозначим его вершины в порядке следования цикла $a_1,a_2,a_3,\dots ,a_m$.\\
Если $m<n$, то рассмотрим любую вершину $v$, не лежащую в цикле. Так как граф связен, для некоторого $i$ существует ребро $(a_i,\:v)$. Тогда путь $va_ia_{i+1}\dots a_ma_1a_2\dots a_{i-1}$ содержит на одну вершину больше, чем рассмотренный нами максимальный. Противоречие.\\
Если же $m=n$, то цикл $a_1a_2a_3\dots a_{m-1}a_ma_1$ ~--- гамильтонов.
\end{proof}


\begin{theorem} Пусть в графе $G$ хотя бы 3 вершины и $k(G)\geq\alpha(G)$. Тогда $G$ содержит гамильтонов цикл.
\end{theorem}
\begin{proof} Если в $G$ нет циклов, то $k(G)\geq\alpha(G)\geq 1 \Rightarrow G$ связен $\Rightarrow k=1, \alpha\geq 2$. Противоречие. Иначе рассмотрим максимальный простой цикл $C = \{v_1,v_2,\dots,v_m\}$ и предположим, что он не гамильтонов, то есть $G\setminus C$ непусто. Пусть $W$ ~--- любая связная компонента $G\setminus C$, $N(W) = \{x\notin W\: |\:\exists y \in W: \: (x,y)\in E(G)\}$. Имеют место следующие утверждения:
\begin{enumerate}
\item $N(W)\subset C$ (сосед компоненты связности, не лежащий в $C$, должен лежать в самом $W$). 
\item Никакие две соседние вершины цикла не лежат в $N(W)$ одновременно. В противном случае для некоторого $i$ в графе есть рёбра $(v_i,x)$, $(y,v_{i+1})$, где $x,y \in W$, а также путь (возможно, нулевой длины) между $x$ и $y$, так как $W$ связно. Тогда, удаляя ребро $(v_i,v_{i+1})$ из $C$ и заменяя его на путь $v_ix\dots yv_{i+1}$, мы получаем цикл большей длины, чем $C$ ~--- значит, $C$ не был максимален.
\item $|N(W)|\geq k(G)$. Действительно, если мы удалим множество $N(W)$ из графа, то $C\setminus N(W)$ и $W$ окажутся в различных компонентах связности.
\item Определим $M = \{v_{i+1}\:|\: v_i\in N(W)\}$ и заметим, что $|M|=|N(W)|$ (по построению).
\item $M \cap N(W) = \emptyset$, что вытекает из пункта 2.
\item $M$ ~--- независимое множество. Иначе рассмотрим индексы $i,j$, для которых $v_{i+1},v_{j+1}\in M$, $v_i,v_j\in N(W)$, $(v_{i+1},v_{j+1})\in E$. Пусть $x$ и $y$ ~--- те вершины в $W$ (возможно, совпадающие), которые соединены с $v_i$ и $v_j$ соответственно. Рассмотрим цикл $v_1v_2\dots v_i x\dots y v_j \dots v_{i+1}v_{j+1}\dots v_1$ ~--- по существу, мы удалили из $C$ два ребра $(v_i,v_{i+1})$ и $(v_j,v_{j+1})$, добавили три ребра $(v_i,x)$, $(v_j,y)$, $(v_{i+1},v_{j+1})$ и прошли путь от $v_j$ до $v_{i+1}$ в обратной последовательности. Значит, $C$ ~--- не максимальный цикл.
\item Пусть $w\in W$ ~--- произвольная вершина, тогда $M\cup w$ ~--- также независимое множество. Действительно, если $v\in M$, $(v,w)\in E$, то по определению $v\in N(W)$, поскольку по построению $M\subset C$. Но тогда $v\in M\cap N(W)$, что противоречит пункту 5.
\end{enumerate}
Пункт 7 означает, что $|M|<\alpha(G)$. хотя из пунктов 3 и 4 следует $|M|\geq k(G)$. Противоречие.

\end{proof}

$\bullet$ \\ 

\section{Билет №6}
\subsection{Хроматическое число, число независимости, кликовое число и соотношения между ними}

\begin{definition-star} $k(G)$ (вершинная связность) ~--- минимальное к-во вершин, от удаления которых граф $G$ теряет связность.
\end{definition-star}
\begin{definition-star} $\alpha(G)$ (число независимости) ~--- максимальная мощность независимого множества.
\end{definition-star}
\begin{definition-star} $w(G)$ (кликовое число) ~--- максимальный размер клики в графе.
\end{definition-star}

\begin{definition-star} Хроматическое число графа $\chi(G)$ ~--- минимальное число цветов, в которое можно раскрасить вершины графа так, что все рёбра соединяют вершины разного цвета.
\end{definition-star}
\begin{proposition-star} $\chi(G)\geq \omega(G),\:\: \chi(G)\geq \frac{n}{\alpha(G)}$.
\end{proposition-star}
\begin{theorem} В последовательности случайных графов при $p(n)=1/2$ АПН $\alpha(G)\leq 2\: \log_2 n$.
\end{theorem}
\begin{proof} Пусть $X_k(G)$ ~--- число независимых множеств на $k$ вершинах, $k = [2\log_2 n]$. 
\myequat{MX_k=C_n^k\,2^{-C_k^2}\leq \frac{n^k}{k!}2^{-\frac{k^2}{2}+\frac{k}{2}}\leq \frac{2^{2 \log_2^2 n}}{k!}2^{-\frac{(2\log_2 n - 1)^2}{2}+\log_2 n}\leq\frac{1}{k!}2^{3\log_2 n}=\frac{n^3}{k!}}
\myequat{k!>(k/e)^k=\pars{\frac{2\log_2 n}{e}}^{2\log_2 n}>8^{2\log_2 n}=n^6}
\myequat{MX_k\to 0}
\end{proof}
Таким образом, вторая оценка для $\chi(G)$, как правило, лучше.
\begin{theorem}[Боллобаш, б/д] При $p(n)=1/2$ существует функция $\phi(n)=o\pars{\frac{n}{\ln n}}$ такая, что АПН $\chi(G)= \frac{n}{2\log_2 n}+\phi(n)$.
\end{theorem}
\begin{definition-star} Жадный алгоритм нахождения хроматического числа: раскрасим последовательно вершины в минимально возможный на данный момент цвет. Обозначим полученный результат $\chi'(G)$.
\end{definition-star}
\begin{definition-star} Жадный алгоритм нахождения числа независимости (кликового числа ~--- аналогично): в найденной ранее раскраске графа рассмотрим наибольшую компоненту. Обозначим полученный результат $\alpha'(G)$.
\end{definition-star}
\begin{theorem} При $p(n)=1/2$ АПН $\alpha'(G)\geq (1-\varepsilon)\log_2 n$.
\end{theorem}
\begin{proof} Пусть событие $A$ означает обратное, т.е. $\alpha'(G)<(1-\varepsilon)\log_2 n$. Отсюда следует, что алгоритм отыскал хотя бы $x=\class{\frac{n}{2(1-\varepsilon)\log_2 n}}$ различных цветов. Так как алгоритм жадный, то каждая вершина из $V(G)/\bigcup_{i=1}^xC_i$ соединена с каждым из первых $x$ цветов.\\
Пусть $a_1,a_2,\dots,a_x$ ~--- размеры первых $x$ цветов, $a_i<(1-\varepsilon)\log_2 n$. В дальнейших выкладках внешнее суммирование ведётся по всем возможным числам $a_i$ от $1$ до $(1-\varepsilon)\log_2 n$ и непересекающимся подмножествам $C_1,C_2,\dots,C_x\subset V$ таким, что $|C_i|=a_i$.
\myequat{P(A)\leq \sum P(\forall\,x\in V/\bigcup_{i=1}^xC_i\;\forall\,i\,\exists\, y\in C_i\: | \: (x,y)\in E)\leq}
\myequat{\leq\sum \class{\prod_{i=1}^x(1-2^{-a_i})}^{n-\sum_{j=1}^xa_j}\leq\sum\class{1-2^{(\varepsilon-1)\log_2 n}}^{x(n-\sum_{j=1}^xa_j)}\leq}
\myequat{\leq\sum [1-n^{\varepsilon-1}]^{\frac{nx}{2}}\leq\sum e^{-\frac{nx}{2n^{1-\varepsilon}}}}
То, что осталось под суммой, оценим окончательно как $e^{\frac{n^{\varepsilon}x}{2}}<e^{-n^{1+\delta}},\:\:\delta>0$.\\
Вернёмся к количеству слагаемых: их не больше, чем
\myequat{(\log_2 n)^x (C_n^{\log_2 n})^x<(\log_2 n)^x n^{x\log_2 n}}
\myequat{(\log_2 n)^x<n^x<n^{x\log_2 n}}
\myequat{n^{2x\log_2 n}=e^{2x\log_2 n\ln n}=e^{C\myO n\log_2 n}}
Итого,
\myequat{P(A)<n^{2x\log_2 n}e^{-n^{1+\delta }}<e^{-n^{1+\delta}+C\myO n\log_2 n}\to 0,}
что и требовалось.
\end{proof}

\section{Билет №7}
\subsection{Системы общих представителей. Тривиальная верхняя и нижняя оценки}
\begin{definition-star} Пусть имеется $s$ $k$-элементных подмножеств $\{1,2,\dots,n\}$. Обозначим систему этих множеств $\myM(n,k,s)$. Система общих представителей для $\myM$ ~--- любое подмножество $\{1,2,\dots,n\}$, пересечение которого с каждым множеством системы непусто. Минимально возможный размер с.о.п. обозначим $\tau(\myM)$.
\end{definition-star}
\begin{proposition-star} Для любой совокупности $\myM$ выполнено $\tau(\myM)\leq min\{s,n-k+1\}$.
\end{proposition-star}
\begin{proof} Можно взять по элементу из каждого множества совокупности $\myM$, а можно взять любое множество размера $n-k+1$ ~--- оно неизбежно пересекается с любым множеством размера $k$.
\end{proof}
\begin{proposition-star} Всегда имеется совокупность $\myM$, для которой $\tau(\myM)\geq min\{[n/k],s\}$.
\end{proposition-star}
\begin{proof} Если $[n/k]\geq s$, то построим совокупность из непересекающихся множеств. Если $[n/k]<s$, то сделаем первые $[n/k]$ множеств не пересекающимися, а остальные возьмём произвольно.
\end{proof}
\subsection{Верхняя оценка с помощью жадного алгоритма. Ее точность (б/д)}
\begin{theorem} Для любой совокупности: $\tau(\myM)\leq max\{\frac{n}{k},\frac{n}{k}\ln \frac{sk}{n}\}+\frac{n}{k}+1$.
\end{theorem}
\begin{proof} Если $s\leq \frac{n}{k}$, то (предложение 13) $\tau(\myM)\leq s\leq \frac{n}{k}$.\\
Если $\frac{n}{k}\,ln\:\frac{sk}{n}\geq n$, то $\tau(\myM)\leq n \leq \frac{n}{k}\ln \frac{sk}{n}$.\\
Иначе воспользуемся жадным алгоритмом: на каждом шаге берём элемент, лежащий в наибольшем числе множеств совокупности, и удаляем из совокупности эти множества. На каждом шаге мы удаляем $sk/n$ множеств. Сделаем $N = \class{\frac{n}{k}\ln \frac{sk}{n}}+1$ шагов, тогда в совокупности останется не более $\frac{n}{k}$ множеств. Значит, $\tau(\myM)\leq N+\frac{n}{k}$.
\end{proof}

\section{Билет №8}
\subsection{Гиперграфы с запрещенными пересечениями ребер}
\subsection{Основы линейно-алгебраического метода}

\end{document}