\documentclass[a4paper]{article}

\usepackage[utf8]{inputenc}
\usepackage[english, russian]{babel}
\usepackage[fleqn]{amsmath}
\usepackage{amsfonts, amssymb, amsthm, mathtools}

\usepackage{mathtools}
\usepackage{fullpage}
\usepackage[colorinlistoftodos]{todonotes}

\usepackage{cmap}
\usepackage[all]{xy}
\usepackage{mdwlist}
\usepackage{tikz}
\usepackage{enumitem}
\usepackage{epstopdf}



\title{Теория вероятностей и математическая статистика}
\author{MIPT DIHT}


\theoremstyle{plain}
\newtheorem*{theorem-star}{Theorem}
\newtheorem{theorem}{Theorem}
\newtheorem*{lem-star}{Lemma}
\newtheorem{lem}{Lemma}
\newtheorem*{proposition-star}{Proposition}
\newtheorem{proposition}{Proposition}
\newtheorem{statement}{Statement}
\newtheorem*{statement-star}{Statement}
\newtheorem{corollary}{Corollary}
\newtheorem*{corollary-star}{Corollary}

\theoremstyle{remark}
\newtheorem*{remark}{Remark}

\theoremstyle{definition}
\newtheorem*{definition-star}{Definition}
\newtheorem{definition}{Definition}
\newtheorem{example}{Example}
\newtheorem*{example-star}{Example}

\renewenvironment{proof}{{\bfseries Proof}}{$\bullet$}

\newcommand{\myequat}[1]{\begin{equation} #1 \nonumber \end{equation}}
\newcommand{\pars}[1]{\left( #1 \right)} 
\newcommand{\class}[1]{\left[ #1 \right]} 
\newcommand{\dd}{\; \mathrm{d}}
\newcommand{\setR}{\mathbb{R}}
\newcommand{\setRn}{\mathbb{R}^n}
\newcommand{\setRinf}{\mathbb{R}^{\infty}}
\newcommand{\setC}{\mathbb{C}}
\newcommand{\setN}{\mathbb{N}}
\newcommand{\setZ}{\mathbb{Z}}
\newcommand{\setQ}{\mathbb{Q}}
\newcommand{\setM}{\mathcal{M}}
\newcommand{\setL}{\mathcal{L}}
\newcommand{\setA}{\mathcal{A}}
\newcommand{\setF}{\mathcal{F}}

\newcommand{\combru}[2]{C_{#1}^{#2}} % russian C_n^k
\newcommand{\comb}[2]{\combru{#1}{#2}}

\newcommand*\circled[1]{\tikz[baseline=(char.base)]{\node[shape=circle,draw,inner sep=2pt] (char) {#1};}}

\newcommand{\walls}[1]{\left | #1 \right |} % |smth_vertically_large|
\newcommand{\braces}[1]{\left\{ #1 \right\}} % {smth_vertically_large}

\newcommand{\condset}[2]{\braces{\, #1 \mid #2 \,}} % definition of set with condition

\newcommand{\expl}[1]{\walls{\text{#1}}} % explanation inside formula

\newcommand{\toup}[1]{\xrightarrow{#1}}
\newcommand{\toae}{\toup{\text{\,п.н.}}} % almost everywhere convergence designation
\newcommand{\todown}[1]{\xrightarrow[#1]{}}

\newcommand{\equp}[1]{\stackrel{#1}{=}}

\newcommand{\conj}[1]{\overline{#1}} % complex conjugation
\newcommand{\comp}[1]{\overline{#1}} % set complement

\DeclareMathOperator{\cov}{cov}

\renewcommand{\emptyset}{\varnothing}
\renewcommand{\epsilon}{\varepsilon}
\renewcommand{\phi}{\varphi}
\renewcommand{\leq}{\leqslant}
\renewcommand{\geq}{\geqslant}
\renewcommand{\Im}{\mathop{\mathrm{Im}}\nolimits}
\renewcommand{\Re}{\mathop{\mathrm{Re}}\nolimits}

\DeclareMathOperator*{\argmin}{arg\,min}

\newcommand{\bigtitle}[1]{\title{\textbf{\underline{#1}}}}
\newcommand{\boldtitle}[1]{\title{\textbf{#1}}}

\newcommand*{\hm}[1]{#1\nobreak\discretionary{}%
{\hbox{$\mathsurround=0pt #1$}}{}} % a\hm=b makes "=" carriable to the next line with duplication of the sign


\begin{document}
\maketitle


\section{Билет №1}
\subsection{Вероятностное пространство, аксиомы Колмогорова, свойства вероятностной меры}
В основе теории вероятностей лежит понятие вероятностного пространства $(\Omega, \setF, P)$ 
(т.н ``тройки Колмогорова'')

\begin{enumerate}[label=\protect\circled{\arabic*},series=kolm_triple]

\item
	$\Omega$ --- \emph{пространство элементарных событий}.\\
	$\omega \in \Omega$ --- называется \emph{элементарным событием}.\\
	В результате случайного эксперимента получаем один и ровно один элемент $\Omega$.

\item
	$\setF$ --- $\sigma$-алгебра подмножеств на $\Omega$.\\
	Элементы $\setF$ называются \emph{событиями}.\\
	$\forall A \in \setF \implies A \subset \Omega$.

\end{enumerate}

\begin{definition}
	Система подмножеств $\setF$ множества $\Omega$ называется \emph{алгеброй}, если:

	\begin{enumerate}
		\item $\Omega \in \setF$
		\item $\forall A,B \in \setF \implies A \cap B \in \setF$
		\item $\forall A,B \in \setF \implies A \bigtriangleup B \in \setF$\\
	\end{enumerate}

\end{definition}


\begin{definition}
 	$\comp{A} = \Omega \setminus A$, называется дополнительным событием к событию $A$.\\
\end{definition}

\begin{example}~
	\begin{enumerate}
		\item $\setF_* = \braces{\emptyset, \Omega}$ --- тривиальная алгебра
		\item $\setF^* = 2^\Omega$ (все подмножества $\Omega$) --- дискретная алгебра
		\item $\setF = \braces{\emptyset, A, \comp{A}, \Omega}$ --- алгебра ``порожденная'' $A$
		\item Конечные объединения подмножеств вида 
					$[a, b), (-\infty; c), [d, +\infty)$ образуют алгебру.\\
	\end{enumerate}
\end{example}

\begin{definition}
	Система подмножеств $\setF$ множества $\Omega$ называется $\sigma$-алгеброй, если:
	\begin{enumerate}
		\item $\setF$ --- алгебра
		\item $\forall \braces{A_n, n \in \setN}, A_n \in \setF \; \forall n 
					\implies \bigcup\limits_{n = 1}^{+\infty} A_n \in \setF$\\
	\end{enumerate}
\end{definition}


\begin{example}~
	\begin{enumerate}
		\item $\setF_* $ --- тривиальная $\sigma$-алгебра
		\item $\setF^* $ --- дискретная $\sigma$-алгебра
		\item $\forall$ конечная алгебра является $\sigma$-алгеброй.
		\item $[a, b), (-\infty; c), [d, +\infty)$ --- не $\sigma$-алгебра.\\
	\end{enumerate}
\end{example}

\begin{enumerate}[resume*=kolm_triple]
	\item
		$P$ - \emph{вероятностная мера} на $(\Omega, \setF)$
\end{enumerate}

\begin{definition}
	Пара $(\Omega, \setF)$ множества $\Omega$ с заданной на нем $\sigma$-алгеброй $\setF$ называется \emph{измеримым пространством}. 
\end{definition}

\begin{definition}
	Отображение $P\colon \setF \rightarrow [0;1]$ \\
	называется вероятностной мерой(или вероятностью) на $(\Omega, \setF)$, если:

	\begin{enumerate}
		\item $P(\Omega) = 1$
		\item Для $\forall$ последовательности $\braces{A_n, n \in \setN}, A_n \in \setF \; \forall n
		\text{ такой, что }  \forall i \neq j: \; A_i \cap A_j = \emptyset$ \\ выполнено свойство счетной аддитивности:
		\begin{equation*}
			P\pars{\bigsqcup\limits_{n=1}^{\infty} A_n} = \sum\limits_{n=1}^{\infty} P(A_n)
		\end{equation*}
	\end{enumerate}
\end{definition}

\newpage

\begin{statement}~
	\begin{enumerate}
		\item $P(\emptyset) = 0$
		\item Если $A \cap B = \emptyset, \text{ то } P(A \cup B) = P(A) + P(B) $ (свойство конечной аддитивности)
		\item $P(\comp{A}) = 1 - P(A)$
		\item $P(A \cup B) = P(A) + P(B) - P(A \cap B)$
		\item $\forall A_1, \ldots, A_m \in \setF \\ 
				P\pars{\bigcup\limits_{n = 1}^{m} A_n} \leq \sum\limits_{n = 1}^{m} P(A_n) $
		\item Если $A \subset B, $ то $P(A) \leq P(B)$ \\
	\end{enumerate}

	\begin{proof}~
		\begin{enumerate}
			\item 
				$\displaystyle \forall n \; A_n = \emptyset \implies
				P\pars{\bigsqcup_{n = 1}^{\infty} A_n} = \sum_{n = 1}^{\infty} P(A_n) = 
				\sum_{n = 1}^{\infty} P(\emptyset) < +\infty
				\implies P(\emptyset) = 0$

			\item 
				$\displaystyle A_1 = A,\; A_2 = B,\; A_3 = A_4 = \ldots = A_n = \ldots = \emptyset \\ 
				P\pars{\bigsqcup_{n = 1}^{\infty} A_n} = P(A \cup B) = \sum_{n = 1}^{\infty} P(A_n) = P(A) + P(B)$

			\item $\Omega = A \sqcup \comp{A} \implies \expl{по 2} \implies 
						1 = P(A) + P(\comp{A})$

			\item 
				$A \cup B = A \sqcup (B \setminus (A \cap B)) \\
					\implies P(A \cup B) = P(A) + P(B \setminus (A \cap B))\\\\
				B = (A \cap B) \sqcup (B \setminus (A \cap B)) \\
					\implies P(B) = P(A \cap B)  + P(B \setminus (A \cap B))$

				Осталось вычесть одно равенство из другого.

			\item 
				Если $m = 2$ --- то это пункт 4). \\
				По индукции\\
				$\displaystyle P\pars{\bigcup_{n = 1}^m A_n} 
				\leq P(A_m) + P\pars{\bigcup_{n = 1}^{m - 1} A_n} 
				\leq \text{|индукция|} \leq P(A_m) + \sum_{n = 1}^{m - 1} P(A_n) = \sum_{n = 1}^{m} P(A_n)$

			\item Следует из 4).
		\end{enumerate}			
	\end{proof}
\end{statement}

\begin{definition}
	Будем обозначать $A_n \downarrow A$ при $n \to +\infty $, если для последовательности событий $\braces{A_n, n \in \setN}$ выполнены свойства: 

	\begin{enumerate}
		\item $A_n \supset A_{n+1} \supset \ldots$
		\item $A = \bigcap\limits_{n}^{\infty} A_n$
	\end{enumerate}
\end{definition}

\begin{theorem}[О непрерывности в нуле вероятностной меры]~

	Пусть $(\Omega, \setF)$ - измеримое пространство, а $P\colon \setF \rightarrow [0, 1]$ удовлетворяет двум свойствам:
	\begin{enumerate}
		\item $P(\Omega) = 1$
		\item $P$ - конечно-аддитивна. 
	\end{enumerate}
	Тогда $P$ - вероятностная мера $\iff P$ - непрерывна в нуле (т.е если $A_n  \downarrow \emptyset$, то $P(A_n) \to 0$).

\end{theorem}

\begin{proof}~

	$(\implies)$ Пусть $P$ - вероятностная мера, а $A_n \downarrow \emptyset.$

	Рассмотрим $B_m = A_m \setminus A_{m+1}.$ Тогда в силу $\bigcap\limits_n A_n = \emptyset
	\implies \bigsqcup\limits_{m = n}^{\infty} B_m = A_n$

	Тогда в силу счетной аддитивности $P(A_n) = \sum\limits_{m = n}^{\infty} P(B_m)$

	Но ряд $P(A_1) = \sum\limits_{m=1}^{\infty} P(B_m) $ сходится
	$\implies \sum\limits_{m=n}^{\infty} P(B_m)$ есть остаток сходящего ряда 
	$\implies P(A_n)~\to~0$\\
	

	$(\impliedby)$ Пусть $P$ непрерывна в нуле. 
	Покажем её счетную аддитивность:

	Пусть ${A_n, n \in \setN} $ т.ч $A_n \in F\; \forall n$ и 
	$A_i \cap A_j = \emptyset$ при $i \neq j$\\
	Рассмотрим $B_m = \bigsqcup\limits_{n=m}^{+\infty} A_n.$ 
	Тогда $B_m \supset B_{m+1} \supset \ldots$

	Покажем, что $\bigcap\limits_m B_m = \emptyset $. \\
	Пусть $\omega \in \bigcap\limits_m B_m 
	\implies \omega \in B_1 \implies \exists k: \omega \in A_k 
	\implies \omega \not\in B_{k+1}$. Противоречие.

	Следовательно, $\bigcap\limits_m B_m = \emptyset$ и в силу непрерывности в нуле 
	$P(B_m) \to 0$.

	Далее $\displaystyle P\pars{\bigsqcup_{n=1}^{\infty} A_n} = 
	P\pars{\bigsqcup_{n=1}^{m} A_m \sqcup B_{m+1}} 
	= \expl{конечная аддитивность} = \\
	= \sum_{n=1}^{m} P(A_n) + P(B_{m + 1}) \to \sum_{n = 1}^{\infty} P(A_n),\; m \to \infty\\
	\implies P\pars{\bigsqcup_n A_n} = \sum_n P(A_n)$
\end{proof}

\begin{corollary}[непрерывность вероятностной меры]~
	\begin{enumerate}
		\item Если $A_n \downarrow A, \text{ то } P(A_n) \to P(A)$
		\item Если $A_n \uparrow A$ (т.е $A_n \subset A_{n + 1} \subset \ldots $,
		и $A = \bigcup\limits_n A_n$, то $P(A_n) \to P(A)$	
	\end{enumerate}
\end{corollary}

\begin{proof}~
	\begin{enumerate}
		\item Надо рассмотреть $B_n = A_n \setminus A$
		\item Надо рассмотреть $B_n = \comp{A_n}$
	\end{enumerate}
\end{proof}

\subsection{Условные вероятности. Формула полной вероятности. Формула Байеса}

Пусть $(\Omega, \setF, P)$ -- вероятностное пространство.

\begin{definition}
	Для $\forall A \in \setF$, т.ч. $P(A) > 0$
	\emph{условной вероятностью} события $B \in \setF$ при условии $A$ называют
	\begin{equation*}
		P(B \mid A) = \frac{P(A \cap B)}{P(A)}
	\end{equation*}
	если же $P(A) = 0,$ то $P(B \mid A) = 0,\; \forall B \in \setF$
\end{definition}

\begin{definition}
	Систему событий $\braces{B_n}_{n = 1}^{\infty}$ называют разбиением множества $\Omega$, если:
	\begin{enumerate}
		\item $\forall i \neq j : B_i \cap B_j = \emptyset$
		\item $\bigsqcup\limits_{n = 1}^{\infty} B_n = \Omega$ 
	\end{enumerate}
	
	В этом случае также говорят, что $\braces{B_n}_{n = 1}^{\infty}$ образует полную группу несовместных событий.
\end{definition}

\begin{lem}[формула полной вероятности]~

	Пусть $\braces{B_n}_{n = 1}^{\infty}$ - разбиение $\Omega$. Тогда для $\forall A \in \setF$:
	\begin{equation*}
		P(A) = \sum_{n = 1}^{\infty} P(A \mid B_n) P(B_n)
	\end{equation*}

	\begin{proof}
		Рассмотрим событие $A$
		\begin{align*}
			&P(A) = P(A \cap \Omega) = P\pars{A \cap \bigsqcup_{n = 1}^{\infty} B_n} =
			P\pars{\bigsqcup_{n = 1}^{\infty} A \cap B_n} = \\
			&= \expl{счетная аддитивность} = \sum_{n = 1}^{\infty} P(A \cap B_n) = 
			\sum_{n = 1}^{\infty} P(A \mid B_n) P(B_n)
		\end{align*}
	\end{proof}
\end{lem}

\begin{lem}[формула Байеса]~

	Пусть $\braces{B_n}_{n = 1}^{\infty}$ -- разбиение $\Omega$, а $A \in \setF : P(A) > 0.$ Тогда $\forall n$
	\begin{equation*}
		P(B_n \mid A) = \frac{P(A \mid B_n) P(B_n)}{\sum_{k = 1}^{\infty} P(A \mid B_k) P(B_k)}
	\end{equation*}
\end{lem}

\begin{definition}
	$P(B_n)$ называется \emph{априорной вероятностью}.\\
	$P(B_n \mid A)$ называется \emph{апостериорной вероятностью} (относительная вероятность при условии известного результата эксперимента)
\end{definition}

\newpage
\section{Билет №2}
\subsection{Случайные величины и векторы. Их характеристики: распределение вероятностей, функция распределения, ее свойства, $\sigma$=алгера, порожденная с. в.}
\begin{definition}
	Пусть ($\Omega, \setF$) и $(E, \mathcal{E})$ -- два измеримых пространства. 
  Отображение $X\colon \Omega \to E$ называется случайным элементом,
   если оно является $\setF$ - измеримым. (или $\setF \setminus \mathcal{E}$ - измеримым) 
   т.е $\forall B \in \mathcal{E}$
  \begin{align*}
    \{ x \in B \} = X^{-1}(B) = \condset{\omega}{X(\omega) \in B} \in \setF.
  \end{align*}
\end{definition}

\begin{definition}~

  Если $(E, \mathcal{E})$ = $(\setR, B(\setR))$, то случайный элемент $X$ 
  называется \emph{случайной величиной}.\\

  Если $(E, \mathcal{E})$ = $(\setRn, B(\setRn))$, то $X$ называется \emph{случайным вектором}.\\
\end{definition}

\begin{lem}[свойства функции распределения]~

	Пусть $F(x)$ -- функция распределения вероятностной меры $P$. Тогда
	\begin{enumerate}
		\item 
			$F(x)$ - неубывающая

		\item 
			$\lim\limits_{x \rightarrow -\infty} F(x) = 0, \, 
			\lim\limits_{x \rightarrow +\infty} F(x) = 1$

		\item 
			$F(x)$ непрерывная справа.
	\end{enumerate}
\end{lem}

\begin{proof}~

	\begin{enumerate}
		\item 
			Пусть $y \geq x$. Тогда
			\begin{align*}
				F(y) - F(x) = P((-\infty; y]) - P((-\infty; x]) = P((x, y]) \geq 0
			\end{align*}

		\item 
			Пусть $x_n \rightarrow -\infty$ при $n \rightarrow \infty$. 
			Тогда $(-\infty; x_n] \downarrow \emptyset\; \implies$ по непрерывности вероятностной меры.
			\begin{align*}
				F(x_n) = P((-\infty, x_n]) \todown{n \to \infty} P(\emptyset) = 0
			\end{align*}

			Аналогично, если $x_n \to +\infty$, то $(-\infty; x_n] \uparrow \setR\; \implies$ 
			в силу непрерывности вероятностой меры.
			\begin{align*}
				F(x_n) = P((-\infty; x_n])) \todown{n \to \infty} P(\setR) = 1
			\end{align*}

		\item 
			Пусть убывающая $x_n \to x + 0$ Тогда $(-\infty, x_n]) \downarrow (-\infty; x]\; \implies$ 
			в силу непрерывности вероятностой меры.
			\begin{align*}
				F(x_n) = P((-\infty; x_n]) \todown{n \to \infty} P((-\infty; x]) = F(x)
			\end{align*}
	\end{enumerate}
\end{proof}

\bigtitle{Распределение случайной величины вектора.}

\begin{definition}
  Пусть $(\Omega, \setF, P)$ -- вероятностное пространство, $\xi$ - случайная величина на нем. Тогда распределением $\xi$ называется вероятностная мера $P_\xi$ на $(\setR, B(\setR))$, заданная по правилу.
  \begin{align*}
    P_\xi (B) = P(\xi \in B),\; B \subset B(\setR).
  \end{align*}
\end{definition}

\begin{definition}
  Пусть $\xi$ - случайный вектор размерности $n$ на $(\Omega, \setF, P)$. 
  Тогда его распределением $P_\xi$ называется вероятностая мера на ($\setRn, B(\setRn)$), заданная по правилу
    \begin{align*}
      P_\xi (B) = P(\xi \in B),\; B \in B(\setRn)
    \end{align*}
\end{definition}

\bigtitle{Функция распределения}

\begin{definition}
  Пусть $(\Omega, \setF, P)$ -- вероятностное пространство.
  $\xi$ - случайная велличина на нем. Тогда \emph{функцией распределения} случайной величины $\xi$ называется
    \begin{align*}
      F_\xi (x) = P(\xi \leq x)
    \end{align*}
\end{definition}

\begin{definition}
  Случайная величина $\xi$ называется
  \begin{itemize}
    \item 
      дискретной, если её функция распределения дискретная.

    \item 
      абсолютно непрерывной, если её функция распределения абсолютно непрерывна. 
      В этом случае
      \begin{align*}
        P(\xi \leq x) = F_\xi (x) = \int_{-\infty}^{x} p_\xi (t)\, dt
      \end{align*}
      и функция $p_\xi (t)$ называется плотностью случайной величины $\xi$.\\

    \item
      сингулярной, если её функция распределения сингулярна

    \item
      непрерывной, если её функция рапределения непрерывна.
  \end{itemize}
\end{definition}

\begin{definition}
  Пусть $\xi = (\xi_1, \ldots, \xi_n)$ -- случайный вектор на $(\Omega, \setF, P)$. 
  Тогда его \emph{функцией распределения} называется 
  \begin{align*}
    F_\xi (x_1, \ldots, x_n) = P(\xi_1 \leq x_1, \ldots, \xi_n \leq x_n).
  \end{align*}
\end{definition}
\bigtitle{Порожденная $\sigma$-алгебра}

\begin{definition}
  Пусть $\xi$ - случайная величина на $(\Omega, \setF, P)$. 
  Тогда \emph{$\sigma$-алгеброй $\setF_\xi$, порожденной $\xi$} называется
  \begin{align*}
    \setF_\xi = \condset{\{ \xi \in B \}}{B \in B(\setR)}
  \end{align*}
\end{definition}

\begin{definition}
  Если $\xi$ -- случайный вектор размерности $n$ на $(\Omega, \setF, P)$, 
  то $\sigma$-алгеброй, порожденной $\xi$ называется
  \begin{align*}
    \setF_\xi = \condset{\{ \xi \in B \}}{B \in B(\setRn)}
  \end{align*}
\end{definition}

\subsection{Примеры конкретных распределений}

\begin{enumerate}
	\item 
		Дискретное равномерное $\mathcal{X} = \braces{1, \ldots, N},\;
		k = 1, \ldots, N$ и $p_k = 1 / N$ для $\forall k \in \mathcal{X}$.
	
	\item 
		Бернуллиевское
		\begin{align*}
			&\mathcal{X} = \{ 0, 1 \}, k = 0, 1\\
			&p_k = p^k (1 - p)^{1 - k},
		\end{align*}
		где $p \in [0, 1]$ - параметр.
		
	\item 
		Биномиальное распределение
		\begin{align*}
			&\mathcal{X} = \{ 0, \ldots , n\}\\
			&p_k = \comb{n}{k}\, p^k (1 - p)^{n - k},
		\end{align*}
		где $p \in [0, 1]$ - параметр.

	\item
		Пуассоновское распределение
		\begin{align*}
			&\mathcal{X} = \setZ_+\\
			&k = 0, 1, 2, \ldots \\
			&p_k = \frac{\lambda^{k}}{k!} e^{-\lambda}, \lambda > 0 -- \text{параметр}
		\end{align*}

		Моделирование: биномиальное $\rightarrow$ пуассоновское \\
		Число событий, произошедших за фиксированное время, при условии, что они происходят с некоторой фиксированной вероятностью и независимы.
	\item 
			Равномерное распределение на отрезке $[a, b]$.
			\begin{equation*}
				p(x) =
				\begin{cases}
					\dfrac{1}{b - a},&x \in [a, b]\\
					0				,&\text{иначе}
				\end{cases}		
			\end{equation*}
			
			\begin{equation*}
				F(x) =
				\begin{cases}
					0, &x < a\\
					\dfrac{x - a}{b - a}, &x \in [a, b]\\
					1, &x \geq b
				\end{cases}
			\end{equation*}
		
		\item 
			Нормальное распределение (с параметрами ($a, \sigma^2$))
			\begin{equation*}
				p(x) = \frac{1}{\sqrt{2\pi \sigma^2}} e^{-\frac{(x - a)^2}{2 \sigma^2}},\;
				a \in \setR, \sigma > 0
			\end{equation*}
			Моделирование: измерения величины $a$ = $a$ + ошибка измерения.
\end{enumerate}

\newpage
\section{Билет №3}
\subsection{Матожидание случайной величины: опр-ние для простых, неотрицательных, произвольних с.в.}
Пусть $(\Omega, \setF, P)$ -- вероятностное пространство, $\xi$ - случайная величина на нем. 
Что такое $E \xi$?

Простые случайные величины.

Пусть $\xi$ -- простая случайная величина, т.е. 
\begin{align*}
  \xi = \sum_{k = 1}^{n} x_k I_{A_k},
\end{align*}
где $x_1 \ldots x_n$ -- различные числа, $A_1, \ldots, A_n$ 
-- разбиение $\Omega$, т.е. $A_k = \{ \xi = x_k \}$

\begin{definition}
  Для простой случайной величины $\xi$ её математическим ожиданием называют
  \begin{align*}
    E\xi = \sum_{k = 1}^n x_k P(A_k)
  \end{align*}
\end{definition}

\begin{definition}
  Пусть $\xi$ -- неотрицательная случайная величина, 
  а $\{ \xi_n,\, n \in \setN \}$ -- $\forall$ последовательность неотрицательных 
  простых случайных величин, т.ч. $\xi_n \uparrow \xi$. 

  Тогда $E \xi_n \leq E \xi_{n + 1} \implies \exists$ предел $E \xi_n$ и
  \begin{align*}
    E \xi := \lim_{n \to \infty} E \xi_n
  \end{align*}
\end{definition}

\begin{definition}
  Пусть $\xi$ -- произвольная случайная величина, $\xi = \xi^+ - \xi^-$

  \begin{enumerate}
    \item 
      Если $E \xi^+$ и $E \xi^-$ -- конечны, то 
      $\boxed{E\xi := E \xi^+ - E \xi^-}$

    \item 
      Если $E \xi^+ = +\infty$ и $E \xi^-$ -- конечно, то $\boxed{E\xi := +\infty}$

    \item
      Если $E \xi^+$ конечно и $E \xi^- = +\infty$, то $\boxed{E\xi := -\infty}$

    \item 
      Если $E \xi^+$ = $E \xi^-$ $= +\infty$, то $E \xi \text{ не существует(не определено)}$

  \end{enumerate}
\end{definition}
\begin{remark}~
  \begin{enumerate}
    \item
      Математическое ожидание случайной величины это интеграл Лебега по вероятностной мере~$P$
      \begin{align*}
        E\xi := \int\limits_{\Omega} \xi dP = \int\limits_{\Omega} \xi(\omega) P(d\omega)
      \end{align*}

    \item
      $E\xi$ -- конечно $\iff E|\xi|$ -- конечно.

    \item
      Множество случ. величин $\xi$ на $(\Omega, \setF, P)$ с условием: $E\xi$ -- конечно, образует
      пространство $L^1(\Omega, \setF, P)$. Далее мы убедимся, что это линейное пространство.

  \end{enumerate}
\end{remark}

\subsection{Основные свойства матожидания (док-ва только для простых с.в.)}
\begin{enumerate}
  \item $\xi = c = const \implies E\xi = c$
  \item 
    Линейность
    \begin{align*}
      E(a\xi + b\eta) = a E \xi + b E \eta, \quad a, b \in \setR
    \end{align*}

    \begin{proof}
      Обозначим $\zeta = a\xi + b\eta$, пусть $\xi$ принимает значения $x_1 \ldots x_n$, 
      $\eta$ -- значения $y_1 \ldots y_m$, $\zeta$ -- значения $z_1 \ldots z_l$

      Обозначим $C_{k, j} = \{ \xi = x_k, \eta = y_j \}$.\\
      Тогда 
      \begin{align*}
        &E\zeta = \sum_{i = 1}^l z_i P(\zeta = z_i) = \sum_{i = 1}^l z_i 
        \sum_{\substack{k,j:\\ a x_k + b y_j = z_i}} P(\xi = x_k, \eta = y_j) =\\
        &\sum_{i = 1}^l \sum_{\substack{k,j:\\ a x_k + b y_j = z_i}} (a x_k + b y_j) 
        P(\xi = x_k, \eta = y_j) =\\
        &\sum_{k = 1}^n \sum_{j = 1}^m (a x_k + b y_j) P(\xi = x_k, \eta = y_j) =\\
        &\sum_{k = 1}^n a x_k P(\xi = x_k) + \sum_{j = 1}^m b y_j P(\eta = y_j) = a E \xi + b E \eta
      \end{align*}

    \end{proof}

  \item Если $\xi \geq 0$, то $E \xi \geq 0$
    \begin{proof}
      Если $\xi \geq 0$, то все $x_k \geq 0 \implies E \xi \geq 0$
    \end{proof}

  \item Если $\xi \leq \eta$, то $E \xi \leq E \eta$
    \begin{proof}
      Рассмотрим $\zeta = \eta - \xi \geq 0$. По свойству 3\\
        \begin{align*}
          0 \leq E \zeta = E (\eta - \xi) = E \eta - E \xi
        \end{align*}
    \end{proof}
\end{enumerate}
\subsection{Дисперсия, ковариация, их св-ва}

\begin{definition}
  \emph{Дисперсией} с.в. $\xi$ называют
  \begin{equation*}
    D\xi = E(\xi - E\xi)^2, \quad \text{если $E\xi$ существует}
  \end{equation*}
\end{definition}

\begin{definition}
  \emph{Ковариацией} случайных величин $\xi$ и $\eta$ называют
  \begin{align*}
    \cov(\xi, \eta) = E(\xi - E\xi) (\eta - E\eta)
  \end{align*}
  Если $\cov(\xi, \eta) = 0$, то $\xi$ и $\eta$ называются \emph{некоррелированными}.
\end{definition}

Если $D\xi$ и $D\eta$ -- конечны и положительны, то можно определить расстояние
\begin{align*}
  \rho(\xi, \eta) = \frac{\cov(\xi, \eta)}{\sqrt{D \xi D \eta}}
\end{align*}

которое называется \emph{коэффициентом корреляции} $\xi$ и $\eta$

\begin{lem}[свойства дисперсии и ковариации]~

  Если все математические ожидания конечны, то 
  \begin{enumerate}
    \item 
      Ковариация билинейна.

    \item 
      $cov(\xi, \eta) = E \xi \eta - E\xi E\eta$

      $D\xi = cov(\xi, \xi) = E\xi^2 - (E \xi)^2$

    \item 
      $D(c\, \xi) = c^2 D \xi, D(\xi + c) = D \xi$

    \item 
      Неравенство Коши-Буняковского.
      \begin{align*}
        |E\xi \eta|^2 \leq E\xi^2 E\eta^2
      \end{align*}

    \item 
      $|\rho(\xi, \eta)| \leq 1$, 
      причем $\rho(\xi, \eta) = 1 \iff \xi$ и $\eta$ -- п.н. линейно зависимы.
  \end{enumerate}

\end{lem}

\begin{proof}~

  Свойства $1) - 3)$ легко вытекают из свойств математического ожидания.

  \begin{enumerate}[start=4]
    \item 
      Рассмотрим для $\lambda \in \setR:$
      \begin{align*}
        f(\lambda) = E(\xi + \lambda \eta)^2 \geq 0
      \end{align*}
      Но $f(\lambda) = E \xi^2 + 2E \xi \eta \lambda + \lambda^2 E \eta^2 \geq 0$
      $\iff$ дискриминант $\leq 0$, т.е.
      $4 [(E \xi \eta)^2 - E \xi^2 E\eta^2] \leq 0$

    \item 
      Рассмотрим $\xi_1 = \xi - E \xi$, $\eta_1 = \eta - E \eta$

      Тогда $\cov(\xi, \eta) = E \xi_1 \eta_1,\quad D \xi = E \xi_1^2,\quad D \eta = E\eta_1^2$

      $\implies |\rho(\xi, \eta)| = 
      \walls{\frac{E \xi_1 \eta_1}{\sqrt{E \xi_1^2 E\eta_1^2}}} \leq 1$, 
      по нер-ву Коши-Буняковского.

      При этом $|\rho(\xi, \eta)| = 1 \iff$ дискриминант $= 0 
      \iff \exists ! \lambda_0 \in \setR$ т.ч. $f(\lambda_0) = 0$. 
      т.е. $E(\xi_1 + \lambda_0 \eta_1)^2 = 0$

      $\implies \xi_1 + \lambda_0 \eta_1 = 0$ п.н. т.е.
      \begin{align*}
        \xi = E \xi - \lambda_0 (\eta - E \eta) \text{ п.н.}
      \end{align*}
  \end{enumerate}
\end{proof}

\begin{corollary}
  Если $\xi_1, \ldots, \xi_n$ -- попарно некоррелируют, $D \xi_i < +\infty$, тогда
  \begin{align*}
    D(\xi_1 + \ldots \xi_n) = \sum_{k = 1}^{n} D\xi_k
  \end{align*}
\end{corollary}
\begin{definition}
  Пусть $\xi = (\xi_1, \ldots, \xi_n)$ -- случ. вектор.

  Тогда его \emph{мат. ожиданием} называется вектор из мат. ожиданий его компонент:
  \begin{align*}
    E \xi = (E\xi_1, \ldots, E\xi_n)
  \end{align*}
\end{definition}

\begin{definition}
  \emph{Дисперсией} вектора $\xi$ называется его матрица ковариаций:
  \begin{align*}
    D\xi = \begin{Vmatrix}\cov(\xi_i, \xi_j)\end{Vmatrix}_{i, j = 1}^{n}\; 
    \text{ --- матрица $n \times n$}
  \end{align*}

\end{definition}
\newpage
\section{Билет №4}
\subsection{Сходимость случайных величин: по вероятности, по распределению, почти наверное, в среднем}
\begin{definition}~

  \begin{enumerate}
    \item 
      Последовательность случайных величин $\{ \xi_n,\; n \in \setN \}$ 
      \emph{сходится по вероятности} к случайной величине $\xi$ 
      (обозначение $\xi_n \toup{p} \xi$), если для $\forall \epsilon > 0:$
      \begin{align*}
        P\pars{|\xi_n - \xi| \geq \epsilon} \todown{n \to \infty} 0
      \end{align*}

    \item 
      Последовательность случайных величин $\{ \xi_n,\; n \in \setN \}$ 
      \emph{сходится с вероятностью 1} к случайной величине $\xi$ 
      (или сходится \emph{почти наверное}), если 
      \begin{align*}
        P(\omega : \lim_{n \to \infty} \xi_n (\omega) = \xi(\omega)) = 1
      \end{align*}

      Обозначения: $\xi_n \toae \xi,\; \xi_n \to \xi \text{ п.н.}$ или 
      $\xi_n \to \xi\; P\text{-п.н.}$

    \item 
      Последовательность случайных величин $\{ \xi_n,\; n \in \setN \}$ 
      \emph{сходится в среднем порядка $p > 0$} к случайной величине $\xi$
      (или \emph{сходится в пространстве $L^p$}), если
      \begin{align*}
        E|\xi_n - \xi|^p \todown{n \to \infty} 0
      \end{align*}
      
      Обозначение: $\xi_n \toup{L^p} \xi$

    \item 
      Последовательность случайных величин $\{ \xi_n, n \in \setN \}$ 
      \emph{сходится по распределению} к случайной величине $\xi$, 
      если для $\forall$ ограниченой непрерывной ф-ции $f(x)$ выполнено
      \begin{align*}
        E f(\xi_n) \todown{n \to \infty} E f(\xi)
      \end{align*}
      
      Обозначение: $\xi_n \toup{d} \xi$\\
  \end{enumerate}
\end{definition}

\subsection{Связь между сходимостями (б/д). Теорема о наследовании сходимости}
\begin{theorem}[взаимоотношение различных видов сходимости]~

  Выполнены соотношение
  \begin{enumerate}
    \item $\xi_n \toae \xi \implies \xi_n \toup{P} \xi$
    \item $\xi_n \toup{L^P} \xi \implies \xi_n \toup{P} \xi$
    \item $\xi_n \toup{P} \xi \implies \xi_n \toup{d} \xi$
  \end{enumerate}
\end{theorem}

\begin{theorem}[Теорема о наследовании сходимости]~

Пусть $\{ \xi_n \}^\infty$ , $\xi$ - случаный вектор, рамезрности m.\\
1) Если $\xi_n \toae \xi$ и $f(x):\setR^m \rightarrow \setR^n$ - ф-ция, непр. п.н. относительно распределения $\xi$ (т.е $\exists B \in B(\setR^m)$ т.ч. f непрерывна на B и $P(\xi \in B) = 1$), то $f(\xi_n) \toae f(\xi)$\\
2) Если $\xi_n \toup{P} \xi$ и f(x) - ф-ция, непр. п.н. относительно распределения $\xi$, то $f(\xi_n) \toup{P} f(\xi)$\\
3) Если $\xi_n \toup{d} \xi$ и f(x) - непр. ф-ция, то $f(\xi_n) \toup{d} f(\xi)$\\
\end{theorem}
\begin{proof}
\begin{enumerate}[label=\protect\circled{\arabic*},series=inequalities]
\item 
$P(\lim f(\xi_n) = f(\xi)) \geq P( \lim \xi_n = \xi, \xi \in B) = 1 \implies f(\xi_n) \toae f(\xi)$
\item
Предположим, что $f(\xi_n) \nrightarrow^P  f(\xi) $. Тогда $\exists \epsilon_0 > 0$, $\delta_0 > 0$ и подпосл. Но $\xi_{n_k} \toup{P}$, тогда $\exists $ подпосл $\xi_{n_{k_l}}$ т.ч.$\xi_{n_{k_l}} \toup{P} \xi$, согласно 1 пункту   $f(\xi_{n_{k_l}}) \toae f(\xi) \implies f(\xi_{n_{k_l}}) \toup{P} f(\xi) $. Противоречие с выбором $\xi_{n_k}$ значит, $f(\xi_n) \toup{P} f(\xi)$\\
\item
Пусть $f(x):\setR^k \rightarrow \setR$ - непр, огр ф-ция. Тогда $h(f(x))$ - непр, огр ф-ция в $R^m \implies$ (тк $\xi_n \toup{d} \xi$) $Mh(f(\xi_n)) \toup Mh(f(\xi))$\\
\end{enumerate}
\end{proof}
\newpage



\section{Билет №5}
\subsection{Неравенства Маркова и Чебышева. ЗБЧ в форме Чебышева}
\begin{enumerate}[label=\protect\circled{\arabic*},series=inequalities]

  \item
    \bigtitle{Неравенство Маркова}

    Пусть $\xi \geq 0$ -- неотрицательная случайная величина. 

    Тогда для $\forall \epsilon > 0:$ \quad
    $
      \boxed{
        P(\xi \geq \epsilon) \leq \frac{E \xi}{\epsilon}
      }
    $

    \begin{proof}
        $P(\xi \geq \epsilon) = E\, I \{ \xi \geq \epsilon \} 
        \leq E\pars{\dfrac{\xi}{\epsilon}\, I \{ \xi \geq \epsilon \}}
        \leq E\pars{\dfrac{\xi}{\epsilon}} = \dfrac{E \xi}{\epsilon}$
    \end{proof}

  \item
    \bigtitle{Неравенство Чебышева}

    Если $D\xi < +\infty$, то для $\forall \epsilon > 0:$
    $
      \boxed{
        P(|\xi - E \xi| \geq \epsilon) \leq \frac{D\xi}{\epsilon^2}
      }
    $

    \begin{proof}
      \begin{align*}
        P(|\xi - E \xi| \geq \epsilon) = P(|\xi - E\xi|^2 \geq \epsilon^2) 
        \leq \expl{нер-во Маркова} \leq \frac{E\walls{\xi - E\xi}^2}{\epsilon^2} = 
        \frac{D\xi}{\epsilon^2}
      \end{align*}
    \end{proof}
\end{enumerate}

\begin{theorem}[Закон больших чисел в форме Чебышева]~

  Пусть $\{ \xi_n,\; n \in \setN \}$ -- последовательность попарно некоррелированных случайных величин, т.ч. $\forall n : D\xi_n \leq C$.

  Обозначим $S_n = \xi_1 + \ldots + \xi_n$. Тогда
  \begin{align*}
    \frac{S_n - E S_n}{n} \toup{P} 0, \quad n \to \infty
  \end{align*}

\end{theorem}

\begin{proof}
  \begin{align*}
    &P\pars{\walls{\frac{S_n - E S_n}{n}} \geq \epsilon} \leq \expl{нер-во Чебышева}
    \leq \frac{D\pars{\frac{S_n - E S_n}{n}}}{\epsilon^2} = 
    \frac{D(S_n - E S_n)}{n^2 \epsilon^2} =\\
    &= \frac{D S_n}{n^2 \epsilon^2} = \expl{$\xi_i$ и $\xi_j$ - некорр.}
    = \frac{\sum_{j = 1}^n D \xi_j}{n^2 \epsilon^2} \leq 
    \frac{C n}{n^2 \epsilon^2} \todown{n \to \infty} 0
  \end{align*}
\end{proof}


\bigtitle{Смысл ЗБЧ:}

$\xi_1 \ldots \xi_n \ldots$ -- результаты независимых проведений одного и того же эксперимента. 

Тогда их среднее арифметическое сходится к среднему значению результата одного эксперимента $E\xi_i$

Если $\xi_i$ -- индикаторы наступления некоторого события $A$:
\begin{align*}
  \xi_i = I \{ A \text{ наступило в $i$-м эксперименте}\}
\end{align*}

то
\begin{align*}
  \frac{\xi_1 + \ldots + \xi_n}{n} \toup{P} E \xi_i = P(A)
\end{align*}

Таким образом ЗБЧ --- это принцип устойчивости частот.

\subsection{УЗБЧ(все) (б/д)}
\begin{theorem}[Усиленный закон больших чисел в форме Колмогорова-Хинчина]~

  Пусть $\{ \xi_n,\; n \in \setN \}$ -- независимые с.в. т.ч. $D \xi_n < +\infty \forall n$.

  Пусть последовательность $\{ b_n,\; n \in \setN \}$ т.ч. $b_n > 0, b_n \uparrow +\infty$ и
  \begin{align*}
    \sum_{n = 1}^{\infty} \frac{D \xi_n}{b_n^2} < +\infty
  \end{align*}

  Обозначим $S_n = \xi_1 + \ldots \xi_n$.
  Тогда 
  \begin{align*}
    \boxed{\frac{S_n - E S_n}{b_n} \toae 0} \quad (\text{при } n \to \infty)
  \end{align*}
\end{theorem}
\begin{theorem}[Усиленный закон больших чисел в форме Колмогорова]~

  Пусть $\{ \xi_n,\; n \in \setN \}$ -- независимые одинаково распределенные случ. величины 
  (н.о.р.с.в), т.ч: $E |\xi_i| < +\infty$. 

  Тогда
  \begin{align*}
    \frac{\xi_1 + \ldots + \xi_n}{n} \toae m = E \xi_1
  \end{align*}
\end{theorem}

\newpage

\section{Билет №6}
\subsection{Характеристические функции с.в и векторов. Их св-ва}
\begin{definition}
  Характеристической функцией с.в. $\xi$ называется
  \begin{align*}
    \phi_{\xi} (t) = E e^{i t \xi}, \quad t \in \setR
  \end{align*}
\end{definition}

\begin{remark}
  Характеристическая функция, вообще говоря, явл. комплекснозначной. 
  Мы понимаем $E e^{i t \xi}$ как
  \begin{align*}
    E e^{i t \xi} = E \cos(t\xi) + i E \sin(t \xi)
  \end{align*}
\end{remark}

\begin{definition}
  Пусть $F(x),\; x \in \setR$ -- функция распределения на $\setR$\\
  Её характеристической функцией наз.
  \begin{align*}
    \phi(t) = \int\limits_{\setR} e^{i t \xi} dF(x)
  \end{align*}

  Если $P$ -- вероятностная мера на $(\setR, B(\setR))$, то её характеристической ф-ей наз.
  \begin{align*}
    \phi(t) = \int\limits_{\setR} e^{it\xi} P(dx)
  \end{align*}
\end{definition}

\begin{corollary}
  $\phi_\xi (t)$ -- х.ф. с.в. $\xi \iff  \phi_{\xi} (t)$ -- х.ф. $F_{\xi}(x)$
  $\iff \phi_{\xi} (t)$ -- х.ф. $P_{\xi}$ (распр. $\xi$)

  \begin{proof}
    \begin{align*}
      \phi_{\xi} (t) = E e^{i t \xi} = \int\limits_{\setR} e^{i t x} P_{\xi} (dx) 
      = \int\limits_{\setR} e^{i t x} dF_{\xi} (x)
    \end{align*}
  \end{proof}

\end{corollary}

\begin{definition}
  Пусть $\xi = (\xi_1, \ldots, \xi_n) $ -- случайный вектор.
  Его характеристической функцией наз.
  \begin{align*}
    \phi_{\xi} (t) = E e^{i \langle t, \xi \rangle}, \text{ где } t = (t_1, \ldots, t_n) \in \setRn, 
    \text{ а } \langle t, \xi \rangle = \sum_{i = 1}^{n} t_i \xi_i
  \end{align*}
\end{definition}

\begin{definition}
  Пусть $F(x), \; x \in \setR$ -- функция распр. в $\setRn$.

  Её х.ф. наз.
  \begin{align*}
    \phi(t) = \int\limits_{\setRn} e^{i \langle t, x \rangle} dF(x),\quad t \in \setRn
  \end{align*}

  Если $P$ -- вероятностная мера в $\setRn$ , то её х.ф. наз
  \begin{align*}
    \phi(t) = \int\limits_{\setRn} e^{i \langle t, x \rangle} P(dx),\quad t \in \setRn
  \end{align*}
\end{definition}

\begin{corollary}
  Если $\xi = (\xi_1, \ldots, \xi_n)$ -- сл. вектор, то
  $\phi_\xi(t)$ -- х.ф. $\xi \iff \phi_\xi(t)$ -- х.ф. $F_\xi (x), x~\in~\setRn
  \iff \phi_\xi (t)$ -- х.ф. $P_\xi$
\end{corollary}
\bigtitle{Основные свойства характеристических функций}

\begin{enumerate}[label=\protect\circled{\arabic*},series=charfunc_properties]
  \item
    Пусть $\phi(t)$ -- х.ф. с.в. $\xi$. 

    Тогда $|\phi(t)| \leq \phi(0) = 1, \; \forall t \in \setR$
    \begin{proof}
      \begin{align*}
        |\phi(t)| = |E e^{i t \xi}| \leq E |e^{i t \xi}| = 1 = \phi(0)
      \end{align*}
    \end{proof}

  \item 
    Пусть $\phi(t)$ -- хар. ф. с.в. $\xi$, а $\eta = a \xi + b, \; a, b \in \setR$.
    Тогда
    \begin{align*}
      \phi_{\eta} (t) = e^{i t b} \phi_{\xi} (t a)
    \end{align*}

    \begin{proof}
      \begin{align*}
        \phi_{\eta} (t) = E e^{i t \eta} = E e^{i t (a \xi + b)} 
        = e^{i t b} E e^{i (a t) \xi} = e^{i t b} \phi_\xi (at)
      \end{align*}
    \end{proof}

  \item
    Пусть $\phi(t)$ -- х.ф.с.в. $\xi$. 
    Тогда $\phi(t)$ равномерно непрерывна на $\setR$.

    \begin{proof}
      \begin{align*}
        |\phi(t + h) - \phi(t)| = \walls{E e^{i (t + h) \xi} - E e^{i t \xi}} 
        = \walls{E(e^{i(t + h)\xi} - e^{i t \xi})} = \walls{E(e^{i t \xi} (e^{i h \xi} - 1))}
        = E |e^{i h \xi} - 1|
      \end{align*}
      
      При $h \to 0, \; e^{i h \xi} - 1 \to 0$ п.н. 

      Кроме того, $E|e^{i h \xi} - 1| \leq 2 \implies$
      по теореме Лебега о мажорируемой сходимости:

      $E|e^{i h \xi} - 1| \todown{h \to 0} 0$ 
      $\implies \phi(t)$  равномерно непрерывна на $\setR$.
    \end{proof}

  \item
    Пусть $\phi(t)$ -- х.ф. с. в. $\xi$. Тогда $\phi(t) = \conj{\phi(-t)}$

    \begin{proof}
      \begin{align*}
        \phi(t) = E e^{i t \xi} = E e^{\conj{-i t \xi}} = \conj{E e^{-i t \xi}} = \conj{\phi(-t)}
      \end{align*}
    \end{proof}

  \item
    Пусть $\phi(t)$ -- х.ф. с.в. $\xi$. 
    Тогда $\phi(t)$ -- действительнозначная $\iff$ распределение $\xi$ 
    симметрично, т.е. $\forall B \in B(\setR)$
    \begin{align*}
      P(\xi \in B) = P(\xi \in -B)
    \end{align*}

    \begin{proof}~

      $(\impliedby)$ Пусть распр. $\xi$ -- симметрично. 
      Тогда $\xi \equp{d} -\xi \implies$
      \begin{align*}
        &E sin(t\xi) = E sin(-t\xi) = -E sin(t\xi)\\
        &\implies E sin(t\xi) = 0 \implies \phi(t) = E e^{i t \xi} = E cos(t \xi) \in \setR
      \end{align*}
      -- действительнозначная.\\

      $(\implies)$ Пусть $\phi(t) \in \setR, \, \forall t \in \setR$.
      Тогда по свойствам \circled{2} и \circled{4}.
      \begin{align*}
        \phi(t) = \phi_\xi (t) = \overline{\phi_\xi (-t)} = \phi_{\xi} (-t) = \phi_{-\xi} (t)
      \end{align*}
      т.е. у $\xi$ и у $-\xi$  одинаковая х.ф. 
      $\implies$ по теореме о единственности функции распр. $\xi$ и $-\xi$ совпадают.

      $\implies \xi \equp{d} -\xi$ и, значит, для $\forall B \in B(\setR):$
      \begin{align*}
        P(\xi \in B) = P(-\xi \in B) = P(\xi \in -B)
      \end{align*}
    \end{proof}

    \item
      Пусть $\xi_1, \ldots, \xi_n$ -- независимые с.в., $S_n = \xi_1 + \ldots + \xi_n$
      Тогда 
      \begin{align*}
        \phi_{S_n} (t) = \prod_{k = 1} \phi_{\xi_k} (t)
      \end{align*}

      \begin{proof}
        \begin{align*}
          &\phi_{S_n} (t) = E e^{i S_n t} = E e^{i \xi_1 t} \ldots e^{i \xi_n t} 
          = \expl{с.в независимы $\implies$ $e^{\text{с.в}}$ независимы} =\\
          &= \pars{E e^{i \xi t}} \ldots \pars{E e^{i \xi_n t}} 
          = \prod_{k = 1}^{n} \phi_{\xi_k} (t)
        \end{align*}
      \end{proof}

\end{enumerate}

\subsection{Теорема непрерывности (б/д)}
\begin{theorem}[непрерывности]~

  Пусть $\{ F_n(x),\; n \in \setN \}$ -- последовательность ф.р. на $\setR$, 
  а $\{ \phi_n(t),\; n \in \setN \}$ -- последовательность их х.ф.

  Тогда
  \begin{enumerate}
    \item
      Если $F_n \toup{w} F$, где $F(x)$ -- ф.р. на $\setR$, 
      то для $\forall t \in \setR: \phi_n(t) \to \phi(t)$ при $n \to \infty$, 
      где $\phi(t)$ - х.ф. $F(x)$

    \item
      Пусть для $\forall t \in \setR \quad \exists$ предел $\lim\limits_{n \to \infty} \phi_n (t)$,
      причем $\phi(t) = \lim\limits_{n \to \infty} \phi_n(t)$ непрерывна в нуле. 
      Тогда $\exists$ ф.р. $F(x)$ т.ч. $F_n \toup{w} F$ и $\phi(t)$ - х.ф. $F(x)$
  \end{enumerate}
 \end{theorem}
\newpage
\section{Билет №7}
\subsection{ЦПТ для незав. одинаково распр-х с.в.}
\begin{theorem}[Центральная предельная теорема]~

  Пусть $\{ \xi_n,\; n \in \setN \}$ -- 
  последовательность независимых одинаково распределенных с.в. т.ч. 
  $0 < D \xi_n < +\infty$. 

  Обозначим $S_n = \xi_1 + \ldots + \xi_n$
  Тогда
  \begin{align*}
    \frac{S_n - E S_n}{\sqrt{D S_n}} \toup{d} N(0, 1)
  \end{align*}

  \begin{proof}~

    Обозначим $a = E \xi_i, \sigma^2 = D \xi_i$. 
    Рассмотрим $\eta_i = \frac{\xi_i - a}{\sigma}
    \implies E \eta_i = 0, D \eta_i = E \eta_i^2 = 1$

    Тогда
    \begin{align*}
      T_n = \frac{S_n - E S_n}{\sqrt{D S_n}} = \expl{независимость}
      = \frac{S_n - na}{\sqrt{n} \sigma} = \frac{\eta_1 + \ldots \eta_n}{\sqrt{n}}
    \end{align*}

    Рассмотрим х.ф. $\eta_i:$
    \begin{align*}
      \phi_{\eta_i} (t) = \phi(t) = 1 + E \eta_i (i t) + 
      \frac{1}{2} E \eta_i^2 (it)^2 + \underset{(t \to 0)}{o(t^2)};
    \end{align*} 

    Отсюда получаем, что 
    \begin{align*}
      &\phi_{T_n} (t) = \phi_{\eta_1 + \ldots + \eta_n} (\frac{t}{\sqrt{n}}) 
      = \expl{независимость} = \pars{\phi\pars{\frac{t}{\sqrt{n}}}}^n 
      = \pars{1 - \frac{t^2}{2n} + o\pars{\frac{t^2}{n}}}^n 
      \todown{n \to \infty} e^{- \frac{t^2}{2}}
    \end{align*}
    Но $e^{-\frac{t^2}{2}}$ -- х.ф. $N(0, 1)$ 
    $\implies$ по теорема непрерывности мы получаем, что 
    \begin{align*}
      T_n = \frac{S_n - E S_n}{\sqrt{D S_n}} \toup{d} N(0, 1)
    \end{align*}
  \end{proof}
\end{theorem}

\begin{corollary}
  В условиях ЦПТ для $\forall x \in \setR$ выполнено
  
  \begin{align*}
    P\pars{\frac{S_n - E S_n}{\sqrt{D S_n}} \leq x} 
    \todown{n \to \infty} \int\limits_{-\infty}^{x} \frac{1}{\sqrt{2\pi}} e^{-\frac{y^2}{2}} dy
  \end{align*}

  \begin{proof}
    По ЦПТ $T_n = \frac{S_n - E S_n}{\sqrt{D S_n}} \toup{d} \xi \sim N(0, 1) 
    \iff F_{T_n} \implies F_{\xi}$, где $F_{\xi} (x)$  -- ф.р. $N(0, 1)$, т.е.
    $\forall x \in \setR:$
    \begin{align*}
      F_{T_n} \todown{n \to \infty} F_{\xi} (x) 
      = \int\limits_{-\infty}{x} \frac{1}{\sqrt{2 \pi}} e^{-\frac{y^2}{2}} dy
    \end{align*}
  \end{proof}
\end{corollary}

\begin{corollary}
  В условиях ЦПТ, если $E \xi_i = a, D \xi_i = \sigma^2$, то
  \begin{align*}
    \sqrt{n} \pars{\frac{S_n}{n} - a} \toup{d} N(0, \sigma^2)
  \end{align*}

  \begin{proof}
    \begin{align*}
      \sigma T_n = \sigma \frac{S_n - E S_n}{\sqrt{D S_n}} 
      = \sigma \frac{S_n - na}{\sqrt{n} \sigma} = \sqrt{n} \pars{\frac{S_n}{n} - a}
    \end{align*}

    Но $T_n \toup{d} N(0, 1) \implies \sigma T_n \toup{d} \sigma N(0, 1) = N(0, \sigma^2)$

    $\implies \sqrt{n} \pars{\frac{S_n}{n} - a} \toup{d} N(0, \sigma^2)$
  \end{proof}
\end{corollary}

\newpage
\section{Билет №8}
\subsection{Выборка, выборочное пр-во.}
\begin{definition}
Пусть X - наблюдение (результат случайных экспериментов), тогда множество всех возможных значений X называется \emph{выборочным пространством}
\end{definition}
\begin{definition}
\emph{Вероятностно-статистическая модель} - тройка\\
$(X, B_x, P)$, где X - выборочное пространство, $B_x$ - $\sigma$ алгебра на X, P - класс распределения вероятностной меры на $(X, B_x)$
\end{definition}

\begin{definition}
Если $X = ( X_1, ... ,X_n )$, где $X_1, ... ,X_n$ - н. о. р. с. в. с распределение P, то X - \emph{выборка} размера n из распределения P.
\end{definition}
\subsection{Точные оценки параметров и их св-ва: смещенность, состоятельность, асимптотическая нормальность}
\begin{definition}
Если P - параметризовано, т.е. P = $ \lbrace P_\theta, \theta \in \Theta \rbrace$, причем $P_{\theta_1} \neq P_{\theta_2}$ при $\theta_1 \neq \theta_2$, то модель - \emph{параметрическая}
\end{definition}
\begin{definition}
Пусть $(X, B_x, P)$ - вер-стат. модель. X - наблдение, a $( E , \varepsilon )$ - измеримое пространство. Пусть $ S: X \rightarrow E$ - измеримое отображение (т.е. $\forall B  \in \varepsilon $ $  S^{-1}(B) = \lbrace x \in X : S(x) \in B \rbrace \in B_x$). Тогда S(x) - \emph{статистика}
\end{definition}
\begin{definition}
Если  P = $ \lbrace P_\theta, \theta \in \Theta \rbrace$ - парематрическая модель, S принимает значения    , в $\Theta$, то S(X) можно назвать \emph{оценкой} $\Theta$
\end{definition}
Свойства оценок: X - наблюдение с распределение $P \in  \lbrace P_\theta, \theta \in \Theta \rbrace,  \Theta \in \setR^k$ \\
a) $\emph{несмещенность} : \forall \theta \in \Theta $   $M_\theta \theta^*(X)=\theta$\\
б) $\emph{состоятельность} : \forall \theta \in \Theta $   $ \theta^*_n(X_1, ... , X_n) \xrightarrow[n \rightarrow \infty]{P_\theta} \theta$\\
в)	$\emph{асимптотическая нормальность} : \forall \theta \in \Theta $   $ \sqrt{n}(\theta^*_n - \theta) \xrightarrow{d_\theta} \mathcal{N}(0, \sigma^2(\theta))$
\subsection{Выборочные среднее, медиана, дисперсия}
\subsection{Сравнение оценок, ф-ция потерь, ф-ция риска}
\begin{definition}
Пусть $\rho(x,y) \geq 0$ - борелевская функция, тогда \emph{функцией потерь} оценки $\theta^*(x)$ называется $\rho(\theta^*(x), \theta)$
\end{definition}
\begin{definition}
Если задана функция потерь $\rho(x,y)$, то \emph{функцией риска} оценки $\theta^*(x)$ называется $R(\theta^*(x), \theta)=M_\theta\rho(\theta^*(x), \theta)$
\end{definition}
\subsection{Подходы к сравнению оценок: равномерный, байесовский, минимаксный}
\begin{enumerate}[label=\protect\circled{\arabic*},series=charfunc_properties]
  \item
  	\bigtitle{Равномерный подход}
  	Оценка $\theta^*(x)$ лучше оценивает $\theta$, чем $\hat{\theta}(x)$, если $\forall \theta \in \Theta R(\theta^*(x) , \theta) \leq R(\theta^*(x) , \theta)$ и для некоторого $\theta \in \Theta$ неравенствое строгое.\\
  	Оценка $\theta^*(x)$ называется наилучшей в классе K, если она лучше $\forall$ другой оценки из K.
  \item
  	\bigtitle{Байесовский подход}
  	Пусть Q - распределение веротяностей на $\Theta$. Тогда $\forall$ оценки $\theta^*(x)$ введем $R_q(\theta^*(x)) = \int\limits_\Theta R(\theta^*(x),t)Q(dt)$\\
  	Оценка $\theta^*(x)$ называется наилучшей в байесовском подходе, если $R_q(\theta^*(x)) = \inf R_q(	\hat{\theta}(x))$
  \item
  	\bigtitle{Минимаксный подход}
  	Для оценки $\theta^*(x)$ введем $\rho(\theta^*(x))=\underset{\theta \in \Theta}{\sup}R(\theta^*(x),\theta)$\\
  	Оценка $\theta^*(x)$ называется наилучшей в минмаксном подходе, если $\rho(\theta^*(x)) =  \underset{\hat{\theta}}{\inf} \rho(\hat{\theta}(x)) $ 
\end{enumerate}  
\newpage
\section{Билет №9}
\subsection{Методы построения оценок: метод моментов и метод максимального правдоподобия}
\begin{definition}
Метод моментов: пусть $(x_1, ... x_n)$ - выборка из распределения $ P \in \{ P_\theta, \theta \in \Theta \subset \setR^k \}$
\end{definition}
\subsection{Состоятельность оценки метода моментов}
\subsection{Теорема о св-вах оценок максимального правдоподобия (б/д)}
\newpage
\section{Билет №10}
\subsection{Доверительные интервалы. Метод центральной статистики}
\newpage
\section{Билет №11}
\subsection{Статистические гипотезы, ошибки первого и второго рода}
\subsection{Общие принципы сравнения критериев, авномерно наиболее мощные критерии}
\subsection{Лемма Неймана-Пирсона. Построение с ее помощью наиболее мощных критериев}


\end{document}